\section{The Mayer-Vietoris Sequence}\label{sec-mayer-vietoris}
\subsection{Statement and proof}
In the previous section, we relied heavily on identifying homomorphisms in a long exact sequence as either isomorphisms of the $0$-map. This is not the case in general, and it can feel like the axioms give very little help with identifying homology groups and their maps when they are not trivial. However, in many cases we can cover a space $X$ by the interiors two spaces $A$ and $B$ whose homology groups we know. In this case we say that $(A,B)$ is a Mayer-Vietoris cover of $X$. The excision axiom gives us a very convenient method for relating the homology groups of $X$ to the homology groups of its Mayer-Vietoris cover:

\begin{theorem}\label{mayer-vietoris}
Let $A$ and $B$ be subsets of $X$ whose interiors cover $X$. 
Then there is a long exact sequence:

% https://tikzcd.yichuanshen.de/#N4Igdg9gJgpgziAXAbVABwnAlgFyxMJZARgBoAGAXVJADcBDAGwFcYkQAJAfWDAGpiAXwAaIQaXSZc+QigBMFanSat23MAAoAggB0dAY3poABACEAlGIkgM2PASIBmRTQYs2iTlzDGtxvQBGWADmEGgscMbqZlaSdjJEACwuyu5q3sai4nHSDijkKW6qnnpQEDgI2TZS9rLIAKyFKh4gpeWVSjBQwfBEoABmAE4QALZIySA4EEjkVUOjMzRTSGSpxa06aPSDeEw8-EKxIPNjiKvLiAprLRpYAHoAVKQAVo+Wc8OnVxfO1+wA1o8ALSMR5HE5IX4XeqCSiCIA
\[\begin{tikzcd}
\dots \arrow[r] & H_{n+1}X \arrow[r, "\partial_{n+1}"] & H_n(A\cap B) \arrow[r, "{(i^*,j^*)}"] & H_n A \bigoplus H_n B \arrow[r, "k^*-l^*"] & H_n X \arrow[r] & \dots
\end{tikzcd}\]
called the \defn{Mayer-Vietoris sequence}. The maps $i^*,j^*,k^*,l^*$ are induced by the inclusions:
% https://tikzcd.yichuanshen.de/#N4Igdg9gJgpgziAXAbVABwnAlgFyxMJZABgBoBGAXVJADcBDAGwFcYkQBBAHS4GN60AAgBCIAL6l0mXPkIoyxanSat2HcZJAZseAkXIUlDFm0QhREqTtn7SimsdVmAGuKUwoAc3hFQAMwAnCABbJDIQHAgkA2UTdiwNfyDQxHDIpAAmBxVTEAArRJBAkOiadMQAZmy4swBrQuKUrIioyuqnEEY3MSA
\[\begin{tikzcd}
A \arrow[r, "k"]                      & X                \\
A\cap B \arrow[u, "i"] \arrow[r, "j"] & B \arrow[u, "l"]
\end{tikzcd}\]


If $A\cap B\neq \emptyset$, there is furthermore a long exact sequence of reduced homology, called the \defn{reduced Mayer-Vietoris sequence}: 

% https://tikzcd.yichuanshen.de/#N4Igdg9gJgpgziAXAbVABwnAlgFyxMJZARgBoAGAXVJADcBDAGwFcYkQAdDvR2YACQC+AfWBgA1MUEANEINLpMufIRQAmCtTpNW7Ljz5DhYABQBBLgGN6aAAQAhAJRyFIDNjwEiAZk00GLGyInNxYvDACImC2ZrZcAEZYAOYQaCxwtvzGDi6KHipEACx+2oF6oeGR2bLyecpeKOQlAbrBXFAQOAi1bkqeqsgArM06QSEdXXJaMFBJ8ESgAGYAThAAtkjFIDgQSOQ9K+t7NDtIZKWtIWj0y3hMohJSuSCHG4jnp4gaF2MmWAB6ACpSAArIHOA6rN7fT6+H7sADWQIAtIwgc9Xkg4Z9BoJKIIgA
\begin{tikzcd}
\dots \arrow[r] & \tilde{H}_{n+1}X \arrow[r, "\partial_{n+1}"] & \tilde{H}_n(A\cap B) \arrow[r, "{(i^*,j^*)}"] & \tilde{H}_n A \bigoplus H_n B \arrow[r, "k^*-l^*"] & \tilde{H}_n X \arrow[r] & \dots
\end{tikzcd}
By Proposition \ref{reduced-homology}, the maps $i^*,j^*,k^*,l^*$ can be thought of as induced by the inclusions
% https://tikzcd.yichuanshen.de/#N4Igdg9gJgpgziAXAbVABwnAlgFyxMJZABgBoBGAXVJADcBDAGwFcYkQAKAQQB0eBjemgAEAIVIAPAJQgAvqXSZc+QijLFqdJq3bdJM+Yux4CRchU0MWbRJ3HS5CkBmMqzpDTSs7bHABr6cpowUADm8ESgAGYAThAAtkhkIDgQSOZa1uxYjtFxiYjJqUgATF7aNiAAVrkgsQnpNMWIAMzlWbYA1rX1BWUpaa3tPiCMQbJAA
\[\begin{tikzcd}
{(A,x)} \arrow[r, "k"]                      & {(X,x)}                \\
{(A\cap B,x)} \arrow[u, "i"] \arrow[r, "j"] & {(B,x)} \arrow[u, "l"]
\end{tikzcd}\]
\end{theorem}

To prove this theorem, we follow the approach laid out in \cite{Eilenberg}, first proving a lemma. 

\begin{lemma}
Consider the following diagram.
% https://tikzcd.yichuanshen.de/#N4Igdg9gJgpgziAXAbVABwnAlgFyxMJZARgBoAGAXVJADcBDAGwFcYkQBxAfXJAF9S6TLnyEUAJlLFqdJq3bdxAcn6CQGbHgJFJAZhkMWbRJy7FVQzaKJkALAbnHT5FQMsjtKclIdGFZ1zUNDzFkb30aQ3kTDnELdWEtULJxX2jOfhkYKABzeCJQADMAJwgAWyRvEBwIJFtIx3ZGM3iS8qQAVhoapF0GvxMACxa3EDaKxCqexDJZAZBmuNHxpElq2sQ+ufThpbUVye6NgDZ+9KweVtKJ0-Xes6cAK0vl69WjpFuopwvzV-bELdpvVtk8Rvs3ogundAQ92Bc9kVIUCNrNvuxnoixpC1tM0Y0TABrLhYg7Q4FwokjSh8IA
\[\begin{tikzcd}
                                                         & G_0 \arrow[ld, "l_1"] \arrow[rd, "l_2"] \arrow[dd, "i_0"] &                                                           \\
G_1'                                                     &                                                           & G_2'                                                      \\
                                                         & G \arrow[dd, "j_0"] \arrow[lu, "j_1"] \arrow[ru, "j_2"]   &                                                           \\
G2 \arrow[rd, "h_1"] \arrow[ru, "i_2"] \arrow[uu, "k_1"] &                                                           & G_1 \arrow[ld, "h_2"] \arrow[lu, "i_1"] \arrow[uu, "k_2"] \\
                                                         & G_0'                                                      &                                                          
\end{tikzcd}\]
Suppose each triangle is commutative, $k_1,k_2$ are isomorphisms, and the diagonals are exact sequences. Then $h_2k_2^{-1}l_2=-h_1k_1^{-1}l_1$.
\end{lemma}

\begin{proof}
MISSING
%By lemma \ref{dsum-lemma}, $(i_1,i_2):G_1\dsum G_2\rightarrow G$ is an isomorphism, and so is $(j_1,j_2):G\rightarrow(G'_1,G'_2)$.
\cite{Eilenberg}
\end{proof}


\begin{proof}[Proof of Theorem \ref{mayer-vietoris}]

MISSING \cite{Eilenberg}
\end{proof}
\subsection{Examples}

In this section we show a few examples of how convenient the Mayer-Vietoris sequence can be for calculating homology groups.

\begin{example}[One-point wedges of circles]
Consider the figure eight $8=S^1\wedge S^1$ where $\wedge$ is the wedge product identifying the south-pole $S$ of the top circle with the north-pole $N$ of the lower circle. We can cover $8$ by the puncture $A=8\setminus \{N\}\homotopic S^1$, where $N$ is the north-pole of the top circle, and $B=D^+\homotopic \bullet$, the upper half circle of the top circle. Then $A\cap B= D^+\setminus\{N\}\homotopic \bullet \sqcup \bullet$. It is easy to see that this cover satisfies Mayer-Vietoris. Now note that $$H_n(\bullet \sqcup \bullet)=\begin{cases}\mathbb{Z}\times\mathbb{Z} & n=0\\ 0 & \text{otherwise}\end{cases}$$ by Proposition \ref{disjoint-union}. Since $\tilde{H}_0S^1\dsum \tilde{H}_0\bullet=0$, the reduced Mayer-Vietoris sequence reads 

% https://tikzcd.yichuanshen.de/#N4Igdg9gJgpgziAXAbVABwnAlgFyxMJZABgBpiBdUkANwEMAbAVxiRGJAF9T1Nd9CKAIzkqtRizYAdKQFs6OABYAjZcABanLjxAZseAkQBMo6vWatEIGXgaxgACU4B9IQA5tvfQKIBmU+IW0nIKKmqanrp8BoLIACwB5pJWHNxe-IYoAKyJEpbWUrb2Ts7EHmlR3pnIAGy5QSlcYjBQAObwRKAAZgBOELJIZCA4EEhCFb39Y9QjSCaBySAA1gB6AFSRkwOI87OI-gv5Mmh0PXiMm33bB3txE1dICcOjiFn3U68zLzWcFJxAA
\[\begin{tikzcd}
0 \arrow[r] & \mathbb{Z} \arrow[r, "k^*"] & \tilde{H}_18 \arrow[r, "\partial"] & \mathbb{Z} \arrow[r] & 0 \arrow[r] & \tilde{H}_08 \arrow[r] & 0
\end{tikzcd}\]
Therefore $\tilde{H}_08=0$, so by Proposition \ref{reduced-homology}, $H_08=\mathbb{Z}$. Furthermore, note that the fold $f:8\rightarrow S^1$, which is the identity on the lower circle, and the reflection on the upper circle, is a retraction of the inclusion $k:S^1\rightarrow 8$ of the lower circle. Since $H_n$ is a functor, $f^*$ is a retraction of $k^*$. By Proposition \ref{retraction-iso}, $H_18\iso \mathbb{Z}\times\mathbb{Z}$. The rest of the Mayer-Vietoris sequence easily gives $H_n8=0$ whenever $n\neq 0,1$.

This approach can easily be generalised to the wedge of $m$ circles (wedged at the same point), . We cover 


This argument can easily be extended by induction to show the wedge of $m$ circles (wedged at the same point) $\bigwedge_m S^1$ has homology groups $H_1\bigwedge_m S^1=\mathbb{Z}^m$ and $H_0\bigwedge_m S^1=\mathbb{Z}$. Let $A=\bigwedge_{m} S^1\setminus\{x\}\homotopic \bigwedge_{m-1} S^1$, the puncture of some point in one of the circles (not the wedge point), and $B=I\homotopic \bullet$, a small interval of the punctured circle containing the puncture in its interior and not intersecting the wedge point. Then $A$ and $B$ easily satisfy Mayer-Vietoris, and $A\cap B\homotopic \bullet \sqcup \bullet$. By induction, the reduced Mayer-Vietoris sequence reads:
% https://tikzcd.yichuanshen.de/#N4Igdg9gJgpgziAXAbVABwnAlgFyxMJZABgBpiBdUkANwEMAbAVxiRGJAF9T1Nd9CKAIzkqtRizYAdKQFs6OABYAjZcABanAHrAATLIC0uzlx4gM2PASK7R1es1aIQMvA1jAAEpwD6QgBymvJYCRADMduKO0nIKKmqaQeZ8VoLIACyRDpLOHNzB-NYoAKxZEk4uUm4e3j7EgfnJIUXIAGxl0blcYjBQAObwRKAAZgBOELJIZCA4EEhCjWMT89SzSLZROSAA1loAVElLk4gba4gRmxUyaHSjeIyH48cXZ+mLT0iZM3OIxe-Lv1WP1anAonCAA
\[\begin{tikzcd}
0 \arrow[r] & \mathbb{Z}^{m-1} \arrow[r, "k^*"] & \tilde{H}_18 \arrow[r, "\partial"] & \mathbb{Z} \arrow[r] & 0 \arrow[r] & \tilde{H}_08 \arrow[r] & 0
\end{tikzcd}\]

This shows $\tilde{H}_0 \bigwedge_m S^1=0$. Again  $k:S^1\rightarrow \bigwedge_m S^1$ admits a retraction $r:\bigwedge_m S^1\rightarrow S^1$ which is the identity on every circle except the one that got punctured, and which folds the circle that was punctured onto any other circle. Therefore Proposition \ref{retraction-iso} gives that $H_1 \bigwedge_m S^1\iso \mathbb{Z}^{m}$.
\end{example}

\begin{example}
Consider the quotient of the torus $T^2:= S^1 \times S^1/\sim$ where $(N,x)\sim (N,y)$ for all $x,y \in S^1$. It can be visualised as in the MISSING image.

A cover that can easily be seen to satisfy Mayer-Vietoris is $A=D^1_+ \times S^1 \homotopic \bullet$, $B=(T^2/\sim) \setminus\{[N,x]\}\homeo S^1$, with $A\cap B \homeo S^1\sqcup S^1$, as in the MISSING image. The deformation retraction $B\homotopic S^1$ composes with the inclusion $i:A\cap B\rightarrow B$ such that $i|_A=i|_B =id_{S^1}$. We expect that inclusion induces the map $i^*:\mathbb{Z}\times \mathbb{Z}\rightarrow \mathbb{Z}, (x,y)\mapsto x+y$, and indeed this can be shown from the following commutative diagram, where the inclusions $i_1,i_2$ are as one expects:

% https://tikzcd.yichuanshen.de/#N4Igdg9gJgpgziAXAbVABwnAlgFyxMJZARgBpiBdUkANwEMAbAVxiRAGUA9YgAgB0+cAI4BjJmh5diIAL6l0mXPkIoyABiq1GLNlNnyQGbHgJE1pAEyb6zVog7d9C48qIXL17XYfSZmmFAA5vBEoABmAE4QALZI5iA4EEhkWrZsWE4gkTFI7glJiPE2OvZYAPrS1Ax0AEYwDAAKiiYqIBFYgQAWOJnZsYgAzNSJcdTF3uUWvVH9Q-nJY17pUCBVtfVNLqb27V09cuEzucMFKePLshQyQA
\[\begin{tikzcd}
                                         & S^1                           &                                          \\
                                         & S^1 \sqcup S^1 \arrow[u, "i"] &                                          \\
S^1 \arrow[ru, "i_1"'] \arrow[ruu, "id"] &                               & S^1 \arrow[lu, "i_2"] \arrow[luu, "id"']
\end{tikzcd}\]

It induces the following commutative map in homology (for n=0,1), from which the formula for $i^*$ can be easily read. 

% https://tikzcd.yichuanshen.de/#N4Igdg9gJgpgziAXAbVABwnAlgFyxMJZARgBpiBdUkANwEMAbAVxiRAB12BbOnACwBGA4AC0Avpzxd4AAk49+Q0WJBjS6TLnyEUZAAxVajFm3m9Bw8avUgM2PASJ7SAJkP1mrRB27mlVtQ17bSIXV3djLx8FC2VVQxgoAHN4IlAAMwAnCC4kZxAcCCQyI082LGsM7NzEMIKixHyPE28ACgAPUj0AShBqBjoBGAYABU0HHRBMrCS+HEqQLJykAGZqQrzqZqjWgE8u3sDF6tX1hpLttl2+kAGh0fGQ72nZ+aOlmrqNxAvItnb4mIgA
\[\begin{tikzcd}
                                                   & \mathbb{Z}                                 &                                                    \\
                                                   & \mathbb{Z}\times \mathbb{Z} \arrow[u, "i"] &                                                    \\
\mathbb{Z} \arrow[ru, "{(x,0)}"'] \arrow[ruu, "x"] &                                            & \mathbb{Z} \arrow[lu, "{(y,0)}"] \arrow[luu, "y"']
\end{tikzcd}\]

The fact that $i_1^*(x)=(x,0)$ and $i_2^*(y)=(0,y)$ follows from Proposition \ref{disjoint-union}.

The Mayer-Vietoris sequence reads as follows:

% https://tikzcd.yichuanshen.de/#N4Igdg9gJgpgziAXAbVABwnAlgFyxMJZABgBpiBdUkANwEMAbAVxiRGJAF9T1Nd9CKAIzkqtRizYAJAPoAmAAQANLjxAZseAkTmjq9Zq0QgAOiYC2dHAAsARreAAtTmbzn4Cs5Zv2nnVbyaAkQAzHrihmxeVnYOzgHqfFqCyAAs4QaSxrJCKtyB-NooAKwZEkamJlAQOAicYjBQAObwRKAAZgBOEOZIZCA4EEhC+SBdPcPUg0i6EVmVaHSdeIzyCeO9iLPTiGFzFVgAegBUALwAHgDUAJ7r3Zt7O+n7bADWJ3cTiM87pS-GZkWyywqyEXAonCAA
\[\begin{tikzcd}
0 \arrow[r] & H_2 X \arrow[r, "\partial_2"] & \mathbb{Z}\times \mathbb{Z} \arrow[r, "i^*=x+y"] & \mathbb{Z} \arrow[r, "k^*"] & H_1X \arrow[r, "\partial_1"] & \dots
\end{tikzcd}\]

Now $ker(i^*)=<(x,-x)>\iso \mathbb{Z}=im(\partial_2)$ by exactness. Since $\partial_2$ is injective, $H_2 X\iso \mathbb{Z}$. Additionally, since $i^*$ is surjective, $k^*=0$. Therefore, the rest of the sequence reads:

% https://tikzcd.yichuanshen.de/#N4Igdg9gJgpgziAXAbVABwnAlgFyxMJZABgBpiBdUkANwEMAbAVxiRGJAF9T1Nd9CKAIzkqtRizYAJAPpCABAA0uPEBmx4CRAEyjq9Zq0QgAOiYC2dHAAsARreAAtTmbzn48s5Zv2nnFbwaAkQAzHrihmxeVnYOzgFqfJqCyAAs4QaSxrLEytyB-FooAKwZEkbsXGIwUADm8ESgAGYAThDmSGQgOBBIQvkgre191D1IuhFZpiZodC14jHIJQx2IE2OIYZMVWAB6AFQAvAAeANQAnsttq1sb6dtsANYHV8OI9xvFnBScQA
\begin{tikzcd}
0 \arrow[r] & H_1 X \arrow[r, "\partial_1"] & \mathbb{Z}\times \mathbb{Z} \arrow[r, "i^*=x+y"] & \mathbb{Z} \arrow[r, "k^*"] & H_0X \arrow[r] & 0
\end{tikzcd}

This diagram is similar to the previous diagram, giving us $H_1 X=\mathbb{Z}$ and $H_0 X= 0$. $H_nX =0$ for all other values of $x$.
\end{example}