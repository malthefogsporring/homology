The Borsuk-Ulam theorem is a fascinating fixed-point result, which can be proven via cellular homology and a technical homotopy remark. It states that every odd map $f:S^n\rightarrow \mathbb{R}^n$ maps two antipodal points to the same point. A popular interpretation of this statement is that there are always two antipodal points on Earth with the same temperature and barometric pressure. This follows from Borsuk-Ulam, where we are assuming the Earth is a 2-sphere, and temperature and barometric pressure are continuous functions on the surface of Earth (although both of these assumptions are dubious).

Instead of proving Borsuk-Ulam, we will prove the following stronger result:
\begin{theorem}\label{strong-borsuk-ulam}
Odd maps $f:S^n\rightarrow S^n, n\geq 1$ have odd degree.
\end{theorem}

Before we examine the proof, let us first see how this implies Borsuk-Ulam.

\begin{corollary}[The Borsuk-Ulam Theorem]
For every odd map $f:S^n\rightarrow \mathbb{R}^n$ there exists $x\in S^n$ such that $f(x)=-f(-x)$.
\end{corollary}
\begin{proof}
Suppose $f$ is odd but has no such point. Then $g:S^n\rightarrow \mathbb{R}^n$ defined as $g(x)=f(x)-f(-x)$ is never zero. We can therefore define $$h(x)=\frac{g(x)}{|g(x)|}:S^n\rightarrow S^{n-1}.$$ $h|_{S^{n-1}}$ has no fixed points, and is therefore homotopic to the antipodal map by Corollary \ref{fixed-points}. However, the restriction $h|_{D_+^n}$ of $h$ to the upper hemisphere of $S^n$ is null-homotopic, as $D_+^n$ is contractible, via the homotopy
$$j(x,t):D^n\times I\rightarrow S^n$$
$$(x,t)\mapsto h(\tilde{j}(x,t))$$
where $\tilde{j}$ is the homotopy $\tilde{j}:id_{D^n}\homotopic c$, and $c$ is a constant function. Since the restriction of a homotopy is also a homotopy, $j|{S^{n-1}\times I}$ defines a homotopy $h|_{S^{n-1}}\homotopic c\not\homotopic (-id)$, a contradiction.
\cite{Hatcher}
\end{proof}

We will now look at Theorem \ref{strong-borsuk-ulam}. The proof will involve the diagram induced by $f^*$ on the homology sequence of $(S^n,S^{n-1})$ to itself, where we are yet to show that $f^*$ induced such a diagram. It will become clear that we are only interested in whether $f^*$ and the associated induced maps have odd or even degree. We can therefore greatly simplify matters by replacing every instance of $\mathbb{Z}$ with $\mathbb{Z}/2\mathbb{Z}$ and every map $f^*:\mathbb{Z}\rightarrow \mathbb{Z}$ with $f^* \mod 2:\mathbb{Z}/2\mathbb{Z}\rightarrow \mathbb{Z}/2\mathbb{Z}$, as then $f^* \mod 2=id$ iff $f$ has odd degree, and $f^* \mod 2=0$ iff $f$ has even degree. We can employ a purely category theory argument for why this is possible for CW-complexes.



%Consider the chain complex for a CW-complex $X$, $C^*_{X;\mathbb{Z}}$ where we specify the coefficients as $\mathbb{Z}$:
% https://tikzcd.yichuanshen.de/#N4Igdg9gJgpgziAXAbVABwnAlgFyxMJZARgBoAGAXVJADcBDAGwFcYkQAJAfTAAoANAHphSQ4GAC0xAL4BuADryAtvRwALAEYbgALWkBKENNLpMufIRQAmCtTpNW7buKnSBglzNEfJVuYpV1LV0DIxMQDGw8AiIAZlsaBhY2RBBFKAgcBGNTKIsicgT7ZPZ0zOy7GCgAc3giUAAzACcIJSRCkBwIJDJix1SoHjDGlrbEXq6kGz6UkEHPaWGQZtakeM7uxA6k-rmucQBqGSNKaSA
%\[\begin{tikzcd}
%\dots \arrow[r, "d_{n+1}"] & {H_n(X^n,X^{n-1};\mathbb{Z})} \arrow[r, "d_n"] & %{H_{n-1}(X^{n-1},X^{n-2};\mathbb{Z})} \arrow[r, "d_{n-1}"] & \dots
%\end{tikzcd}\]

We have a category of \defn{chain complexes} of CW-complexes with coefficients in an abelian group $\mathbb{G}$, called $C^*(G)$. The maps between cellular chain complexes are the maps $f^*$ induced by induced by cellular maps $f:X\rightarrow Y$, such that $f(X^n)\subseteq Y^n$, as these then induce maps $f^*:H_n(X^n,X^{n-1})\rightarrow H_n(Y^n,Y^{n-1})$ for all $n$.

Consider the category of left $R$-modules, $R-\mathbf{Mod}$, for some ring $R$. Any ring homomorphism $f:R\rightarrow S$ into a subring $S\subset R$ induces a functor $F:R-\mathbf{Mod}\rightarrow S-\mathbf{Mod}$ $$rM\mapsto f(r)M$$ and
$$(g:rM\rightarrow sN)\mapsto fg:f(r)M\rightarrow f(s)N.$$ This functor furthermore extends to a functor $F:C^*(R)\rightarrow C^*(S)$, which maps a chain complex with coefficients $R$ to a chain complex with coefficients $S$, where every $R^n$ is mapped to $S^n$, and every $d:R^n\rightarrow R^m$ maps to $fd:S^n\rightarrow S^m$, the map that first identifies $S^n\subset R^n$, applies $d$, and then applies $f$ on each component of the image in $R^m$.

A specific case is $R=\mathbb{Z}$, $S=\mathbb{Z}/2\mathbb{Z}$, $f:\mathbb{Z}\rightarrow \mathbb{Z}/2\mathbb{Z}$ $$x\mapsto x\Mod 2.$$ This induces a Functor $F:C^*(\mathbb{Z})\rightarrow C^*(\mathbb{Z}/2)$ which replaces every $\mathbb{Z}^n$ with $(\mathbb{Z}/2)^n$, and every $d:\mathbb{Z}^n\rightarrow \mathbb{Z}^m$ with $d\Mod 2:(\mathbb{Z}/2)^n\rightarrow (\mathbb{Z}/2)^m$ defined in the obvious way. 

The induced functor $F:\mathbb{Z}-Mod\rightarrow \mathbb{Z}/2-Mod$ takes $C^*_{X;\mathbb{Z}}$ to $C^*_{X;\mathbb{Z}/2}$, which maps $\mathbb{Z}^m$ to $(\mathbb{Z}/2)^m$ and $f:\mathbb{Z}^m\rightarrow \mathbb{Z}^m$ to $f\Mod 2:(\mathbb{Z}/2)^m\rightarrow (\mathbb{Z}/2)^n$, thought of as applying $\Mod 2$ to every component of $f$.

\begin{proposition}
If $f:S^n\rightarrow S^n$ has degree $m$, then the induced map in $\mathbb{Z}/p\mathbb{Z}$ coefficients, where $p$ is a prime, is multiplication by $m \Mod p$.
\end{proposition}
\begin{proof}
MISSING
\end{proof}

Our proof of Theorem \ref{strong-borsuk-ulam} the approach in \cite{borsuk}. We will need the following technical result.

\begin{theorem}[Cellular approximation]\label{cellular-approximation}
Every map $f:X\rightarrow Y$ is homotopic to a cellular map $g$, that is a map such that $g(X^n)\subseteq Y^n$ for each $n$. If $g$ is already cellular on a subcomplex of $X$, the homotopy can be taken to be relative on the subcomplex.
\end{theorem}
\begin{proof}
Omitted. See \cite{Hatcher}.
\end{proof}



\begin{proof}[Proof of Theorem \ref{strong-borsuk-ulam}]
We prove this by induction on $n$. For $n>0$, an odd map $f:S^{n-1}\rightarrow S^{n-1}$ has odd degree if and only if it induces an isomorphism. We can take this to be the induction property, and note that it hold trivially for $n=0$.  Suppose the statement holds for $n-1$. We can give $S^n$ a cell structure with $1$ $0$-cell and $1$ $n-1$-cell corresponding to the equatorial sphere $S^{n-1}$, and $2$ $n$-cells corresponding to the upper and lower hemispheres of $S^n$, glued by the identity map on the boundary onto $S^{n-1}$. Note this is not the usual cell-structure of $S^n$ with $1$ $0$-cell and $1$ $n$-cell. By Theorem \ref{cellular-approximation}, $f\homotopic g$ where $g$ is cellular.$f^*=g^*$, $f^*$ inherits the properties of both an odd map and a cellular map. We may therefore assume $f$ is both odd and cellular.

$f(S^{n-1})\subseteq S^{n-1}$, $f$ gives a map from the homology sequence of $(S^n,S^{n-1}$ to itself. This looks as follows: 

% https://tikzcd.yichuanshen.de/#N4Igdg9gJgpgziAXAbVABwnAlgFyxMJZAJgBoAGAXVJADcBDAGwFcYkQAJAfTAAoBlAHphSQ4GAC0ARgC+AShAzS6TLnyEUAZgrU6TVu27jpMsZNmLlIDNjwEiAFh00GLNohAAdT1Ag4ESiq26kRSznpuhjwABEKEgdaqdhrI5OGuBh7klkFq9ihpUroZ7iDZCTZ5KU5FLvql3r7+OYnB+chhtRGZnDFxLZXJRGRdJVF8caKCxrIKFUkhWqSj9VEzptPmMoq6MFAA5vBEoABmAE4QALZITiA4EEiaCedXjzT3SGndpVgtL9eIL4fRBhb7sbxoehnPBMP4XAGg4HEZ7wpAAVneD0QAHYUa8cZikAAOOqRDy-PEAkl3LEATlJPQhUJhjDh+PpNKQADZKW9OQSwR4ToIAFRsgFArHUsZC0XipCIukMhqeLBwB4ySgyIA
\begin{tikzcd}
0 \arrow[r] & H_n S^n \arrow[r, "i"] \arrow[d, "f^*"] & {H_n(S^n,S^{n-1})} \arrow[r, "\partial"] \arrow[d, "f^*"] & H_{n-1}S^{n-1} \arrow[r] \arrow[d, "\iso"] & \dots \\
0 \arrow[r] & H_n S^n \arrow[r, "i"]                  & {H_n(S^n,S^{n-1})} \arrow[r, "\partial"]                  & H_{n-1}S^{n-1} \arrow[r]                   & \dots
\end{tikzcd}

The homology groups $H_n(S^n,S^{n-1}$ are $\mathbb{Z}/2\mathbb{Z} \times \mathbb{Z}/2\mathbb{Z}$ with generators corresponding to the upper and lower hemispheres, as $S^n$ is a CW-complex with two $n$-cells glued onto a copy of $S^{n-1}$.

In coefficients, the above diagram becomes:
% https://tikzcd.yichuanshen.de/#N4Igdg9gJgpgziAXAbVABwnAlgFyxMJZAJgBoAGAXVJADcBDAGwFcYkQAdDgW3pwAsARoOAAtAL4B6Yl14DhY8Vzzd4AAll8hIidM3yd4kONLpMufIRQBmCtTpNW7ABIB9YGAC0ARnEBlAD0PHyMTM2w8AiIAFjsaBhY2RE4OKAgcBDCQDAjLIm84h0T2fW1FPR4tBQljU2zzSKtkckKEp2TyWvCLKJQW73s2pJBOrJyeptiB+MdhrjSMrvrc3uQC6aL2lLky3RlKg0Ul8caiMg2hkoPdqX2d6qUOFXVSh4r7w2OGvJtSC9mXO4vL5AsFfMZ7DAoABzeBEUAAMwAThBuEhYiAcBAkNYssjUTiaFikC1NsMsEt8WjEKTiYgCmSrmh6Ei8ExKSjqQy6cQ8ZykABWInYxAAdj5BLFwqQAA4ZsVkhSJdS5ZiRQBOeVbLjM1lYdnKpCatVIABshsQthNUsZyQRAQAVBzJbSRarLnbHc6udLEMaPSksHBseJKOIgA
\begin{tikzcd}
0 \arrow[r] & \mathbb{Z}/2\mathbb{Z} \arrow[r, "i"] \arrow[d, "f^*"] & \mathbb{Z}/2\mathbb{Z}\times \mathbb{Z}/2\mathbb{Z} \arrow[r, "\partial"] \arrow[d, "f^*"] & H_{n-1}S^{n-1} \arrow[r] \arrow[d, "\iso"] & \dots \\
0 \arrow[r] & \mathbb{Z}/2\mathbb{Z} \arrow[r, "i"]                  & \mathbb{Z}/2\mathbb{Z}\times \mathbb{Z}/2\mathbb{Z} \arrow[r, "\partial"]                  & H_{n-1}S^{n-1} \arrow[r]                   & \dots
\end{tikzcd}

$f^*\neq (0,0)$ by commutativity as $\partial$ is injective, hence not $0$. $f$ commutes with the antipodal map by oddness: $f(-id)=(-id)f$. It follows that $f^*(-id)^*=(-id)^*f^*$. Note that the antipodal map also gives a map between these sequences, and that it swaps the hemispheres (REFERENCE): $(-id)^*(x,y)=(y,x).$ So $f^*(x,0)=(-id)^*f^*(0,y)$. Since $f^*\neq (0,0)$, $f^*(x,0)=(x,0)$ or $(0,x)$ and $f^*(0,y)$ is the other one. So $f^*$ is an isomorphism.

By commutativity in the left square, since $i$ is injective, and $f^*i$ is injective, $f^*:S^n\rightarrow S^n$ is also injective, hence is an isomorphism. This implies the degree $m$ of $f:S^n\rightarrow S^n$ is s.t. $m mod 2=1$, i.e. $m$ is odd.



%There are few options for $\partial$. It cannot be the zero map, as it is surjective. That leaves $\partial(x,y)=x,y$ or $x+y$. However, note that the antipodal map also gives a map between these sequences, and that it swaps the hemispheres (REFERENCE). Therefore we have the following commutative diagram, which disallows $\partial(x,y)=x,y$. 

% https://tikzcd.yichuanshen.de/#N4Igdg9gJgpgziAXAbVABwnAlgFyxMJZABgBpiBdUkANwEMAbAVxiRAB12BbOnACwBGA4AC0AvgHoATJx78hosZzxd4AAlm9Bw8dM3ydYkGNLpMufIRQBGclVqMWbfdsV7uWheOOmQGbHgERGTW9vTMrIgcHgZuMjGu4spYqnAaCV6S8XKJRiZmAZZEtqHU4U5RLpnuOZnG9jBQAObwRKAAZgBOEFxIZCA4EEi2DhHO7Gh0nXiMPh3dvYhS1INIAMxljpHRk9NYs-kgXT19K0NLm2NRABQAnqQAHgCUc0cLw2frlxUgWFD1YiAA

%Therefore $\partial(x,y)=x+y$. $f^*:\mathbb{Z}/2\mathbb{Z}\rightarrow \mathbb{Z}/2\mathbb{Z}$ is not the zero map, as $id \partial\neq 0$. Furthermore, $f^*$ commutes with the antipodal map by oddness: $f(-id)=(-id)f$. It follows that $f^*(-id)^*=(-id)^*f^*$. So $f^*(x,0)=(-id)^*f^*(0,y)$ 
\cite{borsuk}
\end{proof}