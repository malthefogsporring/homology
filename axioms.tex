\section{Axioms}\label{sec-axioms}
\subsection{The Eilenberg-Steenrod axioms}
The axioms for a homology theory were first laid out by Eilenberg and Steenrod in \cite{Eilenberg}. In this section, we will follow a simplified treatment given in \cite{Werndli}. We define $\Top{2}$ to be the category of pairs of topological spaces $(X,A)$, where $A,B\in \mathbf{Top}$ and $A\subseteq X$. Maps $\Top{2}((X,A),(Y,B)$ are continuous maps $f:X\rightarrow Y$ such that $f(A)\subseteq B$. It is not hard to see this space satisfies the definition of a category.

\begin{lemma}
$\Top{2}$ is a category.
\end{lemma}
\begin{proof}
It is easy to see that the identity map $id_X$ satisfies the requirement for each $(X,A)$, as $id_X(A)=A$. Furthermore, if $f\in \Top{2}((X,A),(Y,B)$ and $g\in \Top{2}((Y,B),(Z,C)$, then $gf$ is continuous and $gf(A)= g(f(A))\subseteq g(B)\subseteq C$ as required. The associativity and identity properties are satisfied because $\mathbf{Top}$ is a category.
\end{proof}

We can similarly define $\Top{3}$ as the triples of topological spaces $(X,A,B)$ where $B\subseteq A \subseteq X$ and where morphisms $f:(X,A,B)\rightarrow (Y,C,D)$ are maps $f:X\rightarrow Y$ such that $f(A)\subseteq C$ and $f(B)\subseteq D$. We identify $X\in \mathbf{Top}$ with $(X,\emptyset)\in \Top{2}$, and $(X,A)\in \Top{2}$ with $(X,A,\emptyset)\in \Top{3}$.

\begin{definition}
A subcategory $\cat{C}\subset \Top{2}$ is \defn{admissible for homology} if the following apply:
\begin{enumerate}[(a)]
\item $C$ contains all points in $\mathbf{Top}$. In the language of category theory, $\cat{C}$ contains all \defn{final objects} in $\mathbf{Top}$, that is, all objects $\bullet\in\mathbf{Top}$ with the property that there is one and only one morphism from any $X\in\mathbf{Top}$ to $\bullet$.
\item For any $(X,A)\in \cat{C}$, the following commutative diagram lies in $\cat{C}$, where all maps are induced by canonical inclusions:
% https://tikzcd.yichuanshen.de/#N4Igdg9gJgpgziAXAbVABwnAlgFyxMJZABgBoBGAXVJADcBDAGwFcYkQAKAHS5gFs0OAJ5wYOUj36CRYgJQgAvqXSZc+QinIVqdJq3YcAghN4DhonPKUrseAkQBMpYjoYs2iTgA0TU83MVlEAxbdUdSB1c9D05jQysgkLV7FABmbRo3fU8OH3jAm2SNZAAWDN13Ax8vKx0YKABzeCJQADMAJwg+JDIQHAgkcmsQDq7Bmn6kB2HR7sQtPoHEVJnOufTFpBLVscQnTcRtoNmtiaWAVgVKBSA
\[\begin{tikzcd}
                                  &                                       & {(X,\emptyset)} \arrow[rd] &                   &         \\
{(\emptyset,\emptyset)} \arrow[r] & {(A,\emptyset)} \arrow[ru] \arrow[rd] &                            & {(X,A)} \arrow[r] & {(X,X)} \\
                                  &                                       & {(A,A)} \arrow[ru]         &                   &        
\end{tikzcd}\]
Furthermore, for any $f\in \cat{C}((X,A),(Y,B))$, $\cat{C}$ contains all the canonical maps induced by $f$ on the above diagram to the corresponding diagram of $(Y,B)$.
\item For $(X,A)\in \cat{C}$, $\cat{C}$ contains the following diagram:
% https://tikzcd.yichuanshen.de/#N4Igdg9gJgpgziAXAbVABwnAlgFyxMJZABgBpiBdUkANwEMAbAVxiRAAoANUgQQEoQAX1LpMufIRQBGclVqMWbLgB1leALbwABAEleqjdp0DBcmFADm8IqABmAJwjqkZEDghIZIAEYwwUJABaAGZXemZWRDcAfSkQagY6XwYABTE8AjZ7LAsACxwhERAHJxdqd09qcMUonGjieJ8-AMRQ00EgA
\[\begin{tikzcd}
{(X,A)} \arrow[r, "\tau_1"', bend right] \arrow[r, "\tau_0", bend left] & {(X\times I,A\times I)}
\end{tikzcd}\]
Where $\tau_t(x)=(x,t)$.
\end{enumerate}
\end{definition}

\begin{remark}
As noted in \cite{Werndli}, the definition certifies that
\begin{enumerate}
    \item $\cat{C}$ contains all final objects in $\Top{2}$, that is all maps $(\bullet,\emptyset)\rightarrow (X,A)$, where $\bullet$ is some fixed one-point set in $\cat{C}$. This is because $\cat{C}$ contains all maps $(\bullet,\emptyset)\rightarrow (X,\emptyset)$ and also the inclusion $(X,\emptyset)\rightarrow (X,A)$,
    \item $\cat{C}$ contains all homotopies $h:f\homotopic g$, for $f,g\in \cat{C}((X,A),(Y,B))$. By (iii), we can identify homotopies as maps $h:(X\times I,A\times I)\rightarrow (Y,B)$ such that $h\tau_0=f$ and $h\tau_1=g$:
% https://tikzcd.yichuanshen.de/#N4Igdg9gJgpgziAXAbVABwnAlgFyxMJZABgBpiBdUkANwEMAbAVxiRAAoANUgQQEoQAX1LpMufIRQBGclVqMWbLgB1leALbwABAEleqjdp0Dho7HgJEATLOr1mrRBwCapAEIm5MKAHN4RUAAzACcIdSQyEBwIJBkQACMYMCgkAFoAZkj7RSccAH0pEGoGOkSGAAUxC0kQYKwfAAscIREQELCI6mjYuwVHKLziIoSklMRM0zbQ8MQ47sQbeQc2Bpag6c6omIXqROSkABYATl7lp0C1qY7ESPnFvbHU49OckB9hkrLK8wk2OsbmoIKIIgA
\[\begin{tikzcd}
{(X,A)} \arrow[r, "t_1"', bend right] \arrow[r, "t_0", bend left] \arrow[rr, "f", bend left=49] \arrow[rr, "g"', bend right=49] & {(X\times I,A\times I)} \arrow[r, "h"] & {(Y,B)}
\end{tikzcd}\]
\end{enumerate}

\end{remark}

In this text, we will assume all spaces and maps are admissible unless otherwise stated. We will also use "space" to denote "topological space". 

We now give the axioms of an (ordinary) homology theory for an admissible category.

\begin{definition}
An \defn{ordinary homology theory} on an admissible category $\cat{C}$ is a family of functors $(H_n:\cat{C}\rightarrow \mathbf{Ab})_{n \in\mathbb{Z}}$ to the category of Abelian groups\footnote{The original definition defines functors into an \defn{abelian category}, which generally speaking is a category where homomorphisms can be added and wehere we can define the kernel and image of a homomorphism. In this text we will only deal with the category of abelian groups.} $\mathbf{Ab}$ and a family of natural transformations $\partial_n:H_n\rightarrow H_{n-1}\circ p$, where $p$ is the functor sending $(X,A)$ to $(A,\emptyset)$ and $f:(X,A)\rightarrow (Y,B)$ to $f|_A^B:(A,\emptyset)\rightarrow(B,\emptyset)$. We will often write $f^*$ for $H_n f$, $\partial$ for $\partial_n$ and $H_nX$ for $H_n(X,\emptyset)$, as it is usually obvious what role they play. $H_n$ and $\partial$ are assumed to hold the following axioms:

\begin{enumerate}[(a)]
\item (Homotopy invariance) If $f\homotopic g$, then $f^*=g^*$.
\item (Long exact sequence) The inclusions 

% https://tikzcd.yichuanshen.de/#N4Igdg9gJgpgziAXAbVABwnAlgFyxMJZABgBpiBdUkANwEMAbAVxiRAAoBBUgHR5gC2aHAE84MHAEoQAX1LpMufIRQBGclVqMWbdgA1e-IaPFTZ8kBmx4CRAEwbq9Zq0QcDnaTM0woAc3giUAAzACcIASQyEBwIJHUtFzYscxDwyMQE2KQHRJ03ACtZChkgA
\[\begin{tikzcd}
{(A,\emptyset)} \arrow[r, "i"] & {(X,\emptyset)} \arrow[r, "j"] & {(X,A)}
\end{tikzcd}\]
give rise to a long exact sequence

% https://tikzcd.yichuanshen.de/#N4Igdg9gJgpgziAXAbVABwnAlgFyxMJZARgBoAGAXVJADcBDAGwFcYkQAJAfWDAGpiAXwAUADVIBBAJQhBpdJlz5CKAEwVqdJq3bcwE2fJAZseAkQDMGmgxZtEnLmFGGFp5UQAs1rXd1OxSRk5NyVzFHIfWx0HV2NFMxVkAFYo7XsQWU0YKABzeCJQADMAJwgAWyRvEBwIJHIQkFKK+ppapDJfGJAAHR60ehK8Jh5+ITjmysRO9sR1LoysAD0AKgmyqfnZqwX2ACtV9ZbEHdnkwUpBIA
\[\begin{tikzcd}
{} \arrow[r] & {H_{n+1}(X,A)} \arrow[r, "\partial_{n+1}"] & H_nA \arrow[r, "i^*"] & H_nX \arrow[r, "j^*"] & {H_n(X,A)} \arrow[r] & {}
\end{tikzcd}\]

\item (Excision) If $U\subset A\subset X$ is open in $X$ and satisfies $\overline{U}\subseteq int(A)$, then the inclusion $(X\setminus U,A\setminus U)\rightarrow (X,A)$ gives rise to an isomorphism $H_n(X\setminus U,A\setminus U)\iso H_n(X,A)$.
\item (Dimension) For any one-point set $\bullet$, $$H_n\bullet=\begin{cases}G & n=0\\0&\text{otherwise},\end{cases}$$ where $G$ is some fixed abelian group.
\end{enumerate}
\end{definition}

\begin{remark}\label{homotopy-isomorphism}
As $H_n$ is a functor, it preserves commutative diagrams. That is, any commutative diagram of spaces in $\Top{2}$ gives rise to a commutative diagram in homology. Furthermore $id^*=id$ by the identity property of functors. Together with axiom (a), this asserts that homology theory can be used to distinguish between homotopy equivalent spaces: if $f:X\rightarrow Y$ is a homotopy equivalence with homotopy inverse $g:Y\rightarrow X$, then as $fg\homotopic id$, $f^*g^*=(fg)^*=id$, so $f^*$ is surjective. Similarly as $gf\homotopic id$, $g^*f^*=id$, so $f^*$ is injective. Hence $f^*:H_nX\rightarrow H_nY$ is an isomorphism.
\end{remark}
The extra dimension axiom is what defines an ordinary homology theory as opposed to a (general) homology theory. It is essential in our study. The choice $H_0=G$, called a "choice of coefficients" distinguishes homology theories from each other.

\subsection{Basic results}
\begin{prop}
If $A\subset X$ is a deformation retract, then $H_n(X,A)=0$.
\end{prop}
\begin{proof}
If $A\subset X$ is a deformation retract, then the inclusion $i:A\rightarrow X$ is a homotopy equivalence. By Remark \ref{homotopy-isomorphism}, $i:H_nA\rightarrow H_nX$ is an isomorphism. Now consider the homology sequence for $(X,A)$:
% https://tikzcd.yichuanshen.de/#N4Igdg9gJgpgziAXAbVABwnAlgFyxMJZABgBpiBdUkANwEMAbAVxiRAAkB9MAQRAF9S6TLnyEUARnJVajFmy5gAGgKEgM2PASIAmadXrNWiDtwAUS0jwCUq4ZrFEAzPtlGFnYGAC0E-n0F7UW0UABZXQ3kTLi9vHX4VfhkYKABzeCJQADMAJwgAWyQpEBwIJD03KJAAKwA9ACo7EFyC8upSpBdK4xAAHV60Ohy8RiaWwsQyErLEYsie-qw4MsDmvImujsRw7rZF5YEKfiA
\[\begin{tikzcd}
H_nA \arrow[r, "\iso"] & H_nX \arrow[r, "j^*"] & {H_n(X,A)} \arrow[r, "\partial"] & H_{n-1}A \arrow[r, "\iso"] & H_{n-2}X
\end{tikzcd}\]

By exactness, $ker(j^*)=H_nX$, so $j^*=0$. Similarly, $im(\partial)=0$, so $\partial=0$. However $im(j^*)=0=ker(\partial)$, so $H_n(X,A)=0$. \cite{Werndli}
\end{proof}

\begin{remark}
\label{contractible-0}
As a special case of this result, $H_n(X,X)=0$, as $X$ is a deformation retract of itself. We are also interested in special case where $A=x$ is a single point, i.e. $X$ is contractible. Then we have $H_nX\iso H_n\bullet=\mathbb{Z}$ and $H_n(X,x)=0$.
\end{remark}

Next, we would like to show how to calculate the homology of a disconnected space.

\begin{lemma}\label{dsum-lemma}
Consider the following commutative diagram.

% https://tikzcd.yichuanshen.de/#N4Igdg9gJgpgziAXAbVABwnAlgFyxMJZABgBoAmAXVJADcBDAGwFcYkQBBEAX1PU1z5CKMsWp0mrdgCEefEBmx4CRcqTE0GLNohDSA5HP5KhqiuK1TdHQ72OCVKAIyknFyTpAANHuJhQAc3giUAAzACcIAFskMhAcCCQAFk0PdiwjEAjo5JoEpDUJbXYAK0zsmMQ4-MQAZlTi3VDyyMqXeMTEQstPAJAaRnoAIxhGAAUBZWEQRhhQnBacxHaalKKrECxbeQrcjqR69c8S20puIA
\[\begin{tikzcd}
B \arrow[rr, "g"] \arrow[rd, "i'"] &                                    & B' \\
                                   & X \arrow[ru, "j"] \arrow[rd, "j'"] &    \\
A \arrow[ru, "i"] \arrow[rr, "f"]  &                                    & A'
\end{tikzcd}\]

If $f$ and $g$ are isomorphisms, and the diagonals are exact, then there exist isomorphisms $(i+i'):A\dsum B\rightarrow X$ and $(j',j):X\rightarrow A'\dsum B'$.
\end{lemma}
\begin{proof}
By commutativity, $ji'=g,$ so $ji'g^{-1}=id_{B'}$. Therefore $i'g^{-1}$ is a section of $j$. By Proposition \ref{short-seq-direct-sum}, we have an isomorphism $$(i+i'g^{-1}):A\dsum B'\rightarrow X$$
$$(a,b)\mapsto i(a)+i'g^{-1}(b)$$
Since $g^{-1}$ is an isomorphism, $(i+i'):A\dsum B\rightarrow X$ is also an isomorphism in an obvious way.

Similarly, $f^{-1}j'$ is a retraction of $i$. Again, by Proposition \ref{short-seq-direct-sum}, we have an isomorphism $$(f^{-1}j',j):X\rightarrow A\dsum B'$$
$$x\mapsto (f^{-1}j'(x),j(x))$$
which since $f^{-1}$ is an isomorphism also gives an isomorphism 
$$(j',j):X\rightarrow A'\dsum B'$$
\end{proof}


\begin{prop}\label{disjoint-union}
The inclusions $i_A:A\rightarrow A\sqcup B,j_B:B\rightarrow A\sqcup B$ induce an isomorphism $(i_A^*+i_B^*):H_n A\dsum H_n B \rightarrow H_n(A\sqcup B)$.
\end{prop}
\begin{proof}
Consider the following diagram

% https://tikzcd.yichuanshen.de/#N4Igdg9gJgpgziAXAbVABwnAlgFyxMJZABgBoAmAXVJADcBDAGwFcYkQAKAIVIB1eYAWzQ4AnnBg4AlCAC+pdJlz5CKchWp0mrdhwAapAIIz5i7HgJEAjKSuaGLNok4H+QkeMkmFIDOZVEZMT22k6chnwCwmIS0nI+fsqWaqTBNA46zvqkXCaaMFAA5vBEoABmAE4QgkhkIDgQSOpajuxYAPpc8eVVNYgAzDQNTemhbe2G3SCV1SP1jYgALKOtzgBWnVMzfc3DiDYtmSAbk6bTvbVDCwcZYYVbFwNXSMuHYWVylLJAA
\[\begin{tikzcd}
{(A,\emptyset)} \arrow[rd, "i_A"] \arrow[rr, "f"] &                                                     & {(X,B)} \\
                                                  & {(X,\emptyset)} \arrow[ru, "j_B"] \arrow[rd, "j_A"] &         \\
{(B,\emptyset)} \arrow[ru, "i_B"] \arrow[rr, "g"] &                                                     & {(X,A)}
\end{tikzcd}\]

It induces the following commutative diagram in homology, which satisfies the previous lemma, as the diagonals are part of the exact sequences of $(X,A)$ and $(X,B)$ respectively, and $f^*$ and $g^*$ are isomorphisms by the excision axiom.
% https://tikzcd.yichuanshen.de/#N4Igdg9gJgpgziAXAbVABwnAlgFyxMJZABgBoAmAXVJADcBDAGwFcYkQAJAfTACEQAvqXSZc+QinIVqdJq3bcwACgAapAIIBKQcJAZseAkQCMpYzIYs2iTjxU6RB8UTLELc67bDqHe0YYlkKTcaS3kbRVVSXm0BGRgoAHN4IlAAMwAnCABbJDIQHAgkKVkrdiwuXgA9ACpfTJykAGYaQuLQj3KudVr6rNzEErbEABYOspsAK0reoXT+9oKixFNS8JBpnrq5kAaB-OHVsM9E2d095tblsbXPNNnKASA
\[\begin{tikzcd}
H_nA \arrow[rd, "i_A^*"] \arrow[rr, "f^*"] &                                              & {H_n(X,B)} \\
                                           & H_nX \arrow[ru, "j_B^*"] \arrow[rd, "j_A^*"] &            \\
H_nB \arrow[ru, "i_B^*"] \arrow[rr, "g^*"] &                                              & {H_n(X,A)}
\end{tikzcd}\]
The result then follows from the lemma.
\cite{Werndli}
\end{proof}


\subsection{Reduced homology}
Results in homology are easy when the associated homology sequence has maps that are isomorphisms or $0$-maps. Therefore, if it is possible to "simplify" a homology sequence by for example replacing some groups with $0$, this is often advantageous. As we will show, it is often possible to factor out $H_n\bullet$ from a homology sequence of $(X,A)$, resulting in a simpler sequence. Furthermore, this transformation is reversible.

\begin{definition}
For non-empty $X$, $\tilde{H}_nX:=ker(p^*:H_nX\rightarrow H_n\bullet)$ where $p^*$ is induced by the initial map $p:X\rightarrow \bullet$. $\tilde{H}_nX$ is called the \defn{$n$-th reduced homology group} of $X$.
\end{definition}

\begin{prop}
\label{reduced-homology}
For non-empty $X$,
$$H_nX\iso\tilde{H}_nX\dsum H_n\bullet$$
and for any $x\in X$ $$\tilde{H}_nX\iso H_n(X,x)$$
\end{prop}
\begin{proof}
Consider the homomorphism $p^*:H_nX\rightarrow H_n\bullet$


Note $p^*$ is a retraction of $i^*:H_n\bullet \rightarrow H_n X$ induced by the inclusion, as $pi:\bullet\rightarrow \bullet$ is trivially the identity, and $H_n$ is a functor. By Corollary \ref{retraction-iso}, $$H_nX\iso ker(p^*)\dsum im(i^*)\dsum \iso \tilde{H}_nX\dsum im(i^*).$$ If we can show $i^*$ is injective, we have our first equality. By the naturality of $\partial$, the following diagram commutes:

% https://tikzcd.yichuanshen.de/#N4Igdg9gJgpgziAXAbVABwnAlgFyxMJZABgBpiBdUkANwEMAbAVxiRAAkB9YMAagEYAvgAoAGqQA6EgMZQIOAJQhBpdJlz5CKMvyq1GLNlx4CRU2fMky5i5apAZseAkX6ld1es1aIOnMOY2dmpOmq7kel6GvlwB1vLKejBQAObwRKAAZgBOEAC2SGQgOBBIAMyeBj4gUmh02XiMwSA5+UhuxaWIAEyV3my19Y0Mza0FiEUl7X3RDgB6AFSjueMVnUi9+v2+aIuJgkA
\[\begin{tikzcd}
{H_{n+1}(X,\bullet)} \arrow[r, "\partial"] \arrow[d, "p^*"] & H_n\bullet \arrow[d, "p^*"] \\
{H_{n+1}(\bullet,\bullet)} \arrow[r, "\partial"]              & H_n\bullet                 
\end{tikzcd}\]
Note $p^*:H_n\bullet\rightarrow H_n\bullet$ is the identity, as $p:\bullet\rightarrow\bullet$ is the identity. Therefore $\partial$ factors through $H_{n+1}(\bullet,\bullet)=0$. By the exactness of the homology sequence for $(X,\bullet)$, $0=im(\partial)=ker(i^*)$, so $i^*$ is injective as required, and $$H_nX\iso \tilde{H}_nX\dsum H_n\bullet.$$


For the second isomorphism, note that the long exact sequence of $(X,\bullet)$
% https://tikzcd.yichuanshen.de/#N4Igdg9gJgpgziAXAbVABwnAlgFyxMJZARgBoAGAXVJADcBDAGwFcYkQAJAfWDAGpiAXwAUADVIAdCQGMoEHAEoQg0uky58hFACYK1Ok1btuYKbPnLVIDNjwEiAZj00GLNok5cwoy2tuaiABZnAzdjLzFJGTlFX2t1Oy1kAFYQ1yMPKRiEFT8NexRyNMN3ECz5HP0YKABzeCJQADMAJwgAWyQikBwIJDJQjJByOJb2vpoepF0B0qwAPQAqEdaOxGnJxCcZ9gArReWxzYnexGDtj2HckFHV1O6T8kFKQSA
\[\begin{tikzcd}
\dots \arrow[r] & {H_{n+1}(X,\bullet)} \arrow[r, "0"] & H_n\bullet \arrow[r, "i^*"] & H_nX \arrow[r, "j^*"] & {H_n(X,\bullet)} \arrow[r, "0"] & \dots
\end{tikzcd}\]
gives a short exact sequence
% https://tikzcd.yichuanshen.de/#N4Igdg9gJgpgziAXAbVABwnAlgFyxMJZABgBpiBdUkANwEMAbAVxiRGJAF9T1Nd9CKAIzkqtRizYAJAPpgAOvIDGUCDi48QGbHgJEATKOr1mrRCFlgAGht46BRAMxHxp6XIAUV0opVqAlLZafLqCyAAsLiaS5hycYjBQAObwRKAAZgBOEAC2SGQgOBBIQtwZ2XmIIoXFiIauMSBYAHoAVEFZuUj1RUjODWYgAFZtHRV91L2I4fGcQA
\[\begin{tikzcd}
0 \arrow[r] & H_n\bullet \arrow[r, "i^*"] & H_nX \arrow[r, "j^*"] & {H_n(X,\bullet)} \arrow[r] & 0
\end{tikzcd}\]

Since $p^*:H_nX\rightarrow H_n\bullet$ is a retraction of $i^*$, Proposition \ref{short-seq-direct-sum} gives that $H_nX\iso H_n(X,\bullet)\dsum H_n\bullet$. Since this is isomorphic to $\tilde{H}_nX\dsum H_n\bullet$, $$H_n(X,x)\iso \tilde{H}_nX.$$
\end{proof}

\begin{corollary}
If $X$ is contractible to $x\in X$, then $\tilde{H}_n(X)=H_n(X,x)=0$ by the previous result and Remark \ref{contractible-0}.
\end{corollary}


As we will see, the reduced homology group also come with a long exact sequence. To define it, we will need a lemma. Consider an admissible category $C$ and a triple $(X,A,B)\in \Top{3}$ such that $(X,A),(X,B),(A,B)\in C$. The homology sequences of these three pairs, which will be labelled $(1),(3),(4)$ respectively, form the following braid diagram:

% https://tikzcd.yichuanshen.de/#N4Igdg9gJgpgziAXAbVABwnAlgFyxMJZARgBpiBdUkANwEMAbAVxiRAAkB9YMAamIC+ACgAapAIIBKEANLpMufIRRkAzFVqMWbLmABCMuSAzY8BIgFZyG+s1aIO3MAFpBB2fNNLLpddVvaDrqiEtIexgpmysgATKQxNlr2jmDihp6K5iiq1v5JOpxgQuKkemFGJpnROX6adgVgIukRXlnIACzxifVBhSFlzZVRRABsXXk9IIOR3igADKRz3YFT4UOzyAsJEyvTrdEL7cvJe1WjpEc7J2szbWNLV2wyGjBQAObwRKAAZgBOEABbJALEA4CBIdrhP6AiHUMFIEZQ-5AxBjUHgxA5EAAIxgYCgSGc7QAnEiYYgyOiIWSUZ0qYgLDSkFZ6XEcXiCYgSUzECD4QzqLj8RDiY8HAAdcVoOi-PCMZrQlGU-looWcomkoyKhFwjEAdh5et1SBiPJZ-KNdRWWAAegAqBXIpCW-lYgLJABW9sd5Kx-IAHDy2QGeaL6SC1cCxSAhMRyj8nRS+RiQe62EIYvGQNqk8aKdGhKoszniMqMZTI7yC+1i4m-eXTVrE8Hy6oBBQBEA
\[\begin{tikzcd}[row sep=small,column sep=small]
{} \arrow[rd, "(1)"]                   &                                                                &                            &                                                       &                                         &                                  & {} \\
                                       & {H_{n+1}(X,A)} \arrow[rd] \arrow[rr, "\partial", bend left=49] &                            & {H_n(A,B)} \arrow[rr, bend left=49] \arrow[rd, "i^*"] &                                         & H_{n-1}B \arrow[rd] \arrow[ru]   &    \\
{} \arrow[ru, "(2)"] \arrow[rd, "(3)"] &                                                                & H_nA \arrow[rd] \arrow[ru] &                                                       & {H_n(X,B)} \arrow[ru] \arrow[rd, "j^*"] &                                  & {} \\
                                       & H_nB \arrow[ru] \arrow[rr, bend right=49]                      &                            & H_nX \arrow[rr, bend right=49] \arrow[ru]             &                                         & {H_n(X,A)} \arrow[ru] \arrow[rd] &    \\
{} \arrow[ru, "(4)"]                   &                                                                &                            &                                                       &                                         &                                  & {}
\end{tikzcd}\]

The sequences $(1),(3),(4)$ commute with each other. The sequence $(2)$ is called the \defn{long exact homology sequence} for the triple $(X,A,B)$. The map $\partial:H_{n+1}(X,A)\rightarrow H_{n}(A,B)$ is defined so the diagram commutes, and all other maps in the sequence are induced by the canonical inclusions, which it is easy to see also commute with the diagram, either by looking at inclusions or the naturality of $\partial$.

\begin{prop}
For a triple $(X,A,B)\in \Top{3}$, the sequence

% https://tikzcd.yichuanshen.de/#N4Igdg9gJgpgziAXAbVABwnAlgFyxMJZARgBoAGAXVJADcBDAGwFcYkQAJAfWDAGpiAXwAUADVIBBAJQhBpdJlz5CKAEwVqdJq3bcwwiaQBCMuQux4CRAMwaaDFm0Scu+8SdnyQGC8qIAWOy1HXVcxSVMvHyUrFHIghx1nAB1kqAgcBDNvRUsVZABWBO0nEFT0zNlNGCgAc3giUAAzACcIAFskQJAcCCRybNaO-ppepDJgpLLktHoWvCZPZrbOxAmxxHVJ0qwAPQAqJZAh1a2N2232ACsDo5OkC42iy5SZuYXGKsEgA
\[\begin{tikzcd}
\dots \arrow[r] & {H_{n+1}(X,A)} \arrow[r, "\partial"] & {H_n(A,B)} \arrow[r, "i^*"] & {H_n(X,B)} \arrow[r, "j^*"] & {H_n(X,A)} \arrow[r, "\partial"] & \dots
\end{tikzcd}\]


is a long exact sequence.
\end{prop}
\begin{proof}
We first show that $(2)$ is a chain complex. By commutativity, the compositions $i\partial$ and $\partial j$ factor through two consecutive maps in a long exact sequence, and are hence $0$. For $j^*i^*$, note that $i$ and $j$ factor through $(A,A)$:
% https://tikzcd.yichuanshen.de/#N4Igdg9gJgpgziAXAbVABwnAlgFyxMJZABgBpiBdUkANwEMAbAVxiRAAoBBUgIQEoQAX1LpMufIRRkAjFVqMWbdgA1eA4aOx4CRMgCY59Zq0QdVndSJAYtEotNKzqRxaa6kLQuTCgBzeESgAGYAThAAtkhkIDgQSA7yxmxYQlahEfHUsUh6zgomIABWqcFhkYjR2YgAzHlJpgDWJSDp5bUxcYi5ia4gABZegkA
\[\begin{tikzcd}
{(A,B)} \arrow[d, "i"] \arrow[rd, "k"] &                         \\
{(X,B)} \arrow[d, "j"]                 & {(A,A)} \arrow[ld, "h"] \\
{(X,A)}                                &                        
\end{tikzcd}\]

In homology, this means $j^*i^*$ factors through $H_n(A,A)=0$, so $j^*i^*=0$.

% https://tikzcd.yichuanshen.de/#N4Igdg9gJgpgziAXAbVABwnAlgFyxMJZABgBpiBdUkANwEMAbAVxiRAAkB9MACgEFSAIQCUIAL6l0mXPkIoyARiq1GLNl14ANIaIlTseAkTIAmZfWatEHbj219dkkBgOyiC0kuoW114uOUYKABzeCJQADMAJwgAWyQyEBwIJA8VSzYsAD0AKnEnaLjU6mSkE29VKxAAK1z8yJj4xETSxABmCoy-epBCpo6klMRy9N8QfzEKMSA
\[\begin{tikzcd}
{H_n(A,B)} \arrow[d, "i^*"] \arrow[rd, "0"] &                   \\
{H_n(X,B)} \arrow[d, "j^*"]                 & 0 \arrow[ld, "0"] \\
{H_n(X,A)}                                  &                  
\end{tikzcd}\]

The result then follows by an application of the Braid Lemma (Lemma \ref{braid-lemma}).
\cite{Werndli}
\end{proof}

\begin{corollary}\label{contractible-reduced-0}
The sequence

% https://tikzcd.yichuanshen.de/#N4Igdg9gJgpgziAXAbVABwnAlgFyxMJZARgBoAGAXVJADcBDAGwFcYkQAJAfWDAGpiAXwAUADVIBBAJQhBpdJlz5CKAEwVqdJq3YAdXXkaxgHQVzATZ8kBmx4CRAMwaaDFm0Qh9h46fOirBTtlIgAWFy13dm4wMUkZOSClBxQAVgi3HU99KAgcBESbRXsVZHIM7Q8vXVz82U0YKABzeCJQADMAJwgAWyRykBwIJDJIrOq0ek68JkCQLt6RmiGkdTGqrAA9ACo5hb7ENZXEZ3X2ACsdve6D0+Pws+zdSemsWcL9pHTB4cRyQUogiAA
\begin{tikzcd}
\dots \arrow[r] & {H_{n+1}(X,A)} \arrow[r, "\partial"] & \tilde{H}_nA \arrow[r, "i^*"] & \tilde{H}_nX \arrow[r, "j^*"] & {H_n(X,A)} \arrow[r, "\partial"] & \dots
\end{tikzcd}
is an exact sequence, called the \defn{reduced homology sequence} of $(X,A)$.
\end{corollary}
\begin{proof}
This is simply the long exact sequence for a triple $(X,A,x)$, where $x\in A$:

% https://tikzcd.yichuanshen.de/#N4Igdg9gJgpgziAXAbVABwnAlgFyxMJZARgBoAGAXVJADcBDAGwFcYkQAJAfWDAGpiAXwAUADVIBBAJQhBpdJlz5CKAEwVqdJq3YdhE0gA8ZchdjwEiAZg00GLNok5ijJ+SAznlRACy2tDrpcYC7Ssu6eSpYoAKz+9jpOADpJUBA4CKYeihYqyOTx2o4gKWkZspowUADm8ESgAGYAThAAtkgFIDgQSGQBiSVJaPRNeEzhjS3tiH3dSOr9xVgAegBUEyDNbfM0c4g2i+wAVmsbW9MHe36HyUMjY4xnU0hxXT2I5IKUgkA
\[\begin{tikzcd}
\dots \arrow[r] & {H_{n+1}(X,A)} \arrow[r, "\partial"] & {H(A,x)} \arrow[r, "i^*"] & {H(X,x)} \arrow[r, "j^*"] & {H_n(X,A)} \arrow[r, "\partial"] & \dots
\end{tikzcd}\]

Using the isomorphisms $\tilde{H}_n X\iso H_n(X,x)$ and  defining the homomorphisms in the reduced sequence as appropriate gives the result. For example, we define $i^*:\tilde{H}_A\rightarrow \tilde{H}_nX$ as the composition 
% https://tikzcd.yichuanshen.de/#N4Igdg9gJgpgziAXAbVABwnAlgFyxMJZABgBpiBdUkANwEMAbAVxiRAB128HZgAJAL4B9MAEEQA0uky58hFAEZyVWoxZs+IgBSjSADwCUEqSAzY8BIgCZl1es1aIQmsFoAa+o5Onm5RAMy2qg5snNy8giJuEiowUADm8ESgAGYAThAAtkhkIDgQSErB6k6cWHAF3iDpWYXU+Ug2xY4gWAB6AFTGqRnZiE0NiIHNoezllRQCQA
\[\begin{tikzcd}
\tilde{H}_nA \arrow[r, "\iso"] & {H_n(A,x)} \arrow[r, "i^*"] & {H_n(X,x)} \arrow[r, "\iso"] & \tilde{H}_nX
.\end{tikzcd}\]

\end{proof}
 \cite{Hatcher}. 