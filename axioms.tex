An admissable category...

The axioms of (generic) homology theory...

\section{Basic results}
\begin{prop}
If $A\subset X$ is a deformation retract, then $H_n(X,A)=0$.
\end{prop}
\begin{proof}
If $A\subset X$ is a deformation retract, then the inclusion $i:A\rightarrow X$ is a homotopy equivalence. Let $r:X\rightarrow A$ be the retraction. Then $ir\simeq id_X$ and $ri\simeq id_A$.
By homotopy invariance of $H_n$ and the identity property of functors, $(ir)_*=(id_{X})_*=id_{H_nX}$ and $(ri)_*=(id_{A})_*=id_{H_nA}$. Since $(ri)_*=r_*i_*$ and vice-versa, we have that $i_*$ is an isomorphism.

Now consider the long exact homology chain:


% https://tikzcd.yichuanshen.de/#N4Igdg9gJgpgziAXAbVABwnAlgFyxMJZAJgBoAGAXVJADcBDAGwFcYkQAJAfTAEEQAvqXSZc+QigDMFanSat23MAA1BwkBmx4CRACwyaDFm0SceACmWleASjUit4ogFYDc44q7AwAWgCMAvxCDmI6KH5uRgqm3N4A1AGW1nbBGqLaEsgAbJHyJiAAOgVQEDgIqZqhmeS5HqZFJWWCsjBQAObwRKAAZgBOEAC2SPogOBBINe7RhQVo9L14TPYgfYMTNGNIEVP5WFwAVMurQ4jbm4hkO+wAVgdH-SeX59JX9bPzi4z3a4gv586pY5IHKjcaIXQCSgCIA
\begin{tikzcd}
\dots \arrow[r] & {H_{n+1}(X,A)} \arrow[r, "\partial"] & H_nA \arrow[r, "i_*"] & H_nX \arrow[r, "j_*"] & {H_n(X,A)} \arrow[r, "\partial"] & H_{n-1}A \arrow[r] & \dots
\end{tikzcd}

Since $i_*$ is an isomorphism, and the chain is exact, $H_nX=im(i_*)=ker(j_*)$ so $0=im(j_*)=ker(\partial)$

However, on the left we also have $0=ker(i_*)=im(\partial)$ since $i_*$ is an isomorphism. It follows that $H_n(X,A)=0$.

\end{proof}

\begin{remark}
\label{contractible-0}
As a special case of this result, $H_n(X,X)=0$, as $X$ is a deformation retract of itself. We are also interested in special case where $A=x$ is a single point, i.e. $X$ is contractible. Then we have $H_n(X,x)=0$.
\end{remark}

It is possible to reduce the abelian objects $H_nX$ into simpler objects $\tilde{H}_nX$ by in some sense factoring out the object $H_n1$. Furthermore, this can be done without losing any information, i.e. the transformation $H_nX\rightarrow \tilde{H}_nX$ is reversible.

First we will need some facts about abelian objects.

\begin{definition}
If there exist maps $f:A\rightarrow B$ and $g:B\rightarrow A$ between abelian objects such that $g\circ f = id_A$ then $g$ is called a \defn{retraction} of $f$, and $f$ is called a \defn{something} of $g$.
\end{definition}

$g$ can be thought of as a one-sided inverse of $f$, as there is no requirement that $g\circ f=id_B$.

\begin{lemma}
If $g:B\rightarrow A$ is a retraction of $f:A\rightarrow B$, then $B\iso im(f)\dsum ker(g)$ 
\end{lemma}

\begin{proof}
\label{direct-sum-iso}
The isomorphism is given by $$h:B\rightarrow im(f)\dsum ker(g)$$
$$x \mapsto (f\circ g(x),x-f\circ g(x))$$
This is well-defined as $f\circ g(x)\in im(f)$ and 
$$g(x-f\circ g(x))=g(x)-g\circ f \circ g(x)=g(x)-g(x)=0$$
by the associativity of homomorphisms, and since $g\circ f= id_A$. Hence $x-f\circ g(x)\in ker(g)$
The inverse is $$h^{-1}:im(f)\dsum ker(g)\rightarrow B$$
$$(a,b)\mapsto a+b$$
One quickly checks that
$$h\circ h^{-1}(a,b)=h(a+b)=(f\circ g(a+b),a+b - f\circ g(a+b))$$
$$=(f\circ g(a)+0,a+b-f\circ g(a))$$
since $b\in ker(g)$. However, $a=f(c)$ for some $c\in A$, so
$$(f\circ g(a)+0,a+b-f\circ g(a))=(f\circ g \circ f(c),a+b-f\circ g \circ f(c))$$
$$=(f(c),a+b-f(c))=(a,b)$$
since $f\circ g=id_B$. Additionally,
$$h^{-1}\circ h(x)=f\circ g(x)+x-f\circ g(x)=x$$
So $h$ and $h^{-1}$ are indeed inverse homomorphisms, and $B\iso im(f)\dsum ker(g)$.
\end{proof}

We will use this Lemma on the following construction.

\begin{definition}
$\tilde{H}_n(X)=ker(p^*:H_nX\rightarrow H_n1)$ where $p^*$ is the map induced by the initial map $p:X\rightarrow 1$. $\tilde{H}_nX)$ is called the \defn{reduced homology} of $X$.
\end{definition}

\begin{prop}
\label{reduced-homology}
For any $x\in X$,
$H_n(X,A)=\tilde{H}_n(X,A)\dsum H_n1=H_n(X,x)\dsum H_n1$
\end{prop}

\begin{proof}
For the first equality, consider the following diagram. $x$ exists by assumption MISSING of an admissable category, and $p$ exists since $1$ is an initial object. Notice $p$ is a retraction of $x$.

DIAGRAM

$H_n$ induces the following diagram, and since functors map compositions to compositions and identities to identities, we have that $p^*\circ x^*=id_{H_n1}$, so $p^*$ is a retraction of $x^*$.

DIAGRAM

By \ref{direct-sum-iso}, $$H_nX\iso im(x^*)\dsum ker(p^*)=im(x^*)\dsum H_n1\dsum \tilde{H}_nX$$
where the last equality holds because $p^*\circ x^* = id_{H_n1}$ guarantees that $x^*$ is injective.

For the second equality, note any two initial objects are isomorphic. In particular, for $x\in X$, $x\iso 1$. We hence have the following long exact chain

DIAGRAM

From which we can extract a short exact chain

DIAGRAM

By exactness, $H_n1\iso im(x^*)\iso ker(j*)$ and $im(j*)\iso ker(\partial)=H_n(X,x)$. Furthermore, by the first isomorphism theorem, $im(j*)\iso H_nX/ker(j*)$. Therefore, $$H_n(X,x)\iso H_nX/H_n1\iso \tilde{H}_nX\dsum H_n1/H_n1 \iso \tilde{H}_nX$$
where the last few isomorphisms are done somewhat informally, to be made precise at a future point. MISSING
\end{proof}

\begin{corollary}
If $X$ is contractible to $x\in X$, then $H_n(X)=H_n(X,x)\dsum H_n1=0\dsum H_n1\iso H_n1$, by \ref{contractible-0}. It follows that $\tilde{H}_nX=0$
\end{corollary}

\ref{reduced-homology} shows that $H_n(X,A)$ always carries around a copy of $H_n1$, which can be safely removed by going to the kernel of $p^*$. Gluing a copy of $H_n1$ onto $\tilde{H}_n(X,A)$ recovers the original object.

We would like to do manipulations using $\tilde{H}_nX$ instead of $H_n X$, as these spaces are simpler. To justify this, we should show that $\tilde{H}_n X$ forms a homology theory whenever $H_n$ does.

\begin{theorem}
If $H_n$ and $\partial$ form a homology theory over some category $C$, then for $\tilde{H}_nX$ there exist $\tilde{\partial}$ such that they form a homology theory as well.
\end{theorem}