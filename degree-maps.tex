\section{Degree maps}\label{sec-degree-maps}
The only group homomorphisms $g: \mathbb{Z} \rightarrow \mathbb{Z}$ are multiplication by an integer $m$, called the degree of $g$. For maps $f:S^n\rightarrow S^n,n>0$, we can therefore define the \defn{degree} $deg(f)$ as the degree of the induced map $f^*: \mathbb{Z}\rightarrow \mathbb{Z}$. It is clear from the definition that $deg(id)=1$, as $id^*=id$. Additionally, if $f\homotopic g$, then $f^*=g^*$, so $deg(f)=deg(g)$. Furthermore, $deg(fg)=deg(f)deg(g)$, as $(fg)^*=f^*g^*$. We are interested in some special cases of $f$.

\begin{proposition}
$deg(r_i)=-1$ where $r_i:S^n\rightarrow S^n$ is the reflection in the $i-th$ coordinate:
$$(x_1,x_2,\dots,x_i,\dots x_{n+1})\mapsto (x_1,x_2,\dots,-x_i,\dots x_{n+1})$$
\end{proposition}
\begin{proof}
By composing with a rotation, we can assume $r_i$ is the reflection in the $(n+1)$th degree which maps the north-pole $N$ to the south-pole $S$, keeping the equator $S^{n-1}$ fixed. We denote this map by $r$. Consider the following Mayer-Vietoris cover of $S^n$:  $A=S^n\setminus D^n_+\homotopic \bullet$, $B=S^n\setminus D^n_-\homotopic \bullet$ and $A\cap B \homotopic S^{n-1}$. $r$ gives a map between the Mayer-Vietoris sequence of $(S^n;A,B)$ and $(S^n;B,A)$. For $n>0$, this gives the commutative diagram:

% https://tikzcd.yichuanshen.de/#N4Igdg9gJgpgziAXAbVABwnAlgFyxMJZARgBoAGAXVJADcBDAGwFcYkQAJAfTAGUA9QgF9S6TLnyEUAJgrU6TVu27AwAWmJCBqjUJAix2PASLk5NBizaIQ5faJAYjkogGZzCq+wA63qBBwEA0dxYylkM2J5SyUbO2CnCRMUdyiLRWsQX39A+0Mk8LI0z1jOHgFhB0SwollimMyVdU1tZr0heRgoAHN4IlAAMwAnCABbJDMQHAgkADZ0rxsh-gAqPJBhsaQyKZnEAHYF0qwodc3xxFldieDziZpp7aPM3zR6IbwmM5GLncfEVy3H5IAAsDz2syBW0Q82uB2ePm8bw+WCYAHJvtDDnCAKwdIRAA
\[\begin{tikzcd}
0 \arrow[r] & H_nS^n \arrow[d, "r^*"] \arrow[r, "\partial"] & H_{n-1}S^{n-1} \arrow[d, "id"] \arrow[r] & \dots \\
0 \arrow[r] & H_nS^n \arrow[r, "\partial'"]                 & H_{n-1}S^{n-1} \arrow[r]                 & \dots
\end{tikzcd}\]
Recall that $\partial=\partial_1k_1^{*-1}l_1^*=-\partial_2k_2^{*-1}l_2^*$ in the following diagram.

% https://tikzcd.yichuanshen.de/#N4Igdg9gJgpgziAXAbVABwnAlgFyxMJZARgBoAGAXVJADcBDAGwFcYkQAJAfTAAIANEAF9S6TLnyEU5UsWp0mrdtzAAKfqQCCASmGiQGbHgJEATLPkMWbRJx7rSAIV0ixRyWdIBmS4pt21TS0AHWCAY3o0Xmc9NwkTaW9fa2V7RxDwyOiXfUN4qRJSABZkpVtuYDAAWmIhVU1QiKiY1wNxYwKyU1L-FQcGzOaXeRgoAHN4IlAAMwAnCABbJBkQHAgkMgUU20YuYgA9ACpYkDnF5Zo1pHMtspBd0yOTs6XEL0v1xBurO4BrLkex1aLyQRQ+GxoP38-wOQP0IMQYNWnwAbJC-OwsACnsD5q80cjrujtiAAFbYuEzPFIAlXRCbKHscmw57Ut7gxAExm2LEs3HnRArOlcjG2MA4+FspF0gCsxLuoTQ9FmeCYe1ZAvehMQctu-kVytVDw1+I5uu5ICgwkoQiAA
\[\begin{tikzcd}
                                                                                  & H_n X \arrow[ld, "l_1^*"] \arrow[rd, "l_2^*"] \arrow[dd, "n^*"]          &                                                                                   \\
{H_n(X,A)}                                                                        &                                                                          & {H_n(X,B)}                                                                        \\
                                                                                  & {H_n(X,A\cap B)} \arrow[ru, "j_2^*"] \arrow[lu, "j_1^*"] \arrow[dd, "d"] &                                                                                   \\
{H_n(B,A\cap B)} \arrow[uu, "k_1^*"] \arrow[ru, "i_2^*"] \arrow[rd, "\partial_1"] &                                                                          & {H_n(A,A\cap B)} \arrow[uu, "k_2^*"] \arrow[lu, "i_1^*"] \arrow[ld, "\partial_2"] \\
                                                                                  & H_{n-1}(A\cap B)                                                         &                                                                                  
\end{tikzcd}\] Swapping the order of $A$ and $B$ corresponds to a horizontal reflection in the diagram, which therefore changes the sign of the boundary map. It follows that $\partial$=$-\partial'$. By commutativity, $r^*=-id$, so $deg(r)=-1$.

\end{proof}

\begin{corollary}\label{antipodal-degree}
$deg(-id)=(-1)^{n+1}$ where $(-id):S^n\rightarrow S^n$ is the antipodal map.
\end{corollary}
\begin{proof}
Since $(-id)$ is the composition $(-id)=r_1\circ r_2 \circ \dots \circ r_{n+1}$, $$deg(-id)=deg(r_1)deg(r_2)\dots deg(r_n{+1})=(-1)^{n+1}.$$
\cite{Hatcher}.
\end{proof}

\begin{corollary}\label{fixed-points}
If $f:S^n\rightarrow S^n$ has no fixed points, then $deg(f)=(-1)^{n+1}$
\end{corollary}
\begin{proof}
Since $f(x)\neq x$, the line segment from $f(x)$ to $-x$ does not pass through $0$. We can therefore define a homotopy $f(x)\homotopic (-id)$, which takes each $f(x)$ to $x$ along an arc segment. Therefore $deg(f(x))=deg(-id)=(-1)^{n+1}$.\cite{Hatcher}.
\end{proof}

\begin{remark}
Note that the antipodal map commutes with the projection $p:S^n\rightarrow \RP{n}$:
% https://tikzcd.yichuanshen.de/#N4Igdg9gJgpgziAXAbVABwnAlgFyxMJZABgBpiBdUkANwEMAbAVxiRAGUA9QgX1PUy58hFGQCMVWoxZsuvfiAzY8BImPKT6zVohAAdPQCUACsDA8QfAcuFrSE6lpm6DJsxZ6SYUAObwioABmAE4QALZIZCA4EEgATI7SOoqWCiHhSOrRsYgAzInabGipQaERiAnZSPlShbpYUCUg6eVRMZkFziAAFAC0DQCUlhQ8QA
\[\begin{tikzcd}
S^n \arrow[r, "p"] \arrow[d, "-id"] & \RP{n} \arrow[d, "id"] \\
S^n \arrow[r, "p"]                    & \RP{n}                
\end{tikzcd}\]
When $n$ is odd, this gives rise to a commutative map in homology

% https://tikzcd.yichuanshen.de/#N4Igdg9gJgpgziAXAbVABwnAlgFyxMJZABgBpiBdUkANwEMAbAVxiRAB12BbOnACwBGA4AC0AviDGl0mXPkIoyARiq1GLNpx78hoiVJnY8BIkvKr6zVohAAJAPphOAJQAKwMPukgMR+adIVaksNGwcndjcPfVUYKABzeCJQADMAJwguJDIQHAgkACZg9WsfAD0AKklvdMykM1z8xABmYqs2NErq1IysxCLGpFa1dpssKG6QWr6cvPq20JAAWgAPSQoxIA
\[\begin{tikzcd}
\mathbb{Z} \arrow[r, "p^*"] \arrow[d, "-id"] & H_n\RP{n} \arrow[d, "id"] \\
\mathbb{Z} \arrow[r, "p^*"]                 & H_n\RP{n}                
\end{tikzcd}\]

But this means $p^*=-p^*$, so $p^*=0$. This is the first hint that there is something fundamentally different about $\RP{n}$ when $n$ is odd and when $n$ is even. 
\end{remark}


Next, we show how the famous Brouwer's fixed point theorem can be proven from degree theory.

\begin{theorem}[Brouwer's fixed-point theorem]
Every continuous $f:D^n\rightarrow D^n$ has a fixed point.
\end{theorem}
\begin{proof}
Suppose $f$ has no fixed points. Define the map $g:S^n\rightarrow S^n$ as the composition 
$$S^n\xrightarrow{q}D^n\xrightarrow{f}D^n\xrightarrow{i}S^n$$
where $q$ is the quotient map of the identification $(x_1,\dots,x_n)\sim (y_1,\dots,-x_n)$, i.e. the map that folds the lower hemisphere onto the upper hemisphere, where we have identified the codomain as $D^n$. $i$ is the inclusion identifying $D^n$ as the upper hemisphere of $S^n$. Note that $g$ also has no fixed points. By Corollary \ref{fixed-points}, $deg(g)=(-1)^{n+1}$. We could alternatively have made the identification $i':D^n\rightarrow S^n$ with the lower hemisphere, and $g$ would still have no fixed points. However, this corresponds to composing $g$ with the reflection $r_{(n+1)}$, which has degree $-1$. This implies
$$(-1)^{n+1}=deg(rg)=deg(r)deg(g)=(-1)^{n}$$
which is a contradiction.
\end{proof}

%\begin{proposition}
%$H_n(\RP{2})=\begin{cases}\mathbb{Z} & n=0,2\\ \mathbb{Z}/2\mathbb{Z} & n=1\\ 0 &\text{otherwise}\end{cases}$
%\end{proposition}
%\begin{proof}
%Consider the pair ($(\RP{2},\RP{1}$). Since $\RP{1}\homotopic \S{1}$ (MISSING);
%\end{proof}

There is a convenient way for calculating degrees of maps $f:S^n\rightarrow S^n$. Suppose that $f$ has the property that for some $x\in S^n$, $f^{-1}(x)$ is a finite set $\{y_i\}_{i\in J}$. We define the \defn{local degree} of $f$, $deg(f)|{y_i}$ at $y_i$ as the degree of the map $f:U_i\setminus \{y_i\}\rightarrow V\setminus\{x\}$, where $U_i$ are disjoint open neighbourhoods of respectively the $y_i$'s and $V$ is an open neighbourhood they are all mapped into. By the metric space structure of $S^n$ we may take the $U_i$'s to be sufficiently small $n$-disks, and $V$ some small $n$-disk containing their images. This shows that $f:U_i\setminus \{y_i\}\rightarrow V\setminus\{x\}$ gives a map $f:S^{n-1}\rightarrow S^{n-1}$ as desired.

\begin{proposition}\label{local-degree}
If $f$ is as proposed, then $$deg(f)=\sum_{i\in J}deg(f)|{y_i}.$$
\end{proposition}
\begin{proof}
Omitted. See \cite{Hatcher}
\end{proof}