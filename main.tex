\documentclass[11pt,a4paper]{article}
\usepackage[utf8x]{inputenc}
\usepackage[T1]{fontenc}
\usepackage{mathptmx} % Use Times Font
%\usepackage{listings}
\usepackage{amsmath}
\usepackage{amssymb}
\usepackage{tikz-cd}
\usepackage{enumerate}
%\usepackage{mathtools}
%\usepackage{theoremref}
\usepackage{amsthm}
\newcounter{thm}
\numberwithin{thm}{section}
\newtheorem{theorem}{Theorem}[thm]
\newtheorem{corollary}{Corollary}[thm]
\newtheorem{lemma}[thm]{Lemma}
\newtheorem{prop}[thm]{Proposition}
\newtheorem{remark}[thm]{Remark}
\newtheorem{definition}[thm]{Definition}
\newtheorem{example}[thm]{Example}

\newcommand{\dsum}{\bigoplus}
\newcommand{\iso}{\cong}
\newcommand{\homeo}{\cong}
\newcommand{\defn}[1]{\textit{#1}}
\newcommand{\cat}[1]{\mathcal{#1}}
\newcommand{\RP}[1]{\mathbb{RP}^{#1}}
\newcommand{\CP}[1]{\mathbb{CP}^{#1}}
\newcommand{\HP}[1]{\mathbb{HP}^{#1}}
\newcommand{\S}[1]{\mathbb{S}^{#1}}

\title{Homology Theory and the Borsuk-Ulam Theorem}
\author{Malthe Sporring}

\begin{document}
\maketitle

\section{Acknowledgements}
This text provides an introduction to homology theory from an axiomatic point of view. The reader is assumed to be aware of undergraduate level algebra and algebraic topology. The project was funded by University of Edinburgh School of Mathematics Vacation Scholarship and College Vacation Scholarship, to which I am very grateful. I am additionally grateful to Dr Clark Barwick for his guidance and advice throughout this project.
\section{Introduction}
One of the goals of algebraic topology is to classify spaces up to some definition of equivalence, typically homotopy equivalence. Establishing equivalence can be tedious, as it often requires the construction of an explicit homotopy equivalence. It is typically easier to determine \textit{inequivalence}, through calculations of \defn{homotopy invariants}. These are properties of a space that are preserved by homotopy equivalences, hence if two spaces have different invariants, they cannot be homotopy equivalent. Some of the first major homotopy invariances the undergraduate student encounters are the homotopy groups $\pi_n(X),n\in \mathbb{N}\cup \{0\}$: the groups of equivalence classes of maps $f:S^{n}\rightarrow X$ with some base point. The homotopy groups carry a lot of geometric information but are notoriously difficult to compute. Even for a simple space like the $2$-sphere $S^2$, it is not obvious that there are non-trivial maps $S^3\rightarrow S^2$ (we show that such a map exists in section \ref{sec-complex-projective-space}), and it is even less obvious what the higher homotopy groups are. The groups $\pi_n(S^2)$ do not seem to follow a pattern and remain an active area of research. Only recently, in 2015, was it proven that $\pi_n(S^2)$ is not zero for all $n\geq 2$ \cite{Ivanov}.

To avoid these complications, we would like to find homotopy invariances that are easier to compute, and ideally not at the cost of too much geometric information. The homology groups $H_n(X)$ are an example of such homotopy invariants. Like the homotopy groups, they are a sequence of groups, one for each $n\in \mathbb{Z}$, but they are much simpler and easier to compute. For example, the spheres have the simple structure $$H_n(S^m)=\begin{cases}\mathbb{Z} & n=0,m\\ 0 & \text{otherwise}\end{cases}$$ 
Homology groups have some key properties that aid in calculations, most importantly a \defn{long exact sequence}, which, broadly speaking, relates the homology groups $H_nX$ to each other and to the homology groups $H_nA$ of a subspace $A\subseteq X$. Therefore, the more homology groups you know, the easier it is to calculate the rest.

Historically, homology groups were calculated from a number of geometric methods. It was Eilenberg and Steenrod who united the different homology theories by laying out a set of axioms that all homology theories satisfy \cite{Eilenberg}. In this text, we will take such an axiomatic approach, proving all results directly from the axioms. In some ways this approach best captures the essence of homology: the main task of a geometric approach to homology is to prove the Eilenberg-Steenrod axioms, and in practical calculations, the axioms are often preferred over the geometric construction. However, this approach is not without its disadvantages. For the results in this text to be true, we have to take as given that there exists a homology theory that satisfies the axioms, and proving this is a major undertaking worth its own project. Singular Homology is an example of a homology theory that satisfies our assumptions, as the reader is invited to confirm in \cite{Hatcher}.

Homology theory is best understood in the language of category theory and chain complexes, which sections \ref{sec-category-theory} and \ref{sec-chain-complexes} are devoted to. In section \ref{sec-axioms}, we lay out the Eilenberg-Steenrod axioms and prove some immediate results for an ordinary homology theory, most importantly the homology groups of the $n$-sphere. In the following sections, we make the choice $H_0 \bullet=\mathbb{Z}$, which corresponds to Singular Homology. In sections \ref{sec-mayer-vietoris}, \ref{sec-degree-maps} and \ref{sec-cellular-homology}, we lay out three practical methods for calculating homology groups: The Mayer-Vietoris Sequence, degree maps and Cellular Homology, and use them to prove some fascinating results. In Section \ref{sec-borsuk-ulam}, we put everything together to prove the celebrated Borsuk-Ulam Theorem.
\begin{definition}
A \defn{chain complex} is a sequence of algebraic objects and homomorphisms between them, such that for any consecutive $f_{i},f_{i+1}$ we have that $im(f_i)\subset ker(f_{i+1})$. If we have equality instead of inclusion, the construction is called an \defn{exact sequence}.
\end{definition}

**MISSING: Not actually sequences, as they can be infinite in both directions

We also distinguish between \defn{short} and \defn{long exact sequences}, where the former is finite sequences of three or fewer nonzero elements, and the latter is all other sequences.

\begin{example}
For any objects $A_1,A_2,\dots$ the following is a long exact sequence:
DIAGRAM MISSING
\end{example}

\begin{remark}
The requirement that $im(f_i)\subseteq ker(f_{i+1})$ for all $i\in \mathbb{Z}$ is equivalent to the requirement that $f_{i+1}\circ f_i=0$ for all $i\in \mathbb{Z}$. This is clear from the definition; if $im(f_i)\subseteq ker(f_{i+1})$ then $f_{i+1}\circ f_i (A_i)=f_{i+1}(im(f_i))=0$. Conversely, if $f_{i+1}\circ f_i=0$ then $f_{i+1}(im(f_i))=0$.
\end{remark}

\begin{lemma}[Braid lemma]

\end{lemma}
An admissable category...

The axioms of (generic) homology theory...

\section{Basic results}
\begin{prop}
If $A\subset X$ is a deformation retract, then $H_n(X,A)=0$.
\end{prop}
\begin{proof}
If $A\subset X$ is a deformation retract, then the inclusion $i:A\rightarrow X$ is a homotopy equivalence. Let $r:X\rightarrow A$ be the retraction. Then $ir\simeq id_X$ and $ri\simeq id_A$.
By homotopy invariance of $H_n$ and the identity property of functors, $(ir)_*=(id_{X})_*=id_{H_nX}$ and $(ri)_*=(id_{A})_*=id_{H_nA}$. Since $(ri)_*=r_*i_*$ and vice-versa, we have that $i_*$ is an isomorphism.

Now consider the long exact homology chain:


% https://tikzcd.yichuanshen.de/#N4Igdg9gJgpgziAXAbVABwnAlgFyxMJZAJgBoAGAXVJADcBDAGwFcYkQAJAfTAEEQAvqXSZc+QigDMFanSat23MAA1BwkBmx4CRACwyaDFm0SceACmWleASjUit4ogFYDc44q7AwAWgCMAvxCDmI6KH5uRgqm3N4A1AGW1nbBGqLaEsgAbJHyJiAAOgVQEDgIqZqhmeS5HqZFJWWCsjBQAObwRKAAZgBOEAC2SPogOBBINe7RhQVo9L14TPYgfYMTNGNIEVP5WFwAVMurQ4jbm4hkO+wAVgdH-SeX59JX9bPzi4z3a4gv586pY5IHKjcaIXQCSgCIA
\begin{tikzcd}
\dots \arrow[r] & {H_{n+1}(X,A)} \arrow[r, "\partial"] & H_nA \arrow[r, "i_*"] & H_nX \arrow[r, "j_*"] & {H_n(X,A)} \arrow[r, "\partial"] & H_{n-1}A \arrow[r] & \dots
\end{tikzcd}

Since $i_*$ is an isomorphism, and the chain is exact, $H_nX=im(i_*)=ker(j_*)$ so $0=im(j_*)=ker(\partial)$

However, on the left we also have $0=ker(i_*)=im(\partial)$ since $i_*$ is an isomorphism. It follows that $H_n(X,A)=0$.

\end{proof}

\begin{remark}
\label{contractible-0}
As a special case of this result, $H_n(X,X)=0$, as $X$ is a deformation retract of itself. We are also interested in special case where $A=x$ is a single point, i.e. $X$ is contractible. Then we have $H_n(X,x)=0$.
\end{remark}

\subsection{Reduced homology}

It is possible to reduce the abelian objects $H_nX$ into simpler objects $\tilde{H}_nX$ by in some sense factoring out the object $H_n1$. Furthermore, this can be done without losing any information, i.e. the transformation $H_nX\rightarrow \tilde{H}_nX$ is reversible.

First we will need some facts about abelian objects.

\begin{definition}
If there exist maps $f:A\rightarrow B$ and $g:B\rightarrow A$ between abelian objects such that $g\circ f = id_A$ then $g$ is called a \defn{retraction} of $f$, and $f$ is called a \defn{something} of $g$.
\end{definition}

$g$ can be thought of as a one-sided inverse of $f$, as there is no requirement that $g\circ f=id_B$.

\begin{lemma}
If $g:B\rightarrow A$ is a retraction of $f:A\rightarrow B$, then $B\iso im(f)\dsum ker(g)$ 
\end{lemma}

\begin{proof}
\label{direct-sum-iso}
The isomorphism is given by $$h:B\rightarrow im(f)\dsum ker(g)$$
$$x \mapsto (f\circ g(x),x-f\circ g(x))$$
This is well-defined as $f\circ g(x)\in im(f)$ and 
$$g(x-f\circ g(x))=g(x)-g\circ f \circ g(x)=g(x)-g(x)=0$$
by the associativity of homomorphisms, and since $g\circ f= id_A$. Hence $x-f\circ g(x)\in ker(g)$
The inverse is $$h^{-1}:im(f)\dsum ker(g)\rightarrow B$$
$$(a,b)\mapsto a+b$$
One quickly checks that
$$h\circ h^{-1}(a,b)=h(a+b)=(f\circ g(a+b),a+b - f\circ g(a+b))$$
$$=(f\circ g(a)+0,a+b-f\circ g(a))$$
since $b\in ker(g)$. However, $a=f(c)$ for some $c\in A$, so
$$(f\circ g(a)+0,a+b-f\circ g(a))=(f\circ g \circ f(c),a+b-f\circ g \circ f(c))$$
$$=(f(c),a+b-f(c))=(a,b)$$
since $f\circ g=id_B$. Additionally,
$$h^{-1}\circ h(x)=f\circ g(x)+x-f\circ g(x)=x$$
So $h$ and $h^{-1}$ are indeed inverse homomorphisms, and $B\iso im(f)\dsum ker(g)$.
\end{proof}

We will use this Lemma on the following construction.

\begin{definition}
$\tilde{H}_n(X)=ker(p^*:H_nX\rightarrow H_n1)$ where $p^*$ is the map induced by the initial map $p:X\rightarrow 1$. $\tilde{H}_nX)$ is called the \defn{reduced homology} of $X$.
\end{definition}

\begin{prop}
\label{reduced-homology}
For any $x\in X$,
$H_n(X,A)=\tilde{H}_n(X,A)\dsum H_n1=H_n(X,x)\dsum H_n1$
\end{prop}

\begin{proof}
For the first equality, consider the following diagram. $x$ exists by assumption MISSING of an admissable category, and $p$ exists since $1$ is an initial object. Notice $p$ is a retraction of $x$.

DIAGRAM

$H_n$ induces the following diagram, and since functors map compositions to compositions and identities to identities, we have that $p^*\circ x^*=id_{H_n1}$, so $p^*$ is a retraction of $x^*$.

DIAGRAM

By \ref{direct-sum-iso}, $$H_nX\iso im(x^*)\dsum ker(p^*)= H_n1\dsum \tilde{H}_nX$$
where the last equality holds because $p^*\circ x^* = id_{H_n1}$ guarantees that $x^*$ is injective.

For the second equality, note any two initial objects are isomorphic. In particular, for $x\in X$, $x\iso 1$. We hence have the following long exact chain

DIAGRAM

From which we can extract a short exact chain

DIAGRAM

By exactness, $H_n1\iso im(x^*)\iso ker(j*)$ and $im(j*)\iso ker(\partial)=H_n(X,x)$. Furthermore, by the first isomorphism theorem, $im(j*)\iso H_nX/ker(j*)$. Therefore, $$H_n(X,x)\iso H_nX/H_n1\iso \tilde{H}_nX\dsum H_n1/H_n1 \iso \tilde{H}_nX$$
where the last few isomorphisms are done somewhat informally, to be made precise at a future point. MISSING
\end{proof}

\begin{corollary}
If $X$ is contractible to $x\in X$, then $H_n(X)=H_n(X,x)\dsum H_n1=0\dsum H_n1\iso H_n1$, by \ref{contractible-0}. It follows that $\tilde{H}_nX=0$
\end{corollary}

\ref{reduced-homology} shows that $H_n(X,A)$ always carries around a copy of $H_n1$, which can be safely removed by going to the kernel of $p^*$. Gluing a copy of $H_n1$ onto $\tilde{H}_n(X,A)$ recovers the original object.

As we will see, there is a long exact chain for the reduced homology. To define it, we will need the following lemma. Consider an admissable category $C$ and a triple $(X,A,B)\in Top_{(3)}$ such that $(X,A),(X,B),(A,B)\in C$. The long exact sequences, which will be labelled $(1),(3),(4)$ respectively, form a braid diagram as shown below. The sequence $(2)$, called the \defn{long exact homology sequence} for the triple $(X,A,B)$ is defined in such a way that the diagram commutes. Explicitly, it is the sequence

DIAGRAM

where $\partial$ is the composition of the transformation $\partial:H_{n+1}(X,A)\rightarrow H_n A$ and $i^*:H_nA\rightarrow H_n(A,B)$. The other two maps are induced by the inclusions $H_n(A,B)\rightarrow H_n(X,B)$ and $H_n(X,B)\rightarrow H_n(X,A)$.

\begin{prop}
For a triple $(X,A,B)\in Top_{(3)}$ such that $(X,A),(X,B),(A,B)\in C$, the sequence

DIAGRAM 

is a long exact sequence.
\end{prop}
\begin{proof}
We will start by noting that the diagram above commutes. In any subdiagram of the diagram not involving $\partial$, this follows by noting the subdiagram is induced (via a composition-preserving functor) from a diagram of inclusions which commute by the definition of an admissible category. Any subsquare or triangle involving $\partial$ commutes since $\partial$ is a natural transformation, or by the definition of the special $\partial$ defined above.

After noting the diagram commutes, it becomes easy to show $(2)$ is a chain complex, i.e. that the composition of any two maps is $0$. For two out of three compositions, you can via commutativity choose an alternative path which goes via two consecutive maps in a exact sequence, and is hence $0$. For the final composition

DIAGRAM

note we have the following commutative diagram (commutative since it is induced by the commutative diagram of inclusions):

Since $H_n(A,A)=0$, the composition is $0$, as required. ($\text{https://ncatlab.org/nlab/show/braid+lemma\# Munkres}$) The result is then achieved by an application of the braid lemma.
\end{proof}

\begin{corollary}
The sequence

DIAGRAM

is a long exact sequence.
\end{corollary}
\begin{proof}
First note that if $(X,A)\in C$ then by the requirements of an admissable category, $(X,x)$ and $(A,x)\in C$ as well for any $x\in A$. Hence by the previous proposition, and using the isomorphism $\tilde{H}_n X\iso H_n(X,x)$, we get the result.
\end{proof}
\subsection{Homology of $\S{n}$}
We will now perform a calculation of the groups $H_kS^{n}$ from the axioms. Our approach will be an adaptation of that taken in cite{Werndli}. 

\begin{lemma}\label{sphere-isomorphism} For all $n\in \mathbb{Z}$,
$$\tilde(H)_kS^n\iso \tilde{H}_{k-1}S^{n-1}$$
\end{lemma}
\begin{proof}
Consider first the pair $(D^n,S^{n-1})$, where $S^{n-1}$ is the boundary of $D^n$. Since $D^n$ is contractible, the reduced homology sequence reads:
% https://tikzcd.yichuanshen.de/#N4Igdg9gJgpgziAXAbVABwnAlgFyxMJZAJgBoAGAXVJADcBDAGwFcYkQAJAfQGsAKACIA9MKQDKQ4GAC0ARgC+AShDzS6TLnyEUAZgrU6TVuwA6JvI1jAO8rsB5z5EqY5VqQGbHgJEALPpoGFjZEEHI3dS8tIlkAw2D2cNVIzR8Ucjig41CzKAgcBGSPDW9tZABWTKMQkFz8woMYKABzeCJQADMAJwgAWyQ9EBwIJHIi7r7RmmGkBXcJ-sRYoZHEYnGexf8VgY3JtenV8vlKeSA
\begin{tikzcd}
\dots \arrow[r] & 0 \arrow[r] & {H_k(D^n,S^{n-1})} \arrow[r] & \tilde{H}_{k-1}S^{n-1} \arrow[r] & 0 \arrow[r] & \dots
\end{tikzcd}
This gives an isomorphism
$$H_k(D^n,S^{n-1})\iso \tilde{H}_{k-1}S^{n-1}$$

Additionally, we have the pair $(S^n,D^n)$, where $D^n$ is identified as the closed lower hemisphere of $S^n$. Since $D^n$ is contractible, $H_k(S^n,D^n)\iso H_k(S^n,\cdot)\iso\tilde{H}_kS^n$.  The open disk $D_{1/2}$ of radius $1/2$ is a subset of $D^n$ which can be excised from the pair $(S^n,D^n)$. The resulting space deformation retracts to the pair $(D^n,S^{n-1})$, where $D^n$ is the upper hemisphere and $S^{n-1}$ its boundary. By the excision axiom,

$$\tilde{H}_kS^n\iso H_k (S^n,D^n)\iso H_k(D^n,S^{n-1})$$
Together with our previous isomorphism, we have
$$\tilde{H}_kS^n\iso H_k(D^n,S^{n-1})\iso \tilde{H}_{k-1}S^{n-1}$$
as required.


\end{proof}

\begin{proposition}
For $n>0$,
$$H_kS^n=\begin{cases}G&k=0,n \\ 0 & \text{otherwise}\end{cases}$$
\end{proposition}
\begin{proof}
We identify $S^{0}=\cdot \sqcup \cdot$. By proposition (MISSING), $$H_k S^{0}=H_k\cdot \bigoplus H_k \cdot=\begin{cases}G\timesG &=0\\ 0 & \text{otherwise}\end{cases}$$. It follows that $\tilde{H}_0S^{0}=G$ and $\tilde{H}_kS^{0}=0$ when $k\neq 0$. By Lemma \ref{sphere-isomorphism}, $$\tilde{H}_kS^n=\tilde{H}_{k-n}S^{0}=\begin{cases}G & n=k\\0 & \text{otherwise}\end{cases}$$
This implies the result.
\end{proof}

\begin{remark}
For the sake of these calculations, the choice of coefficients $H_0\cdot=\mathbb{Z}$ was arbitrary and played no role. The result could be written more generally as $H_nS^n=\begin{cases}H_0\cdot & n=k\\0 & \text{otherwise}\end{cases}$ for ordinary homology theories. This will be important in section (MISSING), where we use these results for a homology theory with coefficients $H_0 \cdot=\mathbb{Z}/2\mathbb{Z}$.
\end{remark}
\subsection{Real projective space $\RP{n}$}
Now we will look at the homologies of $\RP{n}$ with respect to an arbitrary (ordinary) homology. Recall that the puncture of $D^n$ is homotopy equivalent to $S^{n-1}$. Recall also that the puncture of $\RP{n}$ is homotopy equivalent to $\RP{n-1}$. Now consider the following diagram:

% https://tikzcd.yichuanshen.de/#N4Igdg9gJgpgziAXAbVABwnAlgFyxMJZARgBoAGAXVJADcBDAGwFcYkQAJAfTAAoAdfgCUACsAByAX1KDREgLTFJAShDT0mXPkIoyxanSat23PrLFSZwi5MFwYOALZYwzOAAJBwAFSCVa0g1sPAIiclJ9GgYWNkROHl4AEQA9MFIUsDsHZ1cPL19+f3UQDGDtMIoDaOM40yTU0gBlZOAwRSKDGCgAc3giUAAzACcIRyRwkBwIJDJDGPYAKwDBkbHEWamkACYoo1iQAAtlkGHRpABmGk3EcmLTtcvJ6Zvd+biB4-uLq+eduZqQFg1JRJEA
\begin{tikzcd}
{H_k(D^n,S^{n-1})} \arrow[r] \arrow[r, "f^*"] \arrow[d, "i^*"] & {H_k(\RP{n},\RP{n-1})} \arrow[d, "j^*"]              \\
{H_k(D^n,D^n\setminus \{*\})}                              & {H_k(\RP{n},\RP{n}\setminus \{*\})} \arrow[l, "h^*"]
\end{tikzcd}

$*$ is defined as the center of $D^n$ when considering $\RP{n}$ as the glue of $\RP{n-1}$ and $D^n$. $i^*$ and $j^*$ are the maps induced by the canonical inclusions. By the previous comment, $i$ and $j$ are homotopy equivalences, so $i^*$ and $j^*$ are isomorphisms. $h$ is the isomorphism guaranteed by the Excision Axiom after noting that $(D^n,D^n\setminus \{*\})$ can be found by excising everything but a smaller, closed n-disk in $D^n\subset \RP{n}$ from the other set. This open set then satisfies all the requirements of excision. We can therefore define $f^*$ as the unique isomorphism that makes the diagram commute. By a previous result, we have 

\[H_k(\RP{n},\RP{n-1})=H_k(D^n,S^{n-1})=
\begin{cases} 
      G & k=n \\
      0 & \text{otherwise}
   \end{cases}
\]

From this observation we can deduce a number of facts about $H_k(\RP{n})$.

\begin{lemma}
\label{projective-space-iso}
$H_k(\RP{n})\iso H_k(\RP{n-1})$ whenever $k\neq n,n-1$.
\end{lemma}
\begin{proof}
By assumption, $$H_k(\RP{n},\RP{n-1})=H_{k+1}(\RP{n},\RP{n-1})=0.$$
Therefore, the long exact homology sequence reads as

% https://tikzcd.yichuanshen.de/#N4Igdg9gJgpgziAXAbVABwnAlgFyxMJZABgBpiBdUkANwEMAbAVxiRGJAF9T1Nd9CKAIzkqtRizYAJAPoBrABQAdJQCUACsDABaIZwCUXHiAzY8BIgCZR1es1aIQsxSo1aDR3mYFEAzDfF7Ng5OMRgoAHN4IlAAMwAnCABbJDIQHAgkPWME5KzqDKRLbjjElMRrdMzEX1DOIA
\begin{tikzcd}
0 \arrow[r] & H_k(\RP{n-1}) \arrow[r] & H_k(\RP{n}) \arrow[r] & 0
\end{tikzcd}

which induces the necessary isomorphism.
\end{proof}

\begin{corollary}
\label{homology-large-n}
$H_k(\RP{n})=0$ when $k>n$ and $n\neq 0$.

Additionally, $H_0(\RP{n}=H_0 1:=G$.
\end{corollary}
\begin{proof}
By the previous lemma, we have, for $k>n, n\neq 1$

$$H_k(\RP{n})\iso H_k(\RP{n-1})\iso \dots \iso H_k(\RP{1})\iso H_k(S^1)=0$$

The second result has already been verified for $n=0,1$. For $n>1$ we can use the previous lemma again with $k=0$ to get 

$$H_0(\RP{n})\iso H_0(\RP{n-1})\iso \dots \iso H_0(\RP{1})\iso H_0(S^1)=G$$
\end{proof}

\begin{lemma}
\begin{enumerate}[i]
\item If $H_n(\RP{n})=0$ then $H_{n+1}(\RP{n+1})=G$ and $H_{n}(\RP{n+1})=0$.
\item If $H_n(\RP{n})=G$ and $n>0$, then $H_{n+1}(\RP{n+1})=0$.
\end{enumerate}
\end{lemma}

\begin{proof}
(i) Consider the following section of the long homology sequence for $(\RP{n+1},\RP{n})$:

% https://tikzcd.yichuanshen.de/#N4Igdg9gJgpgziAXAbVABwnAlgFyxMJZABgBpiBdUkANwEMAbAVxiRAAkB9YMAagEYAvgAoAOqIBKABR6CAlCEGl0mXPkIp+5KrUYs2XHgJHjpRoQqUrseAkQBM26vWatEHbnyFjJMr0tM-eUVlEAwbdSIAZiddVwNOMB8zMGCrMNVbDWQAFliXfXcuJMDzNNDwtTsUAFZ8vTcPHgBab1L-Una0nRgoAHN4IlAAMwAnCABbJDIQHAgkIVCxyYXqOaR7dOWpxEdZ+cQorfGdmP2kHOOVxDzzxBrBCkEgA
\begin{tikzcd}
H_{n+1}(\RP{n}) \arrow[r] & H_{n+1}(\RP{n+1}) \arrow[r] & {H_{n+1}(\RP{n+1},\RP{n})} \arrow[r] & H_n(\RP{n}) \arrow[r] & H_n(\RP{n+1}) \arrow[r] & {H_{n-1}(\RP{n+1},\RP{n})}
\end{tikzcd}

By assumption, \ref{homology-large-n} and \ref{projective-space-iso}, this reduces to:

% https://tikzcd.yichuanshen.de/#N4Igdg9gJgpgziAXAbVABwnAlgFyxMJZABgBpiBdUkANwEMAbAVxiRGJAF9T1Nd9CKAIzkqtRizYAJAPrAwAaiGcAFAB01AJQAK8pZwCUXHiAzY8BIgCZR1es1aIQAcWO9zAogGZb4h2w5ud35LFAAWX3tJJ1kwdS1dRWUjINM+C0FkAFZIiUd2LjEYKABzeCJQADMAJwgAWyQyEBwIJGUTGvq26hakK1TOhsQbZtbELwHaoZ9RpDDJrsQI2cQszgpOIA
\begin{tikzcd}
0 \arrow[r] & H_{n+1}(\RP{n+1}) \arrow[r] & G \arrow[r] & 0 \arrow[r] & H_n(\RP{n+1}) \arrow[r] & 0
\end{tikzcd}

This induces the two required isomorphisms.

(ii) In this case, the same diagram reduces to

% https://tikzcd.yichuanshen.de/#N4Igdg9gJgpgziAXAbVABwnAlgFyxMJZABgBpiBdUkANwEMAbAVxiRGJAF9T1Nd9CKAIzkqtRizYAJAPrAwAaiGcAFAB01AJQAK8pZwCUXHiAzY8BIgCZR1es1aIQAcWO9zAogGZb4h21dud35LFAAWX3tJJ1kwdS1dRWUjINM+C0FkAFZIiUd2LjEYKABzeCJQADMAJwgAWyQyEBwIJGUTGvq26hakK1TOhsQbZtbELwHaoZ9RpDDJrsQI2cQszgpOIA
\begin{tikzcd}
0 \arrow[r] & H_{n+1}(\RP{n+1}) \arrow[r] & G \arrow[r] & G \arrow[r] & H_n(\RP{n+1}) \arrow[r] & 0
\end{tikzcd}

If I can show the map $\partial: G\rightarrow G$ is injective, I have my result. (... MISSING)

\end{proof}
\section{Complex projective space}
The complex projective space, $\CP{n}$ is defined similarly to $\RP{n}$ as the space of complex lines in $\mathbb{C}^{n+1}$. Explicitly it is the quotient $\mathbb{C}^{n+1}\setminus \{0\} / \tilde$ where $z\tilde \lambda w$ for $\lamba \in \mathbb{C}$.

It is trivial to see that $\CP{0}=\cdot$, as any $z\tilde 1$ via multiplication by $\frac{1}{z}$. To understand $\CP{1}$, we can first see it as a quotient of $S^3$, thought of as sitting inside $\mathbb{C}^2$. This follows from the identification $$
\begin{bmatrix} z\\ w \end{bmatrix}\tilde \big|\begin{bmatrix} z\\ w \end{bmatrix}\big| \begin{bmatrix} z\\ w \end{bmatrix}:=\begin{bmatrix} \bar{z}\\ \bar{w} \end{bmatrix}$$
Here we are using the typical norm $$\big|\begin{bmatrix} x+iy\\ a+ib \end{bmatrix}\big|=\big| \begin{bmatrix} a\\ b\\ c\\ d \end{bmatrix}\big|$$
Next note that each $\begin{bmatrix} \bar{z}\\ \bar{w} \end{bmatrix}\in S^3$ is equivalent to exactly all elements in a circle $S^1\in S^3$. This is because $\{\lambda \begin{bmatrix} \bar{z}\\ \bar{w} \end{bmatrix}, |\lambda|=1\}$ is the set of all rotations of  $\begin{bmatrix} \bar{z}\\ \bar{w} \end{bmatrix}$ along some direction.

Consider the case $w\neq 0$. Then, by multiplying by $\frac{\bar{w}^*}{|\bar{w}|}$ we can choose a unique representative $\begin{bmatrix} \tilde{z}\\ \tilde{w} \end{bmatrix}:= \frac{\bar{w}^*}{|\bar{w}|}\begin{bmatrix}\bar{z}\\ \bar{w} \end{bmatrix}$ such that $\tilde{w}> 0$. This can then be identified as an element of the (open) upper disk of the equator $S^2\in S^3$, as the only restriction on $\tilde{z}$ is that $\big|\begin{bmatrix}\tilde{z}\\ \tilde{w} \end{bmatrix} \big|=1$. For example the element $\begin{bmatrix}\tilde{a} \\ \tilde{w}\end{bmatrix} \in S^2$ where $\tilde{w}>0$ is achieved in the previous construction by setting $z=a/(\frac{w^*}{|w|})$. When $w=0$, $\{\begin{bmatrix} \bar{z}\\ \bar{w} \end{bmatrix}\}=\CP{0}\cong \cdot$. It follows that $\CP{1}$ is the (closed) upper hemisphere of $S^2$, quotient its equator, which is homeomorphic to $S^2$. So as a cell complex, $\CP{1}$ has 1 0-cell, 1 1-cell and 2 2-cells in the usual way for $S^2$.

We can use this cell structure to define $\CP{2}$. As before, we can find a representative $\begin{bmatrix}\bar{z}\\ \bar{w}\\ \bar{u}\\ \end{bmatrix}\in S^5$. For $u=0$ this is $\CP{1}$. For $|u|>0$ we can multiply by $\frac{u^*}{|u|}$ to find a representative in the (open) upper hemisphere of the equator $S^4\in S^5$. As previously, this covers the whole open upper hemisphere of $S^4$, as there are no other restrictions on $\bar{z},\bar{w}$. Therefore $\CP{2}$ has the cell structure of $\CP{1}$ with an additional 4-cell. The gluing map is the projection $S^3\subset \mathbb{C}^2 \rightarrow \CP{1}$. Since $\CP{2}$ has no 3-cells, cellular homology asserts that $H_4(\CP{2})=\mathbb{Z}$. For $n\neq 4$, $H_n(\CP{2})=H_n(\S{2})$, as the cellular homology is untouched here. So $H_n(\CP{2})=\mathbb{Z}$ for $n=0,2,4$ and $0$ otherwise.

The general case can be done by induction.

\begin{theorem}
As a cell complex, $\CP{n}$ has 1 0-cell, 1 1-cell, 2 2-cells, and 1 2m-cell for $1<m\leq n$. As a consequence,
$$H_k(\CP{n})=\begin{cases} 
      \mathbb{Z} & k even, 0\leq (k/2)\leq n \\
      0 & \text{otherwise}
   \end{cases}
$$
\end{theorem}
\begin{proof}
We have proved the case $n=2$. Suppose the statement is true for some $n$. We will then prove it is true for $n+1$. As before, we can identify elements of $\CP{n+1}$ by representatives of $S^{2n+1}$ by dividing by norm. These can furthermore be identified as either elements of $\CP{n}$ as a quotient of the equator of $S^{2n}$ or elements of the open upper $2n$-cell of $S^{2n}$. The gluing map is the projection of $\partial D^{2n}$ onto $\CP{n}$. Therefore $\CP{n+1}$ has the cell structure of $\CP{n}$, with an additionall $2n$-cell. This proves the inductive case.

The statement on homology is then easily achieved by noting that $\CP{n}$ has no adjacent $n-cells$ for $n>2$, whence the homology groups for $n>2$ correspond exactly to the generators on $n-cells$. For $n\leq 2$ the homology groups of $\CP{n}$ are equal to that of $S^2$.
\end{proof}
\end{document}
