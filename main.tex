\documentclass[11pt,a4paper]{article}
\usepackage[utf8x]{inputenc}
\usepackage[T1]{fontenc}
\usepackage{mathptmx} % Use Times Font
%\usepackage{listings}
\usepackage{amsmath}
\usepackage{amssymb}
\usepackage{tikz-cd}
\usepackage{enumerate}
%\usepackage{mathtools}
%\usepackage{theoremref}
\usepackage{amsthm}
\newcounter{thm}
\numberwithin{thm}{section}
\newtheorem{theorem}{Theorem}[thm]
\newtheorem{corollary}{Corollary}[thm]
\newtheorem{lemma}[thm]{Lemma}
\newtheorem{prop}[thm]{Proposition}
\newtheorem{remark}[thm]{Remark}
\newtheorem{definition}[thm]{Definition}
\newtheorem{example}[thm]{Example}

\newcommand{\dsum}{\bigoplus}
\newcommand{\iso}{\cong}
\newcommand{\defn}[1]{\textit{#1}}
\newcommand{\cat}[1]{\mathcal{#1}}
\newcommand{\RP}[1]{\mathbb{RP}^{#1}}
\newcommand{\S}[1]{\mathbb{S}^{#1}}

\title{Axiomatic Homology Theory and the Borsuk-Ulam Theorem}
\author{Malthe Sporring}

\begin{document}
\maketitle

\section{Axioms}
\section{Chain Complexes}\label{sec-chain-complexes}
\begin{definition}
A \defn{chain complex} is a family $(A_i)_{i\in\mathbb{Z}}$ of abelian groups, as well as a family $(f_i:A_i\rightarrow A_{i+1})_{i\in \mathbb{Z}}$ of homomorphisms between consecutive groups, such that $im(f_i)\subset ker(f_{i+1})$. If we have equality instead of inclusion, the family is called an \defn{exact sequence}.
\end{definition}

We distinguish between \defn{short} and \defn{long exact sequences}, where short exact sequences are sequences with three or fewer consecutive non-zero groups, i.e. sequences of the form

% https://tikzcd.yichuanshen.de/#N4Igdg9gJgpgziAXAbVABwnAlgFyxMJZABgBpiBdUkANwEMAbAVxiRGJAF9T1Nd9CKAIzkqtRizYBBLjxAZseAkQBMo6vWatEIAEKzeigUQDM68VrYBhA-L5LByACznNknR05iYUAObwiUAAzACcIAFskMhAcCCQRC3d2W1CI+OpYpDVE7RAglLDIxGzMxDMctl8CtLKMuMQXCo8uCk4gA
\[\begin{tikzcd}
0 \arrow[r, "0"] & A \arrow[r, "f"] & B \arrow[r, "g"] & C \arrow[r, "0"] & 0
\end{tikzcd}\]
All other exact sequences are called long exact sequences. 

\begin{example}
For any abelian group $A$ the following is a long exact sequence:
% https://tikzcd.yichuanshen.de/#N4Igdg9gJgpgziAXAbVABwnAlgFyxMJZARgBoAGAXVJADcBDAGwFcYkQBBEAX1PU1z5CKAEwVqdJq3Zde-bHgJEAzOJoMWbRJx58QGBUKIAWNZM3sAOpagQcCOfoGLhycmY3Tt12-Z4SYKABzeCJQADMAJwgAWyR3EBwIJDJzLxAsKF0I6LjEVKSkMTStEHJskCjYoppCxFUS9kyKqrzTROTEBM9S8u5KbiA
\[\begin{tikzcd}
\dots \arrow[r, "0"] & A \arrow[r, "id"] & A \arrow[r, "0"] & A \arrow[r, "id"] & \dots
\end{tikzcd}\]
\end{example}

\begin{remark}
The requirement that $im(f_i)\subseteq ker(f_{i+1})$  is equivalent to the requirement that $f_{i+1}\circ f_i=0$. This is clear from the definition: if $im(f_i)\subseteq ker(f_{i+1})$ then $$\forall x\in A_i, f_{i+1}\circ f_i (x)\in f_{i+1}(im(f_i))=\{0\}.$$ Additionally, if $\forall x\in A_i, f_{i+1}\circ f_i=0$ then $f_{i+1}(im(f_i))=\{0\}$.
\end{remark}


\begin{example}
Suppose the following is part of a long exact sequence of groups.
% https://tikzcd.yichuanshen.de/#N4Igdg9gJgpgziAXAbVABwnAlgFyxMJZARgBoAGAXVJADcBDAGwFcYkQBBEAX1PU1z5CKAEwVqdJq3YAhHnxAZseAkQDM4mgxZtEIAMLz+yoUQAsmyTvYARI4oErhyAKyXt0vQB0vUCDgReY0FVFHJ3KV0QHz8AngkYKABzeCJQADMAJwgAWyRwkBwIJDIrTxByeyzckpoipDEyqPSq7LzERvrEDSb2SqCQavaerrMBoaQ3QuLEcm5KbiA
\[\begin{tikzcd}
\dots \arrow[r] & A \arrow[r, "0"] & B \arrow[r, "f"] & C \arrow[r, "0"] & D \arrow[r] & \dots
\end{tikzcd}\]
We say this \defn{gives rise to} the following short exact sequence, as they carry the same information.
% https://tikzcd.yichuanshen.de/#N4Igdg9gJgpgziAXAbVABwnAlgFyxMJZABgBpiBdUkANwEMAbAVxiRGJAF9T1Nd9CKAIzkqtRizYAhLjxAZseAkQBMo6vWatEIAMKzeigUQDM68VrYdOYmFADm8IqABmAJwgBbJCJA4ISGoWkjouBiDuXkhkfgGIQtyuHt6IQf5IJjacQA
\[\begin{tikzcd}
0 \arrow[r] & B \arrow[r, "f"] & C \arrow[r] & 0
\end{tikzcd}\]
Note we can omit specifying any homomorphisms from or into $0$, as there is only one: the $0$-homomorphism. Since $0=im(0)=ker(f)$, $f$ is injective. Since $im(f)=ker(0)=C$, $f$ is surjective. Therefore $f$ is an isomorphism.
\end{example}

\begin{definition}
If there exist maps $f:A\rightarrow B$ and $g:B\rightarrow A$ between abelian groups such that $g\circ f = id_A$ then $g$ is called a \defn{retraction} of $f$. If alternatively $f\circ g=id_B$, then $f$ is called a \defn{section} of $f$.
\end{definition}
This definition formalises the idea of a "one-sided inverse".

\begin{prop}\label{short-seq-direct-sum}
Let the following be a short exact sequence
% https://tikzcd.yichuanshen.de/#N4Igdg9gJgpgziAXAbVABwnAlgFyxMJZABgBpiBdUkANwEMAbAVxiRGJAF9T1Nd9CKACzkqtRizYduvbHgJEAjKOr1mrRCACCXHiAxyBRAEwrx6tgCFds-gpQBmM2smaAwlzEwoAc3hFQADMAJwgAWyQyEBwIJGMZEBDwpBFo2MRFBKSIxFM0pCdzV0SbRNCcwpiU1QkNEB9PTiA
\[\begin{tikzcd}
0 \arrow[r] & A \arrow[r, "f"] & B \arrow[r, "g"] & C \arrow[r] & 0
\end{tikzcd}\]
\begin{enumerate}[(a)]
    \item If there exists a \defn{section} $s:C\rightarrow B$ of $g$, then $f$ and $s$ define an isomorphism $$(f+s):A\dsum C\rightarrow B$$
$$(a,c)\mapsto f(a)+s(c).$$
\item If there exists a \defn{retraction} $r:B\rightarrow A$ of $f$, then $r$ and $g$ define an isomorphism $$(r,g):B\rightarrow A\dsum C$$
$$b\mapsto (r(b),g(b)).$$
\end{enumerate}
\end{prop}

\begin{proof}
\begin{enumerate}[(a)]
\item First we show that $(f+s)$ is injective. By exactness we have that $f$ is injective, as $0=im(0)=ker(f)$, and that $g$ is surjective, as $im(g)=ker(0)=C$. Suppose $f(a)+s(c)=0$. Then $$0=g(0)=g(f(a)+s(c))=gf(a)+gs(c)=c$$
Which implies $c=0$. But then $0=f(a)$, which implies $a=0$ as $f$ is injective. Therefore $ker(f+s)=0$, so $(f+s)$ is injective.

Next we show $(f+s)$ is surjective. Let $b\in B$, $c=g(b)\in C$ and $a\in A$ be the unique element that maps to $b-sg(b)\in B$. This element exists, since $$g(b-s(g(b)))=g(b)-g(b)=0,$$ so $b-sg(b)\in ker(g)=im(f)$. It is unique by the injectivity of $f$. It follows that $$f(a)+s(c)=b-sg(b)+sg(b)=b,$$ so $(f,s)$ is surjective. It is therefore an isomorphism.

\item First we show that $(r,g)$ is injective. Suppose $(r(b),g(b))=(0,0)$. Then $g(b)=0,$ so $b\in ker(g)=im(f)$. Now $r$ is injective on $im(f)$, since $rf=id_A$. Therefore $r(b)=0\implies b=0$. So $(r,g)$ is injective.

Next we show $(r,g)$ is surjective. Let $(a,c)\in A\dsum C$. Since $g$ is surjective, there exists $b\in B$ such that $c=g(b)$. Let $x=f(a)+b-fr(b)\in B$. Then $g(x)=g(b)$, as $gf(-)=0$ by exactness. Additionally, $$r(x)=rf(a)+r(b)-rfr(b)=a+r(b)-r(b)=a,$$
since $rf=id_A$. It follows that $(r,g)$ is also surjective, so it is an isomorphism.
\end{enumerate}

\end{proof}

The next proposition is not necessarily about chain complexes, but is very much in the flavour of Proposition \ref{short-seq-direct-sum} and will be very useful in our study.

\begin{proposition}\label{retraction-iso}
If $f:A\rightarrow B$ admits a retraction $g:B\rightarrow A$, then $$B\iso im(f)\dsum ker(g)$$
\end{proposition}

\begin{proof}
We define a homomorphism $$h:B\rightarrow im(f)\dsum ker(g)$$
$$b\mapsto (fg(b),b-fg(b))$$
This is well defined as $fg(b)\in im(f)$ trivially, and $g(b-fg(b))=g(b)-g(b)=0$.
First we show $h$ is injective. Let $(fg(b),b-fg(b))=(0,0)$. Then $b-fg(b)=b=0$, as $fg(b)=0$, so $h$ is injective. Next we show $h$ is surjective. Let $(x,y)\in im(f)\dsum ker(g)$, and $a\in A$ be such that $f(a)=x$. Then $$h(x+y)=(fgf(a)+fg(y),f(a)+y-fgf(a))=(f(a),y)=(x,y),$$
as $gf=id_A$ and $g(y)=0$.
It follows that $h$ is surjective, so it is in fact an isomorphism.\end{proof}

We finish this section with a technical lemma that will become useful in section \ref{sec-axioms}.

\begin{lemma}[Braid lemma]\label{braid-lemma}
Suppose three long exact sequences and a chain complex make the following commutative diagram.

% https://tikzcd.yichuanshen.de/#N4Igdg9gJgpgziAXAbVABwnAlgFyxMJZARgBpiBdUkANwEMAbAVxiRAEEQBfU9TXfIRRkAzFVqMWbAELdeIDNjwEiAVnLj6zVohABxOXyWC1pMdS1TdACUML+yocgBMpZ5sk6QAYTuKBKigiGhaebAAifg4mQWYe2mwAolHGgcgALG7xViAAYikBTgBsWaEJugCSBY5EAAyktdle1THI9e5lOQDyLWn16U1svcUNg7rDRCUDnc1c4jBQAObwRKAAZgBOEAC2SPUgOBBImRLlIGsA+sR2mzvH1IdIJac5l843W7uIz4+IwSAAIxgYCgSAAtOkAJwzNiXEQfO6IMgHI6IE6WLwACyuCK+J1+6heWIu7x460+SEJv1cgOBoLR0KJbGx8LJ5wpiH2BOoQJBx0ZGLYixxbNuX2Rv2evPpEIFYV0ACsRfIxU8HqiAOwwxUk3FILUopA0wU61kqjlUzXakDC0nmxEG37-E02i5m8mI-6-AAc1su6T1iBpPut2IDoo5jN++2lSBEtQjiOIXNR+xdwtqgeT6qQyJdSszifFEtRyNjiDB8aLRpzSJp5fj1qV4ftXy9pf+5craflrvDFC4QA
\begin{tikzcd}
{} \arrow[rd, bend left]              &                                                      &                                       &                                                      &                                       &                                                   & {} \\
                                      & A \arrow[rd, "f_1"] \arrow[rr, "g_1", bend left=49]  &                                       & D \arrow[rr, "h_3", bend left=49] \arrow[rd, "g_2"]  &                                       & G \arrow[rd, "h_4"] \arrow[ru, "j_4", bend left]  &    \\
O \arrow[ru, "g_0"] \arrow[rd, "j_0"] &                                                      & C \arrow[rd, "f_2"] \arrow[ru, "h_2"] &                                                      & F \arrow[ru, "j_3"] \arrow[rd, "g_3"] &                                                   & I  \\
                                      & B \arrow[ru, "h_1"] \arrow[rr, "j_1", bend right=49] &                                       & E \arrow[rr, "f_3", bend right=49] \arrow[ru, "j_2"] &                                       & H \arrow[ru, "f_4"] \arrow[rd, "g_4", bend right] &    \\
{} \arrow[ru, bend right]             &                                                      &                                       &                                                      &                                       &                                                   & {}
\end{tikzcd}

Then the chain complex is also a long exact sequence.
\end{lemma}
\begin{proof}
By symmetry of the diagram, it does not matter which sequence is the chain complex. We can assume it is the sequence with homomorphisms $f_i$. We are given that $im(f_i)\subseteq ker(f_{i+1})$, and need to show that $ker(f_{i+1})\subseteq im(f_i)$. By the symmetry of the diagram, it is enough to show this for $i=1,2,3$. We will show that $ker(f_2)\subseteq im(f_1)$ here, and do the other two cases in the Appendix.

Let $x\in ker(f_2)$. Then $0=f_2(x)=j_2f_2(x)=g_2h_2(x)$ by commutativity. It follows that $h_2(x)\in ker(g_2)=im(g_1)$. So $\exists x_1\in A$ s.t. $g_1 (x_1)=h_2(x)$. By commutativity, $g_1(x_1)=h_2f_1 (x_1)$. So we have that $0=g_1 (x_1)-h_2(x)=h_2(f_1 (x_1)-x)$. Let $x_2:=f_1(x_1)-x\in ker(h_2)=im(h_1)$. Then $\exists x_3\in B$ s.t. $h_1(x_3)=x_2$.

Now note that $$j_1(x_3)=f_2h_1(x_3)=f_2(x_2)=f_2(f_1(x_1)-x)=0,$$
where the last equality follows from $f_2f_1(-)=0$ and $f_2(x)=0$. We therefore have that $x_3\in ker(j_1)=im(j_0)$. So there exists $x_4\in O$ s.t. $j_0(x_4)=x_3$. Consider $g_0(x_4).$ It satisfies $f_1g_0(x_4)=h_1j_0(x_4)=h_1(x_3)=x_2=f_1(x_1)-x$. Therefore we have
$$x=f_1(x_1-g_0(x_4)).$$
This shows $x\in im(f_1)$ as required.
\cite{Eilenberg}
\end{proof}
An admissable category...

The axioms of (generic) homology theory...

\section{Basic results}
\begin{prop}
If $A\subset X$ is a deformation retract, then $H_n(X,A)=0$.
\end{prop}
\begin{proof}
If $A\subset X$ is a deformation retract, then the inclusion $i:A\rightarrow X$ is a homotopy equivalence. Let $r:X\rightarrow A$ be the retraction. Then $ir\simeq id_X$ and $ri\simeq id_A$.
By homotopy invariance of $H_n$ and the identity property of functors, $(ir)_*=(id_{X})_*=id_{H_nX}$ and $(ri)_*=(id_{A})_*=id_{H_nA}$. Since $(ri)_*=r_*i_*$ and vice-versa, we have that $i_*$ is an isomorphism.

Now consider the long exact homology chain:


% https://tikzcd.yichuanshen.de/#N4Igdg9gJgpgziAXAbVABwnAlgFyxMJZAJgBoAGAXVJADcBDAGwFcYkQAJAfTAEEQAvqXSZc+QigDMFanSat23MAA1BwkBmx4CRACwyaDFm0SceACmWleASjUit4ogFYDc44q7AwAWgCMAvxCDmI6KH5uRgqm3N4A1AGW1nbBGqLaEsgAbJHyJiAAOgVQEDgIqZqhmeS5HqZFJWWCsjBQAObwRKAAZgBOEAC2SPogOBBINe7RhQVo9L14TPYgfYMTNGNIEVP5WFwAVMurQ4jbm4hkO+wAVgdH-SeX59JX9bPzi4z3a4gv586pY5IHKjcaIXQCSgCIA
\begin{tikzcd}
\dots \arrow[r] & {H_{n+1}(X,A)} \arrow[r, "\partial"] & H_nA \arrow[r, "i_*"] & H_nX \arrow[r, "j_*"] & {H_n(X,A)} \arrow[r, "\partial"] & H_{n-1}A \arrow[r] & \dots
\end{tikzcd}

Since $i_*$ is an isomorphism, and the chain is exact, $H_nX=im(i_*)=ker(j_*)$ so $0=im(j_*)=ker(\partial)$

However, on the left we also have $0=ker(i_*)=im(\partial)$ since $i_*$ is an isomorphism. It follows that $H_n(X,A)=0$.

\end{proof}

\begin{remark}
\label{contractible-0}
As a special case of this result, $H_n(X,X)=0$, as $X$ is a deformation retract of itself. We are also interested in special case where $A=x$ is a single point, i.e. $X$ is contractible. Then we have $H_n(X,x)=0$.
\end{remark}

It is possible to reduce the abelian objects $H_nX$ into simpler objects $\tilde{H}_nX$ by in some sense factoring out the object $H_n1$. Furthermore, this can be done without losing any information, i.e. the transformation $H_nX\rightarrow \tilde{H}_nX$ is reversible.

First we will need some facts about abelian objects.

\begin{definition}
If there exist maps $f:A\rightarrow B$ and $g:B\rightarrow A$ between abelian objects such that $g\circ f = id_A$ then $g$ is called a \defn{retraction} of $f$, and $f$ is called a \defn{something} of $g$.
\end{definition}

$g$ can be thought of as a one-sided inverse of $f$, as there is no requirement that $g\circ f=id_B$.

\begin{lemma}
If $g:B\rightarrow A$ is a retraction of $f:A\rightarrow B$, then $B\iso im(f)\dsum ker(g)$ 
\end{lemma}

\begin{proof}
\label{direct-sum-iso}
The isomorphism is given by $$h:B\rightarrow im(f)\dsum ker(g)$$
$$x \mapsto (f\circ g(x),x-f\circ g(x))$$
This is well-defined as $f\circ g(x)\in im(f)$ and 
$$g(x-f\circ g(x))=g(x)-g\circ f \circ g(x)=g(x)-g(x)=0$$
by the associativity of homomorphisms, and since $g\circ f= id_A$. Hence $x-f\circ g(x)\in ker(g)$
The inverse is $$h^{-1}:im(f)\dsum ker(g)\rightarrow B$$
$$(a,b)\mapsto a+b$$
One quickly checks that
$$h\circ h^{-1}(a,b)=h(a+b)=(f\circ g(a+b),a+b - f\circ g(a+b))$$
$$=(f\circ g(a)+0,a+b-f\circ g(a))$$
since $b\in ker(g)$. However, $a=f(c)$ for some $c\in A$, so
$$(f\circ g(a)+0,a+b-f\circ g(a))=(f\circ g \circ f(c),a+b-f\circ g \circ f(c))$$
$$=(f(c),a+b-f(c))=(a,b)$$
since $f\circ g=id_B$. Additionally,
$$h^{-1}\circ h(x)=f\circ g(x)+x-f\circ g(x)=x$$
So $h$ and $h^{-1}$ are indeed inverse homomorphisms, and $B\iso im(f)\dsum ker(g)$.
\end{proof}

We will use this Lemma on the following construction.

\begin{definition}
$\tilde{H}_n(X)=ker(p^*:H_nX\rightarrow H_n1)$ where $p^*$ is the map induced by the initial map $p:X\rightarrow 1$. $\tilde{H}_nX)$ is called the \defn{reduced homology} of $X$.
\end{definition}

\begin{prop}
\label{reduced-homology}
For any $x\in X$,
$H_n(X,A)=\tilde{H}_n(X,A)\dsum H_n1=H_n(X,x)\dsum H_n1$
\end{prop}

\begin{proof}
For the first equality, consider the following diagram. $x$ exists by assumption MISSING of an admissable category, and $p$ exists since $1$ is an initial object. Notice $p$ is a retraction of $x$.

DIAGRAM

$H_n$ induces the following diagram, and since functors map compositions to compositions and identities to identities, we have that $p^*\circ x^*=id_{H_n1}$, so $p^*$ is a retraction of $x^*$.

DIAGRAM

By \ref{direct-sum-iso}, $$H_nX\iso im(x^*)\dsum ker(p^*)=im(x^*)\dsum H_n1\dsum \tilde{H}_nX$$
where the last equality holds because $p^*\circ x^* = id_{H_n1}$ guarantees that $x^*$ is injective.

For the second equality, note any two initial objects are isomorphic. In particular, for $x\in X$, $x\iso 1$. We hence have the following long exact chain

DIAGRAM

From which we can extract a short exact chain

DIAGRAM

By exactness, $H_n1\iso im(x^*)\iso ker(j*)$ and $im(j*)\iso ker(\partial)=H_n(X,x)$. Furthermore, by the first isomorphism theorem, $im(j*)\iso H_nX/ker(j*)$. Therefore, $$H_n(X,x)\iso H_nX/H_n1\iso \tilde{H}_nX\dsum H_n1/H_n1 \iso \tilde{H}_nX$$
where the last few isomorphisms are done somewhat informally, to be made precise at a future point. MISSING
\end{proof}

\begin{corollary}
If $X$ is contractible to $x\in X$, then $H_n(X)=H_n(X,x)\dsum H_n1=0\dsum H_n1\iso H_n1$, by \ref{contractible-0}. It follows that $\tilde{H}_nX=0$
\end{corollary}

\ref{reduced-homology} shows that $H_n(X,A)$ always carries around a copy of $H_n1$, which can be safely removed by going to the kernel of $p^*$. Gluing a copy of $H_n1$ onto $\tilde{H}_n(X,A)$ recovers the original object.

We would like to do manipulations using $\tilde{H}_nX$ instead of $H_n X$, as these spaces are simpler. To justify this, we should show that $\tilde{H}_n X$ forms a homology theory whenever $H_n$ does.

\begin{theorem}
If $H_n$ and $\partial$ form a homology theory over some category $C$, then for $\tilde{H}_nX$ there exist $\tilde{\partial}$ such that they form a homology theory as well.
\end{theorem}
\section{Homology of $\S{n}$}


\subsection{Real projective space $\RP{n}$}
Now we will look at the homologies of $\RP{n}$ with respect to an arbitrary (ordinary) homology. Recall that the puncture of $D^n$ is homotopy equivalent to $S^{n-1}$. Recall also that the puncture of $\RP{n}$ is homotopy equivalent to $\RP{n-1}$. Now consider the following diagram:

% https://tikzcd.yichuanshen.de/#N4Igdg9gJgpgziAXAbVABwnAlgFyxMJZARgBoAGAXVJADcBDAGwFcYkQAJAfTAAoAdfgCUACsAByAX1KDREgLTFJAShDT0mXPkIoyxanSat23PrLFSZwi5MFwYOALZYwzOAAJBwAFSCVa0g1sPAIiclJ9GgYWNkROHl4AEQA9MFIUsDsHZ1cPL19+f3UQDGDtMIoDaOM40yTU0gBlZOAwRSKDGCgAc3giUAAzACcIRyRwkBwIJDJDGPYAKwDBkbHEWamkACYoo1iQAAtlkGHRpABmGk3EcmLTtcvJ6Zvd+biB4-uLq+eduZqQFg1JRJEA
\begin{tikzcd}
{H_k(D^n,S^{n-1})} \arrow[r] \arrow[r, "f^*"] \arrow[d, "i^*"] & {H_k(\RP{n},\RP{n-1})} \arrow[d, "j^*"]              \\
{H_k(D^n,D^n\setminus \{*\})}                              & {H_k(\RP{n},\RP{n}\setminus \{*\})} \arrow[l, "h^*"]
\end{tikzcd}

$*$ is defined as the center of $D^n$ when considering $\RP{n}$ as the glue of $\RP{n-1}$ and $D^n$. $i^*$ and $j^*$ are the maps induced by the canonical inclusions. By the previous comment, $i$ and $j$ are homotopy equivalences, so $i^*$ and $j^*$ are isomorphisms. $h$ is the isomorphism guaranteed by the Excision Axiom after noting that $(D^n,D^n\setminus \{*\})$ can be found by excising everything but a smaller, closed n-disk in $D^n\subset \RP{n}$ from the other set. This open set then satisfies all the requirements of excision. We can therefore define $f^*$ as the unique isomorphism that makes the diagram commute. By a previous result, we have 

\[H_k(\RP{n},\RP{n-1})=H_k(D^n,S^{n-1})=
\begin{cases} 
      G & k=n \\
      0 & \text{otherwise}
   \end{cases}
\]

From this observation we can deduce a number of facts about $H_k(\RP{n})$.

\begin{lemma}
\label{projective-space-iso}
$H_k(\RP{n})\iso H_k(\RP{n-1})$ whenever $k\neq n,n-1$.
\end{lemma}
\begin{proof}
By assumption, $$H_k(\RP{n},\RP{n-1})=H_{k+1}(\RP{n},\RP{n-1})=0.$$
Therefore, the long exact homology sequence reads as

% https://tikzcd.yichuanshen.de/#N4Igdg9gJgpgziAXAbVABwnAlgFyxMJZABgBpiBdUkANwEMAbAVxiRGJAF9T1Nd9CKAIzkqtRizYAJAPoBrABQAdJQCUACsDABaIZwCUXHiAzY8BIgCZR1es1aIQsxSo1aDR3mYFEAzDfF7Ng5OMRgoAHN4IlAAMwAnCABbJDIQHAgkPWME5KzqDKRLbjjElMRrdMzEX1DOIA
\begin{tikzcd}
0 \arrow[r] & H_k(\RP{n-1}) \arrow[r] & H_k(\RP{n}) \arrow[r] & 0
\end{tikzcd}

which induces the necessary isomorphism.
\end{proof}

\begin{corollary}
\label{homology-large-n}
$H_k(\RP{n})=0$ when $k>n$ and $n\neq 0$.

Additionally, $H_0(\RP{n}=H_0 1:=G$.
\end{corollary}
\begin{proof}
By the previous lemma, we have, for $k>n, n\neq 1$

$$H_k(\RP{n})\iso H_k(\RP{n-1})\iso \dots \iso H_k(\RP{1})\iso H_k(S^1)=0$$

The second result has already been verified for $n=0,1$. For $n>1$ we can use the previous lemma again with $k=0$ to get 

$$H_0(\RP{n})\iso H_0(\RP{n-1})\iso \dots \iso H_0(\RP{1})\iso H_0(S^1)=G$$
\end{proof}

\begin{lemma}
\begin{enumerate}[i]
\item If $H_n(\RP{n})=0$ then $H_{n+1}(\RP{n+1})=G$ and $H_{n}(\RP{n+1})=0$.
\item If $H_n(\RP{n})=G$ and $n>0$, then $H_{n+1}(\RP{n+1})=0$.
\end{enumerate}
\end{lemma}

\begin{proof}
(i) Consider the following section of the long homology sequence for $(\RP{n+1},\RP{n})$:

% https://tikzcd.yichuanshen.de/#N4Igdg9gJgpgziAXAbVABwnAlgFyxMJZABgBpiBdUkANwEMAbAVxiRAAkB9YMAagEYAvgAoAOqIBKABR6CAlCEGl0mXPkIp+5KrUYs2XHgJHjpRoQqUrseAkQBM26vWatEHbnyFjJMr0tM-eUVlEAwbdSIAZiddVwNOMB8zMGCrMNVbDWQAFliXfXcuJMDzNNDwtTsUAFZ8vTcPHgBab1L-Una0nRgoAHN4IlAAMwAnCABbJDIQHAgkIVCxyYXqOaR7dOWpxEdZ+cQorfGdmP2kHOOVxDzzxBrBCkEgA
\begin{tikzcd}
H_{n+1}(\RP{n}) \arrow[r] & H_{n+1}(\RP{n+1}) \arrow[r] & {H_{n+1}(\RP{n+1},\RP{n})} \arrow[r] & H_n(\RP{n}) \arrow[r] & H_n(\RP{n+1}) \arrow[r] & {H_{n-1}(\RP{n+1},\RP{n})}
\end{tikzcd}

By assumption, \ref{homology-large-n} and \ref{projective-space-iso}, this reduces to:

% https://tikzcd.yichuanshen.de/#N4Igdg9gJgpgziAXAbVABwnAlgFyxMJZABgBpiBdUkANwEMAbAVxiRGJAF9T1Nd9CKAIzkqtRizYAJAPrAwAaiGcAFAB01AJQAK8pZwCUXHiAzY8BIgCZR1es1aIQAcWO9zAogGZb4h2w5ud35LFAAWX3tJJ1kwdS1dRWUjINM+C0FkAFZIiUd2LjEYKABzeCJQADMAJwgAWyQyEBwIJGUTGvq26hakK1TOhsQbZtbELwHaoZ9RpDDJrsQI2cQszgpOIA
\begin{tikzcd}
0 \arrow[r] & H_{n+1}(\RP{n+1}) \arrow[r] & G \arrow[r] & 0 \arrow[r] & H_n(\RP{n+1}) \arrow[r] & 0
\end{tikzcd}

This induces the two required isomorphisms.

(ii) In this case, the same diagram reduces to

% https://tikzcd.yichuanshen.de/#N4Igdg9gJgpgziAXAbVABwnAlgFyxMJZABgBpiBdUkANwEMAbAVxiRGJAF9T1Nd9CKAIzkqtRizYAJAPrAwAaiGcAFAB01AJQAK8pZwCUXHiAzY8BIgCZR1es1aIQAcWO9zAogGZb4h21dud35LFAAWX3tJJ1kwdS1dRWUjINM+C0FkAFZIiUd2LjEYKABzeCJQADMAJwgAWyQyEBwIJGUTGvq26hakK1TOhsQbZtbELwHaoZ9RpDDJrsQI2cQszgpOIA
\begin{tikzcd}
0 \arrow[r] & H_{n+1}(\RP{n+1}) \arrow[r] & G \arrow[r] & G \arrow[r] & H_n(\RP{n+1}) \arrow[r] & 0
\end{tikzcd}

If I can show the map $\partial: G\rightarrow G$ is injective, I have my result. (... MISSING)

\end{proof}
\end{document}
