\section{Notes and acknowledgements}
This text provides an introduction to homology theory from an axiomatic point of view. References are provided where appropriate. The text assumes knowledge of the basic ideas of algebraic topology. In particular, the reader is assumed to be aware of \textit{homotopy, homotopy equivalence, quotient spaces} and \textit{fundamental groups}.

This project was supervised by Clark Barwick, to whom I am most grateful. I am also grateful to the University of Edinburgh School of Maths Vacation Scholarship and College Vacation Scholarship funds for funding this project.

\section{Introduction}
One of the goals of topology is classifying spaces up to some definition of equivalence, typically homotopy equivalence or the stricter homeomorphism. While establishing equivalence requires the explicit construction of a homotopy equivalence/homeomorphism, inequivalence can in many cases be determined more easily by calculating \textit{topological invariants}. These are properties of a space that are preserved by our definition of equivalence, thus if two spaces have different invariants, they cannot be equivalent. One of the first 