\section{Notes and acknowledgements}
This text provides an introduction to homology theory from an axiomatic point of view. The reader is assumed to have knowledge of undergraduate level Algebra and Algebraic Topology, particularly \textit{groups, homotopies, homotopy equivalences} and \textit{quotient spaces}.

I am very grateful to Clark Barwick for supervising this project and for sharing both his knowledge and passion for the subject. I also wish to extend my gratitude to the University of Edinburgh School of Mathematics Vacation Scholarship and College Vacation Scholarship funds for funding this project.
\section{Introduction}
One of the goals of algebraic topology is classifying spaces up to some definition of equivalence, typically homotopy equivalence. Establishing equivalence can be tedious, as it often requires the construction of an explicit homotopy equivalence. However, \textit{inequivalence} is often easier to establish by calculating \defn{homotopy invariants}. These are properties of a space that are preserved by homotopy equivalences, hence if two spaces have different invariants, they cannot be homotopy equivalent. One of the first major homotopy invariances the undergraduate student encounters are the homotopy groups $\pi_n(X),n\in \mathbb{N}\cup \{0\}$: the groups of maps $f:S^{n}\rightarrow X$ with some base point. The homotopy groups carry a lot of geometric information, but are notoriously difficult to compute. Even for a simple space like the $2$-sphere $S^2$, it is not obvious that there are non-trivial maps $S^3\rightarrow S^2$ (we show that such a map exists in section \ref{sec-complex-projective-space}), and it is even less obvious what the higher homotopy groups are. The groups $\pi_n(S^2)$ do not seem to follow a pattern, and remain an active area of research. Only recently in 2015 was it proven that $\pi_n(S^2)$ is not zero for all $n\geq 2$ \cite{Ivanov}.

To avoid these complications, we would like to find homotopy invariances that are easier to compute, and ideally, not at the cost of too much geometric information. The homology groups $H_n(X)$ are an example of such a homotopy invariant. Like the homotopy groups, they are a sequence of groups, one for each $n\in \mathbb{Z}$, but they are much simpler and easier to compute. For example, the spheres have the simple structure $$H_n(S^m)=\begin{cases}\mathbb{Z} & n=0,m\\ 0 & \text{otherwise}\end{cases}$$
Part of the reason why they are easier to compute is that they come with extra structure, most importantly a \defn{long exact sequence}, which, broadly speaking, relates the homology groups $H_nX$ to each other and to the homology groups $H_nA$ of a subspace $A\subseteq X$. Therefore, the more you know about some of the homology groups of a space and a subspace, the easier it is to calculate the rest.

Historically, homology groups were produced from a number of geometric methods. It was Eilenberg and Steenrod who united the different homology theories by laying out a set of axioms that all homology theories satisfy \cite{Eilenberg}. In this text, we will take such an axiomatic approach, proving all results directly from the axioms. In some ways this approach best captures the essence of homology: the main task of a geometric approach to homology is to prove the Eilenberg-Steenrod axioms for that approach, and in practical calculations the axioms are often preferred over the geometric construction. However, this approach is not without its disadvantages. For the results in this text to be true we have to take given that there exists at least one homology theory which satisfies the axioms, and proving this is a major undertaking worth its own project. Singular Homology is an example of a homology theory which satisfies our assumptions, and the reader is invited to confirm that it satisfies the axioms in \cite{Hatcher}.

Homology theory is best understood in the language of category theory and chain complexes, which sections \ref{sec-category-theory} and \ref{sec-chain-complexes} are devoted to. In section \ref{sec-axioms}, we lay out the Eilenberg-Steenrod axioms and prove some immediate results for a general ordinary homology theory, most importantly the homology groups of the $n$-sphere. In the following sections we make the choice $H_0 \bullet=\mathbb{Z}$, which corresponds to Singular Homology. In sections \ref{sec-mayer-vietoris}, \ref{sec-degree-maps} and \ref{sec-cellular-homology} we lay out three practical methods for calculating homology groups: The Mayer-Vietoris Sequence, degree maps and Cellular Homology, and use them to prove some fascinating results.