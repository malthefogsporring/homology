\section{Notes and acknowledgements}
This text provides an introduction to homology theory from an axiomatic point of view. References are provided where appropriate. The text assumes knowledge of the basic ideas of algebra algebraic topology.s In particular, the reader is assumed to be aware of \textit{groups, group homomorphisms, homeomorphisms, homotopies, homotopy equivalences} and \textit{quotient spaces}.

This project was supervised by Clark Barwick, to whom I am most grateful. I am also grateful to the University of Edinburgh School of Mathematics Vacation Scholarship and College Vacation Scholarship funds for funding this project.

\section{Introduction}
One of the goals of topology is classifying spaces up to some definition of equivalence, typically homotopy equivalence or the stricter homeomorphism. While establishing equivalence requires the explicit construction of a homotopy equivalence/homeomorphism, inequivalence can in many cases be determined more easily by calculating \textit{topological invariants}. These are properties of a space that are preserved by our definition of equivalence, thus if two spaces have different invariants, they cannot be equivalent. One of the first major homotopy invariances the undergraduate student encounters are the homotopy groups $\pi_n(X)$, the groups of loops $f:\S{n}\rightarrow X$. While these groups carry a lot of geometric information, they can be very difficult to compute; the problem of computing $\pi_n(\S{m})$ for $n,m\in \mathbb{Z}$ remains unsolved. As of 2021, only a handful of these groups have been calculated, and this only after the development of several new branches of topology (REF NEEDED).

This motivates the investigation of other homotopy invariances. These should be much easier to compute, and ideally, not at the cost of too much geometric information. The homology groups $H_n(X)$ are an example of such a homotopy invariant. They are best explained in the language of category theory and exact sequences, discussed in sections MISSING and MISSING, so we will fall short of giving a definition just yet.

Historically, homology groups were calculated using one of a number of geometric methods (REF MISSING). It was in YEAR MISSING, that Eilenberg and Steenrod noticed a common thread between these different theories, and defined a set of axioms for what a homology theory should be. In this text we will take an axiomatic approach to homology, setting out the Eilenberg-Steenrod axioms and proving results directly. We will take on faith that at least one homology theory satisfying these axioms exist, which the reader is invited to confirm for themselves in (REF MISSING). While this approach comes at the cost of some geometric intuition, it comes with several advantages:
\begin{itemize}
\item For many homology theories, the geometric constructions are mainly used to give proofs of the Eilenberg-Steenrod axioms. In proving individual theorems, the axioms are often used more often than the individual geometric definitions.
\item Individual homology theories are equivalent up to which axioms they satisfy and a certain set of "initial conditions" (REF NEEDED). Therefore, in some sense the axioms are not only necessary but also sufficient for defining (a class of) homology theories.
\item There are now a large number of homology theories, which can all be understood from the axioms.
\item Proving that a homology theory satisfies the axioms is a major undertaking worth its own project.
\end{itemize}

A brief outline of the project is as follows: In sections REF and REF we give the key definitions and theorems of category theory and exact sequences, which will serve as the language of homology theory. In section REF, we give the Eilenberg-Steenrod axioms and show some immediate results. Techniques for further calculation are given in sections REF and REF, along with examples of results. This includes the Borsuk-Ulam Theorem, which initially served as a motivator for the project. MISSING REST OF SECTIONS