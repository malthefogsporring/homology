\section{Cellular homology}\label{sec-cellular-homology}
\subsection{CW-complexes}
A useful class of spaces is the class of \defn{cell complexes} or \defn{CW complexes}, which are spaces constructed by iteratively gluing copies of $D^n$ in a manner defined by a "gluing map" on the boundary $\partial D^n \iso S^{n-1}$. As will become apparent, CW-complexes have the right structure for establishing a very practical way of calculating their homology groups.

\begin{definition}
A CW-complex $X$ is the union of a sequence of spaces $X^n$, called the \defn{n-skeleta} of $X^n$ defined as follows: $X^0$ is a discrete set, and for each $X^{n-1}$, $X^{n}$ is obtained by gluing copies of $D_{\alpha}^n$, called $n$-cells, along a \defn{gluing map} $\phi_{\alpha}:S_{\alpha}^{n-1}\rightarrow X^{n-1}$ defined on the boundary of $D_{\alpha}^n$. Explicitly, $X^n$ is the quotient of the disjoint union $X^{n-1}\sqcup_{\alpha} D_{\alpha}^n/\sim$ under the identification $x\sim \phi_{\alpha}(x)$ for $x\in S_{\alpha}^{n-1}\subset D_{\alpha}^n$. If $X=X^n$ for some $n\in \mathbb{N}\cup\{0\}$, then $X$ is called a \defn{finite cell complex}. Otherwise, $X$ is given the weak topology: a subset $A\subset X$ is open iff $A$ is open in $X^n$ for every $n\in \mathbb{N}\cup \{0\}$.
\end{definition}

Many familiar spaces naturally arise as cell complexes.

\begin{example}
$S^{n}$ is a CW-complex with 1 0-cell and 1 $n$-cell when $n>0$. 
\end{example}

\begin{example}
$\RP{2}$ is a CW-complex with 1 0-cell, 1 1-cell and 1 2-cell, and where the gluing map $\phi:S^1\rightarrow X^1=\RP{1}$ is the projection $z\mapsto [z]$. Iteratively, $\RP{n}$ can be understood as a copy of $D^n$ glued to a copy of $\RP{n-1}$ via the projection map $z\mapsto [z]$ along the boundary $\partial D^n$. So $\RP{n}$ has a $k$-cell for $0\leq k \leq n$.
\end{example}

\begin{example}
The torus $T^2$ has a cellular structure composed of 1 0-cell, 2 1-cells and 1 2-cell. We can use the familiar identification of $T^2$ as a quotient of the square, where the two $1$-cells $a$ and $b$ have their endpoints glued to a single $0$-cell and the corners are identified to a single point. 

\begin{center}
\begin{tikzpicture}[scale=1]
    \fill [gray!40] (0,0) rectangle (3,3);
    \filldraw [black] (0,0) circle (1.5pt) node [left,scale=1] {$x$} 
    -- (3,0) circle (1.5pt) node [right,scale=1] {$x$} 
    node[currarrow,pos=0.5, xscale=1,sloped,scale=1.5]{}
    node[pos=0.5,below,scale=1]{$a$} 
    -- (3,3) circle (1.5pt) node [right,scale=1] {$x$} 
    node[currarrow,pos=0.5, xscale=1,sloped,scale=1.5]{} node[pos=0.5,right,scale=1]{$b$}
    -- (0,3) circle (1.5pt) node [left,scale=1] {$x$} 
    node[currarrow,pos=0.5, xscale=1,sloped,scale=1.5]{}
    node[pos=0.5,above,scale=1]{$a$}
    -- (0,0) circle (0pt) 
    node[currarrow,pos=0.5, xscale=-1,sloped,scale=1.5]{}
    node[pos=0.5,left,scale=1]{$b$};
\end{tikzpicture}
\end{center}
The gluing map $\phi:S^1\rightarrow X^1$ is the concatenation $a\cdot b\cdot -a \cdot -b$, where we are thinking of $a$ and $b$ as paths.
\end{example}

\begin{remark}
The cell structure of a CW-complex need not be unique. For example, an alternative cell structure of $S^2$ is with one 0-cell, 1 1-cell, and 2 2-cells (the upper and lower hemispheres) both glued onto the equator via the identity map on their boundaries.
\end{remark}

\subsection{Cellular homology groups}

We will show that the homology groups of CW-complexes can be identified as the homology groups of a certain chain complex of relative homology groups $H_n(X^n,X^{n-1})$. By "homology group of a chain complex" we mean the following:

\begin{definition}
Let the following diagram be a chain complex of abelian groups.

% https://tikzcd.yichuanshen.de/#N4Igdg9gJgpgziAXAbVABwnAlgFyxMJZARgBoAGAXVJADcBDAGwFcYkQBBAfWDAGpiAXxCDS6TLnyEUAJgrU6TVu269ho8djwEiAZnk0GLNok48wAWiEixIDFqlEALAcXH2AHQ9QIOBBrsJbWlkclcjZVMvHz8RBRgoAHN4IlAAMwAnCABbJDCQHAgkMjdIkChzAXVbTJzimkKkOVKTcvNq9KzcxGbGxH0W9greKw6QWu6XAqLEckFKQSA
\[\begin{tikzcd}
\dots \arrow[r] & A_{n+1} \arrow[r, "d_{n+1}"] & A_{n} \arrow[r, "d_{n}"] & A_{n-1} \arrow[r, "d_{n-1}"] & \dots
\end{tikzcd}\]
The abelian group $\tilde{H}_n(A_n):=Ker(d_n)/Im(d_{n+1})$ is called the \defn{homology group of the chain complex}.
\end{definition}

\begin{remark}
Note both $Ker(d_n)$ and $Im(d_{n+1})$ are abelian subgroups, and $Im(d_{n+1}) \subseteq Ker(d_n)$ as we are dealing with a chain complex. By \cite{proof-wiki}, a subgroup of an abelian group is normal, so $Ker(d_n)/Im(d_{n+1})$ is well-defined and is abelian.
\end{remark}

We will restrict our study to finite cell complexes, but the reader is invited to confirm that the established results also hold for general cell complexes \cite{Hatcher}. The proof is adapted from \cite{Hatcher}, with axiomatic replacements to arguments based on Singular Homology. We first establish some basic results.

\begin{lemma}\label{cell-lemma}
The following hold for a finite cell complex $X$:
\begin{enumerate}[(i)]
\item $H_n(X^n,X^{n-1})=\mathbb{Z}^m$ where $m$ is the number of n-cells of $X$ and $n\in \mathbb{N}\cup \{0\}$.
\item $H_m(X^n)=0$ for $m>n$.
\item The inclusion $X^n\xhookrightarrow{i} X$ gives rise to an isomorphism $H_k(X^n)\iso H_k(X)$ whenever $k>n$.
\end{enumerate}
\end{lemma}

\begin{proof}
\begin{enumerate}[(i)]
    \item The statement is trivial for $n=0$. For $n>0$, notice that $X^{n-1}$ is a deformation retract of $A:=X^{n}\setminus \sqcup_m \bullet$, where the $m$ copies of $\bullet$ are the centers of the $m$ $n$-cells of $X$. The subset $B:= X^{n}\setminus \sqcup_m D_{1/2}^n$ satisfies the conditions of excision, where $D_{1/2}^n$ is the disk of radius $\frac{1}{2}$ in the center of an $n$-cell.
It follows that
$$H_k(X^n,X^{n-1})\iso H_k(X^n,A)\iso H_k(X^n\setminus B,A\setminus B)\iso H_k(\sqcup_m D^n,\sqcup_m S^{n-1})$$
These homology groups are familiar: they are $\mathbb{Z}^m$ if $k=n$, and $0$ otherwise, by the proof of Proposition \ref{sphere-isomorphism} and an application of Proposition \ref{disjoint-union}.
\item By the previous result, the homology sequence for $(X^n,X^{n-1})$ reads% https://tikzcd.yichuanshen.de/#N4Igdg9gJgpgziAXAbVABwnAlgFyxMJZABgBpiBdUkANwEMAbAVxiRGJAF9T1Nd9CKAIzkqtRizYAJAPrAA1pwAaAPWBhOXHiAzY8BIgCZR1es1aIQshZwAUq9QGohnAJRbeegUQDMJ8eZsHJxiMFAA5vBEoABmAE4QALZIZCA4EEgu2vFJmdTpSIbcsQnJiMZpGYg+IZxAA
\[\begin{tikzcd}
0 \arrow[r] & H_{m}X^{n-1} \arrow[r] & H_{m}(X^{n}) \arrow[r] & 0
\end{tikzcd}\]
whenever $m\neq n,n-1$. Therefore, for $m>n$ we have $$H_m{X^n}\iso H_m{X^{n-1}}\iso \dots \iso H_m{X^0}=0.$$
\item By the same exact sequence, when $m<n$ we have that $$H_m(X^n)\xhookrightarrow{i} H_m(X^{n+1})$$ is an isomorphism. By induction, we get a chain of inclusions, all of which are isomorphisms:

$$H_m(X^n)\xhookrightarrow{i} H_m(X^{n+1})\xhookrightarrow{i}\dots \xhookrightarrow{i} H_m(X^{m})$$

Since the inclusion $H_m(X^n)\xhookrightarrow{i} H_m(X^{m})$ is the composition of the above inclusions, it too is an isomorphism. For finite cell complexes, $X=X^m$ for some $m$, proving (iii).
\end{enumerate}
\end{proof}

\begin{theorem}
The following is a chain complex, where the map $d_n$ is defined as the composition $H_n(X^n,X^{n-1})\xrightarrow{\partial_n}H_{n-1}(X^{n-1})\xhookrightarrow{i} H_{n-1}(X^{n-1},X^{n-2})$:

% https://tikzcd.yichuanshen.de/#N4Igdg9gJgpgziAXAbVABwnAlgFyxMJZABgBpiBdUkANwEMAbAVxiRAB12oIcEBfUuky58hFAEZyVWoxZsAEgH1gYPgAoAGgD0wpbSoC04vgEoQAodjwEiAJinV6zVohBLDxzVo8D9YA7am5oIgGFaiRADMDjLObJzcvObSMFAA5vBEoABmAE4QALZIZCA4EEiSsXKuUMpgANTGwTn5RYiVZUj2VS4gtYQWIHmFXdSdiNE9bLU+yXxAA
\[\begin{tikzcd}
\dots \arrow[r, "d_{n+1}"] & {H_{n}(X^n,X^{n-1})} \arrow[r, "d_n"] & {H_{n-1}(X^{n-1},X^{n-2})} \arrow[r, "d_{n-1}"] & \dots
\end{tikzcd}\]
Furthermore, $Ker(d_n)/Im(d_{n+1})  \iso H_n(X)$
\end{theorem}

\begin{proof}
The relative homology sequences of $X^n$ and $X^{n-1}$ for all $n\in \mathbb{N}$ fit into the following commutative diagram:

% https://tikzcd.yichuanshen.de/#N4Igdg9gJgpgziAXAbVABwnAlgFyxMJZARgBoBmAXVJADcBDAGwFcYkQAJAfWDAGpiAXwAUADQB6vAYNITeggJQgZ6TLnyEU5CtTpNW7brwC0QsZLCmZcywCZFy0qux4CRW6Vu6GLNok5cYOZgSiogGC4aRNrE3vp+AUE20qFO4WqumsgADKQALHG+7AA6xVAQOAhhEepuKHmk2YUG-txJ4mCyFlapzrVZHrE0Pi2J5iZCvemRdcgeVMPxhoHBXRMO1RlRKB4Fi0WtPJZmNj2OfZlEDQt6ByCl5ZXn0-1EZACszQlGxyKn9lMapd6p4viUyhUqmkgdscjp9qMHpDlLoYFAAObwIigABmACcIABbJC5EA4CBIDy3RHFND0PF4JhHaTPfFEyk0clIbTUhJYVkE4mIBpkimIbJhNlCnlcxDvSWCpAANk5YvsaSlJNVSAA7AiElBmUIBezEFTZXreewAFYmoWW2UADn14LpDKwTMICtNDrFZCt-kNXo1isQztFSH9IwSjDtke1iAAnC7A0crHHEMRSU7vUL-bLiEIQ6biOaxRLKIIgA
% https://tikzcd.yichuanshen.de/#N4Igdg9gJgpgziAXAbVABwnAlgFyxMJZARgBoBmAXVJADcBDAGwFcYkQAJAfWDAGpiAXwAUADQB6vAYNITeggJQgZ6TLnyEU5CtTpNW7brwC0QsZLCmZcywCZFy0qux4CRW6Vu6GLNok5cYOZgSiogGC4aRNrE3vp+AUE20qFO4WqumsgADKQALHG+7I7O6m4oeaTZhQb+3EniYLIWVqmlme6ksTQ+tYnmJkJt6ZHlyB5UPfGGgcHNgw5hEWVZHgVTRXU8lmY2rSUjK0SVk3qbIAfLHShkAKw1CUY7Inv2w1dRFZ4PxUsZnzkdBs+spdDAoABzeBEUAAMwAThAALZIXIgHAQJAeM59AA6uLQ9HheCY22kBwRyKxNAxSG0OISWApiJRiEq6MxiGyYUprPptMQtx5LKQADYaZz7GleaiJUgAOzAhJQMlCZlUxDYgWKhnsABW6tZOoFAA4lex8YTiVhSYRhRrjZyyLr-Cq7dKRYgzRykM7eglGIbfXLEABOc2u7ZWIOIYho0321nOgXEIQejXELWc7mUQRAA
\[\begin{tikzcd}[row sep =normal, column sep=small]
              &                                                                            &                                                             &                                                            & {H_n(X^n,X^{n-1})} \\
              &                                                                            & H_n(X^{n-1}) \arrow[d]                                      & H_n(X^{n+1}) \arrow[ru]                                    &                    \\
              &                                                                            & H_n(X^n) \arrow[ru, "i"] \arrow[d, "j"]                     &                                                            & {}                 \\
{} \arrow[r]  & {H_{n+1}(X^{n+1},X^{n})} \arrow[ru, "\partial_{n+1}"] \arrow[r, "d_{n+1}"] & {H_n(X^n,X^{n-1})} \arrow[d, "\partial_n"] \arrow[r, "d_n"] & {H_{n-1}(X^{n-1},X^{n-2})} \arrow[r, "d_{n-1}"] \arrow[ru] & {}                 \\
{} \arrow[ru] &                                                                            & H_{n-1}(X^{n-1}) \arrow[ru, "l"]                            &                                                            &                    \\
              & H_{n-1}(X^{n-2}) \arrow[ru]                                                &                                                             &                                                            &                   
\end{tikzcd}\]
Via Lemma \ref{cell-lemma}, we can reduce some groups to $0$, and make an identification with $H_n(X)$:
% https://tikzcd.yichuanshen.de/#N4Igdg9gJgpgziAXAbVABwnAlgFyxMJZARgBoBmAXVJADcBDAGwFcYkQAJAfWDAGpiAXwAUADQB6vAYNITeggJQgZ6TLnyEU5CtTpNW7brwC0QsZLCmZcywCZFy0qux4CRW6Vu6GLNok5cYOZgSiogGC4aRNrE3vp+AUE20qFO4WqumsgADKQALHG+7I7O6m4oeaTZhQb+2SXpkeXIHrE0PrUg9WERZVkeVO3xhoHBshZWqaWZ7vk1CUaWZjaTDb0zFTpDRf5rGVEoZACs8+zdaesHyJVe2517TVm5g3o7IA99RJVtr53cYAACUTKXQwKAAc3gRFAADMAE4QAC2SFyIBwECQHl+CQAOji0PQ4XgmDx+EIGvCkZiaOikNpsewsBSEcjEJU0RjENkwpTWfTaYgjjyWUgAGw0zn2NK8lESpAAdjuCSgpOkzKpiCxAsVDP8ACt1aydQKABxK9h4glErAkwjCjXGzlkXUgFV26UixBmjlIZ0dBKMQ2+uWIACc5v8KpM5PtrOIqNNseDPsQxCEHo1xC1nO5Gb5IeI9P9FpxWDgGMElEEQA
\[\begin{tikzcd}[row sep=normal, column sep=small]
              &                                                                            &                                                             &                                                            & 0     \\
              &                                                                            & 0 \arrow[d]                                                 & H_n(X^{n+1}) \arrow[ru] \arrow[r, "\iso"]                  & H_n X \\
              &                                                                            & H_n(X^n) \arrow[ru, "i"] \arrow[d, "j"]                     &                                                            & {}    \\
{} \arrow[r]  & {H_{n+1}(X^{n+1},X^{n})} \arrow[ru, "\partial_{n+1}"] \arrow[r, "d_{n+1}"] & {H_n(X^n,X^{n-1})} \arrow[d, "\partial_n"] \arrow[r, "d_n"] & {H_{n-1}(X^{n-1},X^{n-2})} \arrow[r, "d_{n-1}"] \arrow[ru] & {}    \\
{} \arrow[ru] &                                                                            & H_{n-1}(X^{n-1}) \arrow[ru, "l"]                            &                                                            &       \\
              & 0 \arrow[ru]                                                               &                                                             &                                                            &      
\end{tikzcd}\]
Since the composition of any two $d_n$ and $d_{n-1}$ factors through the compositions of two successive maps of an exact sequence, this composition is $0$. The horizontal sequence is therefore a chain complex.

We have $$im(d_{n+1})=im(j \partial_{n+1})=im(\partial_{n+1})=ker(i),$$
where the second equality comes from the injectivity of $j$, and the third equality from the exact sequence. Similarly,

$$ker(d_n)=ker(l \partial_n)=ker(\partial_n)=im(j)\iso H_n(X^n).$$
The second equality follows from the injectivity of $l$, the third from the exact sequence, and the fourth from the injectivity of $j$. We therefore have that 
$$ker(d_n)/im(d_{n+1})\iso H_n(X^n)/ker(i).$$ By the first isomorphism theorem, and since $i$ is injective, $$H_n(X^n)/ker(i)\iso im(i)\iso H_n X.$$ 
Therefore $$ker(d_n)/im(d_{n+1})\iso H_nX.$$\cite{Hatcher}\end{proof}

We also have a method of calculating the boundary maps $d_n$ from degree calculations:
\begin{theorem}\label{boundary-formula} Let $(D^n_{\alpha})_{\alpha\in A}$ represent the generating elements of the $\alpha$ $n$-cells. Then 
$d_{n+1}(D^{n+1}_{\alpha})=\sum_{\beta} d_{\alpha,\beta} D^{n}_{\alpha}$, where $d_{\alpha,\beta}$ is the degree of the map $S^{n}_{\alpha}\rightarrow X^n\rightarrow S^n_{\beta}$, that is the composition of the gluing map of $D^n_{\alpha}$ with the quotient map identifying $X^n\setminus int(D_{\beta}^n)$ as a single point. This quotient space is homeomorphic to $S^n$, as $S^n$ is the one-point compactification of $\mathbb{R}^n\iso int(D^n)$.
\end{theorem}
\begin{proof}
Omitted. See \cite{Hatcher}.
\end{proof}

\subsection{Examples}
We now give a few examples of homology groups of CW-complexes. The following corollary becomes very useful.

\begin{corollary}\label{cell-adjacent} If a CW-complex $X$ has no cells in adjacent dimensions, then $H_n X$ is the free abelian group generated by the $n-$cells of $X$.
\end{corollary}
\begin{proof}
Suppose $X$ has $m>0$ $n$-cells. The chain complex reads:

% https://tikzcd.yichuanshen.de/#N4Igdg9gJgpgziAXAbVABwnAlgFyxMJZARgBoAGAXVJADcBDAGwFcYkQAdDgW3pwAsARoOAAtAL4A9biHGl0mXPkIpyFanSat25WfJAZseAkQBM6mgxZtEIXeI0woAc3hFQAMwBOEGYjIgOBBIaprW7FAA+sBgANTE4nqePn6hQUjmYdq2UYQO4kA
\begin{tikzcd}
0 \arrow[r, "d_{n+1}"] & \mathbb{Z}^m \arrow[r, "d_n"] & 0
\end{tikzcd}

It follows that $H_n(X)=Ker(d_n)/Im(d_{n+1})=\mathbb{Z}^m$.
\end{proof}

\begin{proposition}
If $X$ has only one $0$-cell, then $d_1=0$.
\end{proposition}
\begin{proof}
The inclusion $j:H_1(X^1)\rightarrow H_1(X^1,\bullet)$ is an isomorphism by prop \ref{reduced-homology}. By exactness, $\partial_1=0,$ so $d_1=l\partial_1=0$.
\end{proof}

%\begin{example}[$S^{n}$]
%Cellular homology gives a very easy calculation of the homology groups of $S^{n}$ for $n>1$. $S^n$ has a cell structure of 1 0-cell and 1 n-cell, and therefore has no cells in adjacent dimension. By Corollary \ref{cell-adjacent}, $$H_m S^n=\begin{cases}\mathbb{Z} & m=0,n\\ 0 & \text{otherwise} \end{cases}$$
%\end{example}

\begin{example}[Orientable surface of genus $g$]
$M_g$ is the orientable surface with $g$ "holes". It is known that it can be identified as a quotient of the regular $4g$-sided polygon, where side $i$ is identified with the reflection of side $i+2$ (counted clockwise). In pictures,  $M_1$ is the torus $T^2$, identified as the familiar quotient of the square:

\begin{center}
\begin{tikzpicture}[scale=1]
    \fill [gray!40] (0,0) rectangle (3,3);
    \filldraw [black] (0,0) circle (1.5pt) node [left,scale=1] {$x$} 
    -- (3,0) circle (1.5pt) node [right,scale=1] {$x$} 
    node[currarrow,pos=0.5, xscale=1,sloped,scale=1.5]{}
    node[pos=0.5,below,scale=1]{$a_1$} 
    -- (3,3) circle (1.5pt) node [right,scale=1] {$x$} 
    node[currarrow,pos=0.5, xscale=-1,sloped,scale=1.5]{} node[pos=0.5,right,scale=1]{$a_2$}
    -- (0,3) circle (1.5pt) node [left,scale=1] {$x$} 
    node[currarrow,pos=0.5, xscale=1,sloped,scale=1.5]{}
    node[pos=0.5,above,scale=1]{$a_1$}
    -- (0,0) circle (0pt) 
    node[currarrow,pos=0.5, xscale=1,sloped,scale=1.5]{}
    node[pos=0.5,left,scale=1]{$a_2$};
\end{tikzpicture}
\end{center}

$M_2$ is the similar quotient of the octagon:\begin{center}
\begin{tikzpicture}[scale=1]
    \fill [gray!40] (-1,1+1.41) -- (1,1+1*1.41) -- (1+1*1.41,1) -- (1+1*1.41,-1) -- (1,-1-1.41) -- (-1,-1-1.41) -- (-1-1.41,-1) -- (-1-1.41,1) -- (-1,1+1.41);
    \filldraw [black] (-1,1+1*1.41) circle (1pt) node [left,scale=1] {$x$} 
    -- (1,1+1*1.41) circle (1pt) node [right,scale=1] {$x$} 
    node[currarrow,pos=0.5, xscale=1,sloped,scale=1.5]{}
    node[pos=0.5,above,scale=1]{$a_1$} 
    -- (1+1*1.41,1) circle (1pt) node [right,scale=1] {$x$} 
    node[currarrow,pos=0.5, xscale=1,sloped,scale=1.5]{} node[pos=0.5,right,scale=1]{$a_2$}
    -- (1+1*1.41,-1) circle (1pt) node [right,scale=1] {$x$} 
    node[currarrow,pos=0.5, xscale=-1,sloped,scale=1.5]{}
    node[pos=0.5,right,scale=1]{$a_1$}
    -- (1,-1-1.41) circle (1pt) node [below,scale=1] {$x$} 
    node[currarrow,pos=0.5, xscale=1,sloped,scale=1.5]{}
    node[pos=0.5,right,scale=1]{$a_2$}
    -- (-1,-1-1.41) circle (1pt) node [below,scale=1] {$x$} 
    node[currarrow,pos=0.5, xscale=-1,sloped,scale=1.5]{}
    node[pos=0.5,below,scale=1]{$a_3$}
    -- (-1-1.41,-1) circle (1pt) node [left,scale=1] {$x$} 
    node[currarrow,pos=0.5, xscale=-1,sloped,scale=1.5]{}
    node[pos=0.5,left,scale=1]{$a_4$}
    -- (-1-1.41,1) circle (1pt) node [left,scale=1] {$x$} 
    node[currarrow,pos=0.5, xscale=-1,sloped,scale=1.5]{}
    node[pos=0.5,left,scale=1]{$a_3$}
    -- (-1,1+1.41) circle (0pt)
    node[currarrow,pos=0.5, xscale=-1,sloped,scale=1.5]{}
    node[pos=0.5,left,scale=1]{$a_4$};
\end{tikzpicture}
\end{center}
It is clear that $M_g$ should be given a CW-complex structure with $1$ $0$-cell, $2g$ $1$-cells and $1$ $2-cell$ glued along the concatenation $f=a_1\cdot a_2 \cdot -a_1 \cdot -a_2 \cdot a_3 \cdot \dots \cdot -a_{2g-1} \cdot -a_{2g}$. According to Theorem \ref{boundary-formula}, we should calculate the degree of the composition of $f$ with the map identifying everything but $a_i\setminus \{x\}$. This is the map $a_i\cdot -a_i\homotopic 0$, so $d_2=0$. The chain complex reads:

% https://tikzcd.yichuanshen.de/#N4Igdg9gJgpgziAXAbVABwnAlgFyxMJZABgBpiBdUkANwEMAbAVxiRGJAF9T1Nd9CKAIzkqtRizYAdKQFs6OABYAjZcABanLjxAZseAkQBMo6vWatEIGfKWqNnAHrAjAcy3de+gUQDMp8QtpOQUVNU1tL35DFAAWAPNJKw5OMRgoV3giUAAzACcIWSQyEBwIJBFApPZIkHzCiuoypBMqyxrPOoKixFbmxH82thSdep7B-vih5K4KTiA
\[\begin{tikzcd}
0 \arrow[r] & \mathbb{Z} \arrow[r, "0"] & \mathbb{Z}^{2g} \arrow[r, "0"] & \mathbb{Z} \arrow[r] & 0
\end{tikzcd}\]

Since all maps are $0$, the homology groups can be read as:
$$H_n(M_g)=\begin{cases}\mathbb{Z} & n=0\\
\mathbb{Z}^{2g}& n=1\\
\mathbb{Z} & n=2 \\
0 & \text{otherwise}\end{cases}$$
\end{example}

\begin{example}[Klein bottle]
The Klein bottle $K$ is the following quotient of the square with its interior:

\begin{center}
\begin{tikzpicture}[scale=1]
    \filldraw [black] (0,0) circle (1.5pt) node [left,scale=1] {$x$} 
    -- (3,0) circle (1.5pt) node [right,scale=1] {$x$} 
    node[currarrow,pos=0.5, xscale=1,sloped,scale=1.5]{}
    node[pos=0.5,below,scale=1]{$a$} 
    -- (3,3) circle (1.5pt) node [right,scale=1] {$x$} 
    node[currarrow,pos=0.5, xscale=-1,sloped,scale=1.5]{} node[pos=0.5,right,scale=1]{$b$}
    -- (0,3) circle (1.5pt) node [left,scale=1] {$x$} 
    node[currarrow,pos=0.5, xscale=1,sloped,scale=1.5]{}
    node[pos=0.5,above,scale=1]{$a$}
    -- (0,0) circle (0pt) 
    node[currarrow,pos=0.5, xscale=-1,sloped,scale=1.5]{}
    node[pos=0.5,left,scale=1]{$b$};
\end{tikzpicture}
\end{center}

We can therefore give it a CW-complex structure of $1$  $0$-cell (the point $x$), $2$ $1$-cells corresponding to the paths $a$ and $b$, and one $2$-cell glued along the path $f=a\cdot b\cdot -a \cdot b$. By Theorem \ref{boundary-formula}, we should calculate the degree of $f$ composed with the quotient collapsing respectively $b$ and $a$ to a point. The first of these is $a\cdot -a \homotopic 0$, and the second is $b\cdot b\homotopic z^2$. It follows that $d_2(x)=(0,2x)$. The chain complex reads:
% https://tikzcd.yichuanshen.de/#N4Igdg9gJgpgziAXAbVABwnAlgFyxMJZABgBpiBdUkANwEMAbAVxiRGJAF9T1Nd9CKAIzkqtRizYAdKQFs6OABYAjZcABanLjxAZseAkQBMo6vWatEIGfKWqNnGXlnwABDYUq1m7b30CiAGZTcQtpOU97H24-fkMUABYQ80krDk4xGCgAc3giUAAzACcIWSQyEBwIJBFQ1PZfEGLSmuoqpBM6yxAACjIjAA8ASkbmssRO9sRgrrZ0nTGkGamk2bSuCk4gA
\[\begin{tikzcd}
0 \arrow[r] & \mathbb{Z} \arrow[r, "{(0,2x)}"] & \mathbb{Z}\times \mathbb{Z} \arrow[r, "0"] & \mathbb{Z} \arrow[r] & 0
\end{tikzcd}\]
We simply read off:
$$H_2 X=ker(d_2)=0$$
$$H_1 X= ker(d_1)/im(d_2)\iso (\mathbb{Z}\times \mathbb{Z})/(0\times 2\mathbb{Z})\iso \mathbb{Z}\times \mathbb{Z}/2\mathbb{Z}$$
$$H_0 X= ker(d_0)=0$$
and $H_n X=0$ for all other values of $n$.
\end{example}
