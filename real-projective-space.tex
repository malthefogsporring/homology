\subsection{Real projective space $\RP{n}$}
Now we will look at the homologies of $\RP{n}$ with respect to an arbitrary (ordinary) homology. Recall that the puncture of $D^n$ is homotopy equivalent to $S^{n-1}$. Recall also that the puncture of $\RP{n}$ is homotopy equivalent to $\RP{n-1}$. Now consider the following diagram:

% https://tikzcd.yichuanshen.de/#N4Igdg9gJgpgziAXAbVABwnAlgFyxMJZARgBoAGAXVJADcBDAGwFcYkQAJAfTAAoAdfgCUACsAByAX1KDREgLTFJAShDT0mXPkIoyxanSat23PrLFSZwi5MFwYOALZYwzOAAJBwAFSCVa0g1sPAIiclJ9GgYWNkROHl4AEQA9MFIUsDsHZ1cPL19+f3UQDGDtMIoDaOM40yTU0gBlZOAwRSKDGCgAc3giUAAzACcIRyRwkBwIJDJDGPYAKwDBkbHEWamkACYoo1iQAAtlkGHRpABmGk3EcmLTtcvJ6Zvd+biB4-uLq+eduZqQFg1JRJEA
\begin{tikzcd}
{H_k(D^n,S^{n-1})} \arrow[r] \arrow[r, "f^*"] \arrow[d, "i^*"] & {H_k(\RP{n},\RP{n-1})} \arrow[d, "j^*"]              \\
{H_k(D^n,D^n\setminus \{*\})}                              & {H_k(\RP{n},\RP{n}\setminus \{*\})} \arrow[l, "h^*"]
\end{tikzcd}

$*$ is defined as the center of $D^n$ when considering $\RP{n}$ as the glue of $\RP{n-1}$ and $D^n$. $i^*$ and $j^*$ are the maps induced by the canonical inclusions. By the previous comment, $i$ and $j$ are homotopy equivalences, so $i^*$ and $j^*$ are isomorphisms. $h$ is the isomorphism guaranteed by the Excision Axiom after noting that $(D^n,D^n\setminus \{*\})$ can be found by excising everything but a smaller, closed n-disk in $D^n\subset \RP{n}$ from the other set. This open set then satisfies all the requirements of excision. We can therefore define $f^*$ as the unique isomorphism that makes the diagram commute. By a previous result, we have 

\[H_k(\RP{n},\RP{n-1})=H_k(D^n,S^{n-1})=
\begin{cases} 
      G & k=n \\
      0 & \text{otherwise}
   \end{cases}
\]

From this observation we can deduce a number of facts about $H_k(\RP{n})$.

\begin{lemma}
\label{projective-space-iso}
$H_k(\RP{n})\iso H_k(\RP{n-1})$ whenever $k\neq n,n-1$.
\end{lemma}
\begin{proof}
By assumption, $$H_k(\RP{n},\RP{n-1})=H_{k+1}(\RP{n},\RP{n-1})=0.$$
Therefore, the long exact homology sequence reads as

% https://tikzcd.yichuanshen.de/#N4Igdg9gJgpgziAXAbVABwnAlgFyxMJZABgBpiBdUkANwEMAbAVxiRGJAF9T1Nd9CKAIzkqtRizYAJAPoBrABQAdJQCUACsDABaIZwCUXHiAzY8BIgCZR1es1aIQsxSo1aDR3mYFEAzDfF7Ng5OMRgoAHN4IlAAMwAnCABbJDIQHAgkPWME5KzqDKRLbjjElMRrdMzEX1DOIA
\begin{tikzcd}
0 \arrow[r] & H_k(\RP{n-1}) \arrow[r] & H_k(\RP{n}) \arrow[r] & 0
\end{tikzcd}

which induces the necessary isomorphism.
\end{proof}

\begin{corollary}
\label{homology-large-n}
$H_k(\RP{n})=0$ when $k>n$ and $n\neq 0$.

Additionally, $H_0(\RP{n}=H_0 1:=G$.
\end{corollary}
\begin{proof}
By the previous lemma, we have, for $k>n, n\neq 1$

$$H_k(\RP{n})\iso H_k(\RP{n-1})\iso \dots \iso H_k(\RP{1})\iso H_k(S^1)=0$$

The second result has already been verified for $n=0,1$. For $n>1$ we can use the previous lemma again with $k=0$ to get 

$$H_0(\RP{n})\iso H_0(\RP{n-1})\iso \dots \iso H_0(\RP{1})\iso H_0(S^1)=G$$
\end{proof}

\begin{lemma}
\begin{enumerate}[i]
\item If $H_n(\RP{n})=0$ then $H_{n+1}(\RP{n+1})=G$ and $H_{n}(\RP{n+1})=0$.
\item If $H_n(\RP{n})=G$ and $n>0$, then $H_{n+1}(\RP{n+1})=0$.
\end{enumerate}
\end{lemma}

\begin{proof}
(i) Consider the following section of the long homology sequence for $(\RP{n+1},\RP{n})$:

% https://tikzcd.yichuanshen.de/#N4Igdg9gJgpgziAXAbVABwnAlgFyxMJZABgBpiBdUkANwEMAbAVxiRAAkB9YMAagEYAvgAoAOqIBKABR6CAlCEGl0mXPkIp+5KrUYs2XHgJHjpRoQqUrseAkQBM26vWatEHbnyFjJMr0tM-eUVlEAwbdSIAZiddVwNOMB8zMGCrMNVbDWQAFliXfXcuJMDzNNDwtTsUAFZ8vTcPHgBab1L-Una0nRgoAHN4IlAAMwAnCABbJDIQHAgkIVCxyYXqOaR7dOWpxEdZ+cQorfGdmP2kHOOVxDzzxBrBCkEgA
\begin{tikzcd}
H_{n+1}(\RP{n}) \arrow[r] & H_{n+1}(\RP{n+1}) \arrow[r] & {H_{n+1}(\RP{n+1},\RP{n})} \arrow[r] & H_n(\RP{n}) \arrow[r] & H_n(\RP{n+1}) \arrow[r] & {H_{n-1}(\RP{n+1},\RP{n})}
\end{tikzcd}

By assumption, \ref{homology-large-n} and \ref{projective-space-iso}, this reduces to:

% https://tikzcd.yichuanshen.de/#N4Igdg9gJgpgziAXAbVABwnAlgFyxMJZABgBpiBdUkANwEMAbAVxiRGJAF9T1Nd9CKAIzkqtRizYAJAPrAwAaiGcAFAB01AJQAK8pZwCUXHiAzY8BIgCZR1es1aIQAcWO9zAogGZb4h2w5ud35LFAAWX3tJJ1kwdS1dRWUjINM+C0FkAFZIiUd2LjEYKABzeCJQADMAJwgAWyQyEBwIJGUTGvq26hakK1TOhsQbZtbELwHaoZ9RpDDJrsQI2cQszgpOIA
\begin{tikzcd}
0 \arrow[r] & H_{n+1}(\RP{n+1}) \arrow[r] & G \arrow[r] & 0 \arrow[r] & H_n(\RP{n+1}) \arrow[r] & 0
\end{tikzcd}

This induces the two required isomorphisms.

(ii) In this case, the same diagram reduces to

% https://tikzcd.yichuanshen.de/#N4Igdg9gJgpgziAXAbVABwnAlgFyxMJZABgBpiBdUkANwEMAbAVxiRGJAF9T1Nd9CKAIzkqtRizYAJAPrAwAaiGcAFAB01AJQAK8pZwCUXHiAzY8BIgCZR1es1aIQAcWO9zAogGZb4h21dud35LFAAWX3tJJ1kwdS1dRWUjINM+C0FkAFZIiUd2LjEYKABzeCJQADMAJwgAWyQyEBwIJGUTGvq26hakK1TOhsQbZtbELwHaoZ9RpDDJrsQI2cQszgpOIA
\begin{tikzcd}
0 \arrow[r] & H_{n+1}(\RP{n+1}) \arrow[r] & G \arrow[r] & G \arrow[r] & H_n(\RP{n+1}) \arrow[r] & 0
\end{tikzcd}

If I can show the map $\partial: G\rightarrow G$ is injective, I have my result. (... MISSING)

\end{proof}