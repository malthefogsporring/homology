\subsection{Real projective space $\RP{n}$}
Cellular homology makes the full calculation of $\RP{n}$ very easy. We use the cell structure with $1$ $k$-cell for $0\leq k\leq n$, where the $k-th$ glue map is projection onto $\RP{k-1}$.

\begin{proposition}
$$H_k(\RP{n})=\begin{cases}\mathbb{Z} & k=0 \\
\mathbb{Z} &k=n \text{ odd}\\
\mathbb{Z}/2\mathbb{Z} & 0<k<n \text{ and } k \text{ odd.} \end{cases}$$
\end{proposition}
\begin{proof}
By theorem (MISSING), the degree of $d_{n+1}:(\RP{n+1},\RP{n})\rightarrow (\RP{n},\RP{n-1})$ is the degree of the composition

% https://tikzcd.yichuanshen.de/#N4Igdg9gJgpgziAXAbVABwnAlgFyxMJZABgBpiBdUkANwEMAbAVxiRAGUA9QgX1PUy58hFAEZyVWoxZsAOrIBKABWBgeIPgOx4CRAEwTq9Zq0QduGyTCgBzeEVAAzAE4QAtkjIgcEJOKkmbGga-CAu7n7UPkgGATJmAI6WPEA
\begin{tikzcd}
S^n \arrow[r, "p"] & \RP{n} \arrow[r, "q"] & S^n
\end{tikzcd}
where $p$ is the projection map and $q$ is the quotient map $\RP{n}\rightarrow \RP{n}/\RP{n-1}\homeo S^{n}$. 

Note that the map applies the identity on the upper hemisphere and the antipodal map on the lower hemisphere, then identifies the equator. The pre-image of a neighbourhood of the North pole $N$ is two neighbourhoods of $N$ and $S$. The neighbourhood near $N$ is mapped via the identity, and the neighbourhood near $S$ is mapped via the antipodal map. By the local degree proposition (MISSING), $deg(d_{n+1})=deg(qp)=1+(-1)^{n+1}=\begin{cases}2 & n \text{ odd}\\0 & n\text{ even}\end{cases}$

The cellular chain complex for $\RP{n}$ is therefore

% https://tikzcd.yichuanshen.de/#N4Igdg9gJgpgziAXAbVABwnAlgFyxMJZARgBoAGAXVJADcBDAGwFcYkQAdDgW3pwAsARoOAAtAL4hxpdJlz5CKchWp0mrduSkyQGbHgJEATCpoMWbRJx58hIidtn6FRAMym1F9l14DhYyWkneUMUABYPcw0rLigIHAQg3TkDRWQAVkj1S2tfOwDHZOdQ5AA2LK8Ymz97QJ09ELSAdgrokC1xVRgoAHN4IlAAMwAnCG4kMhAcCCRyJJGx2ZpppBNPNqNChfHENZXEd3Wcjp1tpAipmcRMo-ZN+dGdm-3y26sTocekQ-2LqJz7qcvohXvsmp1xEA
\begin{tikzcd}
0 \arrow[r] & \mathbb{Z} \arrow[r, "2"] & \mathbb{Z} \arrow[r, "0"] & \mathbb{Z} \arrow[r, "2"] & \dots \arrow[r, "2"] & \mathbb{Z} \arrow[r, "0"] & \mathbb{Z} \arrow[r] & 0
\end{tikzcd}

when $n$ is even, and

% https://tikzcd.yichuanshen.de/#N4Igdg9gJgpgziAXAbVABwnAlgFyxMJZARgBoAGAXVJADcBDAGwFcYkQAdDgW3pwAsARoOAAtAL4hxpdJlz5CKchWp0mrduSkyQGbHgJEATCpoMWbRJx58hIidtn6FRAMym1F9l14DhYyWkneUMUABYPcw0rLigIHAQg3TkDRWQAVkj1S2tfOwDHZOdQ5AA2LK8Ymz97QJ09ELSAdgrokC1xVRgoAHN4IlAAMwAnCG4kMhAcCCRyJJGx2ZpppBNPNo6dBfHENZXEd3Wco0LtpAipmcRMo-YT+dGdm-3y26tNocekQ-2LqJyPiAzohXvsmp1xEA
\begin{tikzcd}
0 \arrow[r] & \mathbb{Z} \arrow[r, "0"] & \mathbb{Z} \arrow[r, "2"] & \mathbb{Z} \arrow[r, "0"] & \dots \arrow[r, "2"] & \mathbb{Z} \arrow[r, "0"] & \mathbb{Z} \arrow[r] & 0
\end{tikzcd}

when $n$ is odd. The result can be read off directly.

\cite{Hatcher}
\end{proof}
