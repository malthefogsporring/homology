\section{Complex projective space}
The complex projective space, $\CP{n}$ is defined similarly to $\RP{n}$ as the space of complex lines in $\mathbb{C}^{n+1}$. Explicitly it is the quotient $\mathbb{C}^{n+1}\setminus \{0\} / \tilde$ where $z\tilde \lambda w$ for $\lamba \in \mathbb{C}$.

It is trivial to see that $\CP{0}=\cdot$, as any $z\tilde 1$ via multiplication by $\frac{1}{z}$. To understand $\CP{1}$, we can first see it as a quotient of $S^3$, thought of as sitting inside $\mathbb{C}^2$. This follows from the identification $$
\begin{bmatrix} z\\ w \end{bmatrix}\tilde \big|\begin{bmatrix} z\\ w \end{bmatrix}\big| \begin{bmatrix} z\\ w \end{bmatrix}:=\begin{bmatrix} \bar{z}\\ \bar{w} \end{bmatrix}$$
Here we are using the typical norm $$\big|\begin{bmatrix} x+iy\\ a+ib \end{bmatrix}\big|=\big| \begin{bmatrix} a\\ b\\ c\\ d \end{bmatrix}\big|$$
Next note that each $\begin{bmatrix} \bar{z}\\ \bar{w} \end{bmatrix}\in S^3$ is equivalent to exactly all elements in a circle $S^1\in S^3$. This is because $\{\lambda \begin{bmatrix} \bar{z}\\ \bar{w} \end{bmatrix}, |\lambda|=1\}$ is the set of all rotations of  $\begin{bmatrix} \bar{z}\\ \bar{w} \end{bmatrix}$ along some direction.

Consider the case $w\neq 0$. Then, by multiplying by $\frac{\bar{w}^*}{|\bar{w}|}$ we can choose a unique representative $\begin{bmatrix} \tilde{z}\\ \tilde{w} \end{bmatrix}:= \frac{\bar{w}^*}{|\bar{w}|}\begin{bmatrix}\bar{z}\\ \bar{w} \end{bmatrix}$ such that $\tilde{w}> 0$. This can then be identified as an element of the (open) upper disk of the equator $S^2\in S^3$, as the only restriction on $\tilde{z}$ is that $\big|\begin{bmatrix}\tilde{z}\\ \tilde{w} \end{bmatrix} \big|=1$. For example the element $\begin{bmatrix}\tilde{a} \\ \tilde{w}\end{bmatrix} \in S^2$ where $\tilde{w}>0$ is achieved in the previous construction by setting $z=a/(\frac{w^*}{|w|})$. When $w=0$, $\{\begin{bmatrix} \bar{z}\\ \bar{w} \end{bmatrix}\}=\CP{0}\cong \cdot$. It follows that $\CP{1}$ is the (closed) upper hemisphere of $S^2$, quotient its equator, which is homeomorphic to $S^2$. So as a cell complex, $\CP{1}$ has 1 0-cell, 1 1-cell and 2 2-cells in the usual way for $S^2$.

We can use this cell structure to define $\CP{2}$. As before, we can find a representative $\begin{bmatrix}\bar{z}\\ \bar{w}\\ \bar{u}\\ \end{bmatrix}\in S^5$. For $u=0$ this is $\CP{1}$. For $|u|>0$ we can multiply by $\frac{u^*}{|u|}$ to find a representative in the (open) upper hemisphere of the equator $S^4\in S^5$. As previously, this covers the whole open upper hemisphere of $S^4$, as there are no other restrictions on $\bar{z},\bar{w}$. Therefore $\CP{2}$ has the cell structure of $\CP{1}$ with an additional 4-cell. The gluing map is the projection $S^3\subset \mathbb{C}^2 \rightarrow \CP{1}$. Since $\CP{2}$ has no 3-cells, cellular homology asserts that $H_4(\CP{2})=\mathbb{Z}$. For $n\neq 4$, $H_n(\CP{2})=H_n(\S{2})$, as the cellular homology is untouched here. So $H_n(\CP{2})=\mathbb{Z}$ for $n=0,2,4$ and $0$ otherwise.

The general case can be done by induction.

\begin{theorem}
As a cell complex, $\CP{n}$ has 1 0-cell, 1 1-cell, 2 2-cells, and 1 2m-cell for $1<m\leq n$. As a consequence,
$$H_k(\CP{n})=\begin{cases} 
      \mathbb{Z} & k even, 0\leq (k/2)\leq n \\
      0 & \text{otherwise}
   \end{cases}
$$
\end{theorem}
\begin{proof}
We have proved the case $n=2$. Suppose the statement is true for some $n$. We will then prove it is true for $n+1$. As before, we can identify elements of $\CP{n+1}$ by representatives of $S^{2n+1}$ by dividing by norm. These can furthermore be identified as either elements of $\CP{n}$ as a quotient of the equator of $S^{2n}$ or elements of the open upper $2n$-cell of $S^{2n}$. The gluing map is the projection of $\partial D^{2n}$ onto $\CP{n}$. Therefore $\CP{n+1}$ has the cell structure of $\CP{n}$, with an additionall $2n$-cell. This proves the inductive case.

The statement on homology is then easily achieved by noting that $\CP{n}$ has no adjacent $n-cells$ for $n>2$, whence the homology groups for $n>2$ correspond exactly to the generators on $n-cells$. For $n\leq 2$ the homology groups of $\CP{n}$ are equal to that of $S^2$.
\end{proof}